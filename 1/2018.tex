\bta{2018}

\begin{enumerate}
	%\renewcommand{\labelenumi}{\arabic{enumi}.}
	% A(\Alph) a(\alph) I(\Roman) i(\roman) 1(\arabic)
	%设定全局标号series=example	%引用全局变量resume=example
	%[topsep=-0.3em,parsep=-0.3em,itemsep=-0.3em,partopsep=-0.3em]
	%可使用leftmargin调整列表环境左边的空白长度 [leftmargin=0em]
	\item
	选择题 (本题满分 50 分,每小题 5 分)
\begin{enumerate}
	%\renewcommand{\labelenumi}{\arabic{enumi}.}
	% A(\Alph) a(\alph) I(\Roman) i(\roman) 1(\arabic)
	%设定全局标号series=example	%引用全局变量resume=example
	%[topsep=-0.3em,parsep=-0.3em,itemsep=-0.3em,partopsep=-0.3em]
	%可使用leftmargin调整列表环境左边的空白长度 [leftmargin=0em]
	\item
	函数 $h(x)$ 的定义如下:
	\[
	h(x)=\left\{\begin{array}{ll}
		1 & x \geqslant 1 \\
		x & -1<x<1 \\
		-1 & x \leqslant-1
	\end{array}\right.
	\]
	则,函数 $g(x)=h\left(\sin x^{2}\right)$ 一定是 \xzanswer{A} 
	

\fourchoices
{有界可微函数}
{有界,不一定连续}
{连续,不一定可微}
{有界连续,不一定可微}

\item 
设 $f(x)=\left(x+\frac{1}{x}\right)^{1000}$, 则 $f^{\prime}(x)=$ \xzanswer{B} 


\fourchoices
{$1000\left(x+\frac{1}{x}\right)^{1000}\left(1-\frac{1}{x}\right)$}
{$1000\left(x+\frac{1}{x}\right)^{999}\left(1-\frac{1}{x^{2}}\right)$}
{$\left(x+\frac{1}{x}\right)^{999}\left(1-\frac{1}{x^{2}}\right)$}
{$1000\left(x+\frac{1}{x}\right)^{1000}\left(1-\frac{1}{x^{2}}\right)$}

\item 
求极限 $\lim\limits _{x \rightarrow 0}\left(\frac{\cos x+\sin x}{\cos x-\sin x}\right)^{\frac{1}{x}}=$ \xzanswer{C} 


\fourchoices
{$\pi^{2}$}
{$\pi$}
{$e^{2}$}
{$e$}

\item 
直线 $L_{1}:\left\{\begin{array}{l}x+2 y=1 \\ y+\frac{1}{2} z=2\end{array} \right.$ 与直线 $L_{2}: \frac{x-2}{-1}=\frac{y+1}{-2}=\frac{z-3}{2}$ 所夹的锐角 $\alpha$ 的余弦 $\cos \alpha=$ \xzanswer{D} 

\fourchoices
{$\frac{1}{9}$}
{$\frac{2}{9}$}
{$\frac{1}{3}$}
{$\frac{4}{9}$}

\item 
求极限 $\lim\limits _{n \rightarrow \infty} \frac{\sqrt[n]{n !}}{n}=$ \xzanswer{A} 


\fourchoices
{$e^{-1}$}
{$e^{-2}$}
{$e$}
{$e^{2}$}

\item 
二重积分 $\int_{0}^{+\infty} \int_{0}^{+\infty} x^{2} e^{-2\left(x^{2}+y^{2}\right)^{2}} d x d y=$ \xzanswer{D} 


\fourchoices
{$\frac{\pi}{64}$}
{$\frac{\pi}{32}$}
{$\frac{\pi}{16}$}
{$\frac{\pi}{8}$}

\item 
$ \lim\limits _{(x, y) \rightarrow(0,0)} \frac{|x y|}{\sqrt[4]{x^{6}+y^{18} }}= $ \xzanswer{D} 


\fourchoices
{$ 0 $}
{$ 1 $}
{$ \infty$}
{不存在}

\item 
若常微分方程初值问题: $y^{\prime}+y=x y^{2}, y(0)=\alpha$ 的解 $y^{*}(x)$ 满足 $\lim\limits _{x \rightarrow 1} y^{*}(x)=\frac{1}{e+2}$, 则可知 $\alpha=$ \xzanswer{A} 


\fourchoices
{$\frac{1}{2}$}
{$\frac{1}{3}$}
{$\frac{1}{4}$}
{$\frac{1}{5}$}


\item 
级数 $\sum\limits_{n=0}^{+\infty} \frac{2^{n}(n+1)}{n !}=$ \xzanswer{C} 


\fourchoices
{$e^{2}$}
{$2 e^{2}$}
{$3 e^{2}$}
{$4 e^{2}$}

\item 
设 $f^{\prime}\left(x^{2}\right)=\frac{1}{x}(x>0)$, 则 $f(x)=$ \xzanswer{A} 


\fourchoices
{$2 \sqrt{x}+C$}
{$\sqrt{x}+C$}
{$4 \sqrt{x}+C$}
{$\frac{2}{\sqrt{x}}+C$}

	
	
\end{enumerate}

	


\item 
(本题满分 10 分)
设函数序列 $y_{n}=y_{n}(x)$ 的定义域为 $x>1$, 当 $n \geqslant 1$ 时, 满足如下的迭代关系:
\[
y_{1}(x)=2 x, y_{n+1}(x)=2 x-\frac{1}{y_{n}(x)}
\]
证明: $\lim\limits _{n \rightarrow \infty} y_{n}$ 存在, 并求这个极限。

\banswer{
	证明略。单调递增,有界,$\lim \limits_{n \rightarrow \infty} y_{n}=x+\sqrt{x^{2}-1}$
}


\item 
(本题满分 10 分)
求由平面 $x+y+z=\pm 1,2 x-y+2 z=\pm 2, x-y-z=\pm 3$, 所界的平行六面体 $\Omega$ 的体积 $V$ 。


\banswer{
	变量代换,$ V=8 $
}


\item 
(本题满分 10 分)
把函数 $f(x)=x \cos x$ 在 $\left[-\frac{\pi}{2}, \frac{\pi}{2}\right]$ 展开成傅里叶级数。


\banswer{
	$f(x)=\frac{2}{\pi} \sum\limits_{n=1}^{\infty} (-1)^{n}\left[\frac{1}{(2 n+1)^{2}}-\frac{1}{(2 n-1)^{2}}\right] \sin 2 n x $,端点处收敛于$ 0 $,其它点处收敛于$ f(x) $
}


\item 
(本题满分 10 分)
求抛物线 $y^{2}=x, y^{2}=3 x$ 和直线 $y=x, y=2 x$ 所围区域 $D$ 的面积。


\banswer{
变量代换,$ S= \frac{7}{6}  $
}




\item 
(本题满分 10 分)
由拉格朗日中值公式有: $e^{x}-1=x e^{x \theta(x)}, \theta(x) \in(0,1)$ 。证明: $\lim\limits _{x \rightarrow 0} \theta(x)=\frac{1}{2}$。

\banswer{
	证明略。解出$ \theta(x) $后求极限
}


\newpage
\item 
(本题满分 10 分)
求微分方程 $y^{\prime \prime}-y^{\prime}-2 y=e^{2 t}(3-t)$ 的通解。


\banswer{
	$y=C_{1} e^{2 t}+C_{2} e^{-t}+\left(\frac{10}{9} t-\frac{1}{6} t^{2}\right) e^{2 t}$	
}


\item 
(本题满分 10 分)
计算曲线积分 $I=\oint_{L} \frac{x d y-y d x}{x^{2}+2 y^{2}}$, 其中 $L$ 是以点 $(1,1),(-1,0),(0,-1)$ 为顶点的三 角形,取逆时针方向。

\banswer{
	$ I=\sqrt{2} \pi $
}


\item 
(本题满分 10 分)
计算曲面积分 $\iint_{S} z d S$, 其中 $S$ 为区域 $\sqrt{x^{2}+y^{2}} \leqslant z \leqslant 1$ 的边界。

\banswer{
$ \pi + \frac{2\sqrt{2}}{3} \pi $
}


\item 
(本题满分 10 分)
设函数 $f(x)$ 和 $g(x)$ 在 $[a, b]$ 上连续, 证明:
\[
\left(\int_{a}^{b} f(x) g(x) d x\right)^{2} \leq \int_{a}^{b} f^{2}(x) d x \int_{a}^{b} g^{2}(x) d x
\]

\banswer{
	证明略。柯西不等式的积分形式,可以将积分化成和式,利用柯西不等式来证。也可以构造函数
	\[ 
	F(x)= \int_{a}^{x} f^{2}(t) d t \int_{a}^{x} g^{2}(t) d t -\left(\int_{a}^{x} f(t) g(t) d t\right)^{2} 
	 \]
	还可以利用$ \left[  f(x)g(y) -f(y)g(x) \right]^{2} \geq 0 $,展开后对$ x $、$ y $积分可证。
}


\item 
(本题满分 10 分)
设函数 $f(x)$ 在 $[0,2]$ 上连续可微, 且 $f(0)=f(2)=0$ 。 证明:
\[
\int_{0}^{2}\left|f(x) f^{\prime}(x)\right| d x \leq \frac{1}{2} \int_{0}^{2}\left|f^{\prime}(x)\right|^{2} d x
\]


\banswer{
	证明略。利用$ F(x)=\int_{0}^{x} |f(t)|dt $的导数$ F ^{\prime} (t)=|f(t)| $
	\[ 
	\int_{0}^{1}\left|f(x) f^{\prime}(x)\right| d x = \int_{0}^{1}\left| \int_{0}^{x} f ^{\prime} (t)dt \cdot f^{\prime}(x)\right| d x \leq \int_{0}^{1} \int_{0}^{x}  \left| f ^{\prime} (t)dt \right| \cdot \left| f^{\prime}(x)\right| d x 
	 \]
	并且需注意$ 	\left| f(t) \right| ^{2}  =2f(t)\cdot f ^{\prime} (t)  \neq 2 | f(t) | \cdot | f ^{\prime} (t) |  $
}	
	
\end{enumerate}


