\bta{2022}






\begin{enumerate}
	%\renewcommand{\labelenumi}{\arabic{enumi}.}
	% A(\Alph) a(\alph) I(\Roman) i(\roman) 1(\arabic)
	%设定全局标号series=example	%引用全局变量resume=example
	%[topsep=-0.3em,parsep=-0.3em,itemsep=-0.3em,partopsep=-0.3em]
	%可使用leftmargin调整列表环境左边的空白长度 [leftmargin=0em]
	\item
	\begin{enumerate}
		%\renewcommand{\labelenumi}{\arabic{enumi}.}
		% A(\Alph) a(\alph) I(\Roman) i(\roman) 1(\arabic)
		%设定全局标号series=example	%引用全局变量resume=example
		%[topsep=-0.3em,parsep=-0.3em,itemsep=-0.3em,partopsep=-0.3em]
		%可使用leftmargin调整列表环境左边的空白长度 [leftmargin=0em]
		\item
		$f(x)=\arcsin \left(\frac{4}{\pi} x-1\right)$, 求 $f[f(x)]$ 的定义域 \xzanswer{A} 
		
		
		
	\fourchoices
	{$\left[\frac{\pi}{4}, \frac{\pi}{2}\right]$}
	{$\left[\frac{\pi}{4}, \frac{\pi}{2}\right]$}
	{$\left[\frac{\pi}{8}, \frac{\pi}{4}\right]$}
	{$\left[0, \frac{\pi}{2}\right]$}
	

	\item 
	求极限 $\lim\limits _{x \rightarrow 0} \tan ^{2} x\left(\frac{1}{\sin x}+\frac{1}{x^{2}}\right)=$ \xzanswer{B} 
	
	
	
\fourchoices
{$ 0 $}
{$ 1 $}
{$ 2 $}
{不存在}

\item 
若函数 $g(x)$ 二阶可导, 且满足 $g(0)=g^{\prime}(0)=0, f(x)=\left\{\begin{array}{cl}\frac{g(x)}{x} & x \neq 0 \\ 0 & x=0\end{array}\right.$, 讨论$f(x)$ 和 $f^{\prime}(x)$ 在 $x=0$ 处的连续性 \xzanswer{A} 

	
\fourchoices
{$f(x)$ 连续, $f^{\prime}(x)$ 连续}
{$f(x)$ 连续, $f^{\prime}(x)$ 不连续, $f^{\prime}(0)$ 存在}
{$f(x)$ 连续, $f^{\prime}(0)$ 不存在}
{$f(x)$ 不连续	}


\item 
$f(x)=\lim\limits _{n \rightarrow+\infty} \frac{e^{x}+x^{2 n}}{2-x^{2 n}}$, 则 $f(x)$ 的间断点为 \xzanswer{C} 




\fourchoices
{$-1$}
{$ 1 $}
{$ \pm 1 $}
{无间断点}


\item 
直线: $\frac{x-1}{2}=\frac{y-2}{-2}=\frac{z-3}{-1}$ 和直线 $\left\{\begin{array}{l}x+2 y-3=0 \\ 3 y+z+2=0\end{array}\right.$ 间夹角的余弦值 $\cos \theta$为 \xzanswer{D} 



\fourchoices
{$\frac{1}{\sqrt{8}}$}
{$\frac{1}{\sqrt{10}}$}
{$\frac{1}{\sqrt{12}}$}
{$\frac{1}{\sqrt{14}}$}

\item 
正项级数 $\sum\limits_{n=1}^{+\infty} a_{n}$ 收敛, 则下列级数:
①$\sum\limits_{n=1}^{+\infty}(-1)^{n} a_{n}$,
②$\sum\limits_{n=1}^{+\infty}\left(a_{n}\right)^{2}$,
③$\sum\limits_{n=1}^{+\infty} \sin \left(a_{n}\right)$,
④$\sum\limits_{n=1}^{+\infty} \ln \left(1+a_{n}\right)$
中一定收敛的级数个数为 \xzanswer{D} 



\fourchoices
{$ 1 $}
{$ 2 $}
{$ 3 $}
{$ 4 $}


\item 
判断曲线: $y=x \sin \frac{1}{x}$ 的渐近线 \xzanswer{B} 

\fourchoices
{只有垂直渐近线}
{只有水平渐近线}
{有垂直渐近线和水平渐近线}
{只有斜渐近线}


\item 
函数 $f(x, y)=\frac{x^{2} y^{2}}{x^{2} y^{2}+(x-y)^{2}}$, 则 $f(x, y)$ 在点 $(0,0)$ 处 \xzanswer{B} 

\fourchoices
{累次极限都存在, 重极限存在}
{累次极限都存在, 重极限不存在}
{累次极限存在其中一个, 重极限不存在}
{至少一个累次极限不存在}

\item 
函数 $z(x, y)$ 满足 $\frac{x}{z}=\ln \frac{z}{y}$, 则 $d z=$ \xzanswer{B} 

\fourchoices
{$\frac{z(z d x+y d y)}{z(x+y)}$}
{$\frac{z(y d x+z d y)}{y(x+z)}$}
{$ \frac{z(z d x+x d y)}{y(x+z)}$}
{$\frac{z(x d x+y d y)}{z(x+y)}$}

\item 
下列广义积分收敛的是 \xzanswer{B} 


\fourchoices
{$\int_{0}^{+\infty} \frac{1}{x\left(e^{\frac{x}{2}}-1\right)} d x$}
{$\int_{0}^{1}(\ln x)^{100} d x$}
{$ \int_{2}^{+\infty} \frac{1}{x \ln x} d x$}
{$\int_{0}^{+\infty} \frac{1}{1+x+\sin x} d x$}







\end{enumerate}



\item 
$u(x)=\left\{\begin{array}{ll}1,& x>0 \\ 0,& x \leq 0\end{array}\right.$, 求极限 $\lim\limits _{x \rightarrow 0}\left[\frac{1+e^{\frac{1}{x}}}{1-e^{\frac{2}{x}}}+u(x)\right]$。

\banswer{
	$ 1 $
}


\item 
函数 $y(x)$ 满足微分方程 $y^{\prime \prime}=e^{2 y}+e^{y}$, 且 $y(0)=0, y^{\prime}(0)=2$, 求 $y(x)$。


\banswer{
	$y=\ln \frac{e^{x}}{2-e^{x}}$
}


\item 
设函数 $y(x)$ 满足参数方程 $\left\{\begin{array}{l}x=1+t^{2} \\ y=\cos t\end{array}\right.$,求下列极限:
\begin{enumerate}
	%\renewcommand{\labelenumi}{\arabic{enumi}.}
	% A(\Alph) a(\alph) I(\Roman) i(\roman) 1(\arabic)
	%设定全局标号series=example	%引用全局变量resume=example
	%[topsep=-0.3em,parsep=-0.3em,itemsep=-0.3em,partopsep=-0.3em]
	%可使用leftmargin调整列表环境左边的空白长度 [leftmargin=0em]
	\item

 $ \frac{d y}{d x}$和$ \frac{d^{2} y}{d x^{2}}$;
\item 
$\lim\limits _{x \rightarrow 1^{+}} \frac{d y}{d x} $和$ \lim\limits _{x \rightarrow 1^{+}} \frac{d^{2} y}{d x^{2}}$。
\end{enumerate}

\banswer{
	\begin{enumerate}
		%\renewcommand{\labelenumi}{\arabic{enumi}.}
		% A(\Alph) a(\alph) I(\Roman) i(\roman) 1(\arabic)
		%设定全局标号series=example	%引用全局变量resume=example
		%[topsep=-0.3em,parsep=-0.3em,itemsep=-0.3em,partopsep=-0.3em]
		%可使用leftmargin调整列表环境左边的空白长度 [leftmargin=0em]
		\item
		$-\frac{\sin t}{2 t}$和$-\frac{t \cos t-\sin t}{4 t^{3}}$
		
		\item 
		$-\frac{1}{2}$和$ \frac{1}{12}$
		
	\end{enumerate}
	
	
}



\item 
证明: $u(x, y)=f(x) g(y)$ 成立的充分必要条件是 $u \frac{\partial^{2} u}{\partial x \partial y}=\frac{\partial u}{\partial x} \frac{\partial u}{\partial y}$。


\banswer{
	证明略
}


\item 
计算半圆 $(x-1)^{2}+y^{2}=1$,$ y>0$, 绕 $y$ 轴旋转所得到的旋转体的体积。


\banswer{
	$ \pi^{2} $
}


\item 
曲线 $C: \frac{x^{2}}{4}+\frac{y^{2}}{9}=1$, 计算积分
\[ I=\oint_{C} e^{x y}\{[y \sin (x y)+\cos (x+y)] d x+[x \sin (x y)+\cos (x+y)] d y\} \]

\banswer{
	$ 0 $
}


\newpage
\item 
\begin{enumerate}
	%\renewcommand{\labelenumi}{\arabic{enumi}.}
	% A(\Alph) a(\alph) I(\Roman) i(\roman) 1(\arabic)
	%设定全局标号series=example	%引用全局变量resume=example
	%[topsep=-0.3em,parsep=-0.3em,itemsep=-0.3em,partopsep=-0.3em]
	%可使用leftmargin调整列表环境左边的空白长度 [leftmargin=0em]
	\item
	在 $[-\pi, \pi]$ 上把函数 $f(x)=x$ 展开为傅里叶级数;
	\item 
	证明: $\sum\limits_{n=0}^{+\infty} \frac{(-1)^{n}}{2 n+1}=\frac{\pi}{4}$。
	
\end{enumerate}


\banswer{
	\begin{enumerate}
		%\renewcommand{\labelenumi}{\arabic{enumi}.}
		% A(\Alph) a(\alph) I(\Roman) i(\roman) 1(\arabic)
		%设定全局标号series=example	%引用全局变量resume=example
		%[topsep=-0.3em,parsep=-0.3em,itemsep=-0.3em,partopsep=-0.3em]
		%可使用leftmargin调整列表环境左边的空白长度 [leftmargin=0em]
		\item
		$f(x)=\sum\limits_{n=1}^{+\infty} \frac{2(-1)^{n+1}}{n} \sin n x$
		\item 
		证明略
		
	\end{enumerate}
	
	
}





\item 
若$f(x)$和$ g(x)$ 在 $[a, b]$ 上满足二阶可导, 并且有 $g^{\prime \prime}(x) \neq 0$,$ f(a)=f(b)=g(a)=g(b)=0$,试证明:
\begin{enumerate}
	%\renewcommand{\labelenumi}{\arabic{enumi}.}
	% A(\Alph) a(\alph) I(\Roman) i(\roman) 1(\arabic)
	%设定全局标号series=example	%引用全局变量resume=example
	%[topsep=-0.3em,parsep=-0.3em,itemsep=-0.3em,partopsep=-0.3em]
	%可使用leftmargin调整列表环境左边的空白长度 [leftmargin=0em]
	\item
	$ \forall x \in(a, b)$,$  g(x) \neq 0 $;
	\item 
	$ \exists \xi \in(a, b)$,$ \frac{f(\xi)}{g(\xi)}=\frac{f^{\prime \prime}(\xi)}{g^{\prime \prime}(\xi)}$。
	
	
	
\end{enumerate}

\banswer{
	证明略
}



\item 
设$I_{n}=\int_{0}^{1} \frac{x^{n}}{1+x} d x$,证明:
\begin{enumerate}
	%\renewcommand{\labelenumi}{\arabic{enumi}.}
	% A(\Alph) a(\alph) I(\Roman) i(\roman) 1(\arabic)
	%设定全局标号series=example	%引用全局变量resume=example
	%[topsep=-0.3em,parsep=-0.3em,itemsep=-0.3em,partopsep=-0.3em]
	%可使用leftmargin调整列表环境左边的空白长度 [leftmargin=0em]
	\item %\label{2022-10-01}
	$I_{n+1}=-I_{n}+\frac{1}{n+1}$;
	\item 
	$\lim\limits _{n \rightarrow+\infty} I_{n}=0$;
	\item 
	用 $(1)$  和 $(2)$ 的结论证明 $\sum\limits_{n=1}^{+\infty} \frac{(-1)^{n-1}}{n}=\ln 2$。
	
\end{enumerate}

\banswer{
	证明略
}





\item 
函数 $f(x, y, z)=x^{2}+y^{2}+z^{2}$, 求其在满足条件 $a x+b y+c z=1$ 下的最小值。
\banswer{
	$\frac{1}{a^{2}+b^{2}+c^{2}}$
}

	
\end{enumerate}

