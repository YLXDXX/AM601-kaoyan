\bta{2019}

\begin{enumerate}
	%\renewcommand{\labelenumi}{\arabic{enumi}.}
	% A(\Alph) a(\alph) I(\Roman) i(\roman) 1(\arabic)
	%设定全局标号series=example	%引用全局变量resume=example
	%[topsep=-0.3em,parsep=-0.3em,itemsep=-0.3em,partopsep=-0.3em]
	%可使用leftmargin调整列表环境左边的空白长度 [leftmargin=0em]
	\item
选择题 (本题满分 50 分,每小题 5 分)
\begin{enumerate}
	%\renewcommand{\labelenumi}{\arabic{enumi}.}
	% A(\Alph) a(\alph) I(\Roman) i(\roman) 1(\arabic)
	%设定全局标号series=example	%引用全局变量resume=example
	%[topsep=-0.3em,parsep=-0.3em,itemsep=-0.3em,partopsep=-0.3em]
	%可使用leftmargin调整列表环境左边的空白长度 [leftmargin=0em]
	\item
求极限:$ \lim \limits_{n \rightarrow \infty}\left[\left(1+\dfrac{\mathstrut 1}{\mathstrut 2!}+\dfrac{1}{3 !}+\dots+\dfrac{1}{n!}\right)+\left(\dfrac{1}{1\cdot3}+\dfrac{1}{3 \cdot 5}+\dfrac{1}{(2 n-1)(2 n+1)}\right)\right] $ \xzanswer{A}


\fourchoices
{$ e-1/2 $}
{$ 5/2 $}
{$ e+1/2 $}
{$ 7/2 $}




\item 
已知函数$ f(x) $在$ x=0 $处连续,且$\lim\limits _{h \rightarrow 0} \frac{f\left(h^{2}\right)}{h^{2}}=1$,则下列正确的是 \xzanswer{C} 


\fourchoices
{$ f(0)=0 $ 且$ f_- ^{\prime} (0) $存在}
{$ f(0)=1 $且$ f_- ^{\prime} (0) $存在}
{$ f(0)=0 $ 且$ f_+ ^{\prime} (0) $存在}
{$ f(0)=1 $ 且$ f_+ ^{\prime} (0) $存在}


\item 
以下$ 4 $个命题中正确的是 \xzanswer{D} 


\fourchoices
{若$ f ^{\prime} (x) $在$ (0,1) $内连续,则$ f(x) $在$ (0,1) $内有界}
{若$ f(x) $在$ (0,1) $内连续,则$ f(x) $在$ (0,1) $内有界}
{若$ f(x) $在$ (0,1) $内有界,则$ f ^{\prime} (x) $在$ (0,1) $内有界}
{若$ f ^{\prime} (x) $在$ (0,1) $内有界,则$ f(x) $在$ (0,1) $内有界}



\item 
设$ f(x)=\left\{\begin{array}{cl}{\dfrac{\mathstrut x \cdot \sin \left(x^{2}-1\right)}{\mathstrut x^{2}-1}} & {x \neq \pm 1} \\ \mathstrut {1} & {x=\pm 1}\end{array}\right. $,则下列说法正确的是 \xzanswer{B}


\fourchoices
{$ f(x) $在$ x=-1 $连续,在$ x=1 $不连续}
{$ f(x) $在$ x=-1 $不连续,在$ x=1 $连续}
{$ f(x) $在$ x=-1 $和$ x=1 $都连续}
{$ f(x) $在$ x=-1 $和$ x=1 $都不连续}



\item 
若向量$ \vec{a},\vec{b} $满足$ (\vec{a}+\vec{b}) \perp(\vec{a}-\vec{b}) $、$ (5\vec{a}+\vec{b}) \perp (3 \vec{a}-4 \vec{b}) $,则$\vec{a}$与$\vec{b}$的夹角$  \alpha  $满足 \xzanswer{D} 

\fourchoices
{$\cos \alpha=\frac{14}{17}$}
{$\cos \alpha=\frac{13}{17}$}
{$\cos \alpha=\frac{12}{17}$}
{$\cos \alpha=\frac{11}{17}$}







\item 
已知$ f(x, y)=\left\{\begin{array}{cl}{\dfrac{\mathstrut x^{2}+y^{2}}{\mathstrut \left|x|+y^{2}\right.},} & {x^{2}+y^{2} \neq 0} \\ \mathstrut {0} & {x^{2}+y^{2}=0}\end{array}\right. $,则关于累次极限$ I=\lim\limits _{x \rightarrow 0} \lim\limits _{y \rightarrow 0} f(x, y) $和$ J=\lim\limits _{y \rightarrow 0 } \lim\limits_{x \rightarrow 0}f(x, y) $的下列说法正确的是 \xzanswer{C} 


\fourchoices
{$I$ 存在但 $J$ 不存在}
{$J$ 存在但 $I$ 不存在}
{$I$ 和 $J$ 都存在但 $I \neq J$}
{$I$ 和 $J$ 都存在且 $I=J$}



\item 
已知数项级数$ \sum\limits_{n=0}^{\infty} a_{n} $收敛,则下列级数
$$
\sum_{n=0}^{\infty}(-1)^{n} a_{n}, \quad \sum_{n=0}^{\infty}\left(a_{n}\right)^{2}, \quad  \sum_{n=0}^{\infty} \sin \left(a_{n}\right),  \quad \sum_{n=0}^{\infty} \ln \left(1+a_{n}\right)
$$
中收敛级数的数量是 \xzanswer{A} 


\fourchoices
{$ 1 $}
{$ 2 $}
{$ 3 $}
{$ 4 $}

\item 
设函数$ f(t) $二次连续可微,令$ u=f(xy) $,则$ \phi(t)=\dfrac{\partial^{2} u}{\partial x \partial y} $的表达式是 \xzanswer{C}


\fourchoices
{$t f^{\prime \prime}(t)-f^{\prime}(t)$}
{$f^{\prime \prime}(t)-t f^{\prime}(t)$}
{$ t f^{\prime \prime}(t)+f^{\prime}(t)$}
{$ f^{\prime \prime}(t)+t f^{\prime}(t)$}


\item 
设非齐次线性微分方程$ y ^{\prime} +P(x)y=Q(x) $有两个不同的解$y_{1}(x), y_{2}(x)$,$ C $为任意常数,则该方程的通解是 \xzanswer{B} 


\fourchoices
{$C\left[y_{1}(x)-y_{2}(x)\right]$}
{$y_{1}(x)+C\left[y_{1}(x)-y_{2}(x)\right]$}
{$C\left[y_{1}(x)+y_{2}(x)\right]$}
{$y_{1}(x)+C\left[y_{1}(x)+y_{2}(x)\right]$}

\item 
考虑积分$ I=\iint_{D}\left(x^{2}-y^{2}\right) d x d y $,其中$ D $是椭圆:$ \{(x, y) \in R^{2} | \dfrac{x^{2}}{a^{2}}+\dfrac{y^{2}}{b^{2}}<1\} $落在第一象限中的部分。则下列说法正确的是 \xzanswer{A}


\fourchoices
{$ a>b $时$ I>0 $}
{$ a>b $时$ I<0 $}
{$ a>b $时$ I=0 $}
{$ a<b $时$ I=0 $}




\end{enumerate}


\item 
(本题满分 10 分)
求常微分方程:$yy''+(y')^2=0$满足初始条件:$y(0)=1,y'(0)= \frac{ 1 }{ 2 } $的特解。


\banswer{
$y^{2}=x+1$
}


\item 
(本题满分 10 分)
求直线$ L $:$ \frac{x-1}{2}=\frac{y+1}{-1}=\frac{z}{1} $绕$ z $轴旋转一周所生成的旋转曲面的方程。


\banswer{
	$ x^{2}+y^{2}=(2z+1)^{2} + (z+1)^{2} =5z^{2}+6z+2$
}


\item 
(本题满分 10 分)
设$ a_0=3,a_1=5 $,且对任何自然数$ n>1 $有$ n a_{n}=\frac{2}{3} a_{n-1}-\left(n-1\right)a_{n-1} $。
证明:当$ |x|<1 $时,幂级数$ \sum\limits_{n=0}^{\infty} a_{n} x^{n} $
绝对收敛,并求其和函数$ S(x) $。


\banswer{
	证明略,$S(x)=\frac{15}{2}(1+x)^{\frac{2}{3}}-\frac{9}{2}$
}


\item 
(本题满分 10 分)
已知函数$ y=y(x) $由方程:$ e^{y}+6 x y+x^{2}-1=0 $确定,试求$ y^{\prime\prime} (0) $。

\banswer{
	$ y^{\prime\prime} (0) =-2 $。从$ y(0) $开始,再$ y ^{\prime} (0) $,一步步来。
}

\newpage
\item 
(本题满分 10 分)
对实数$ R>0 $定义积分:
\[ 
I_R=\iiint\limits_{1/R \leq x^2+y^2+z^2 \leq R}\dfrac{e^{-(x^2+y^2+z^2)}}{x^2+y^2+z^2}dxdydz
 \]
证明极限$ I=\lim \limits _{R \rightarrow+\infty} I_{R} $存在并计算其值。

\banswer{
	证明略,单调增加有上界。$ \lim\limits_{R \rightarrow +\infty} I_{R} =2\pi^{3/2} $
}


\item 
(本题满分 10 分)
计算积分
\[ 
I=\oint\limits_{L} \frac{\left(x^{2}-y^{2}-x\right) d y+(1-2 x) y d x}{\left(x^{2}+y^{2}\right)\left[(x-1)^{2}+y^{2}\right]}
 \]
其中$ L=\left\{(x, y) \in R^{2} \big| x^{2}+y^{2}=4\right\} $沿逆时针方向。

\banswer{
	$ I=0 $。这道题推荐直接用参数化算,也可以设
	\[ 
	P=\frac{(1-2 x) y }{\left(x^{2}+y^{2}\right)\left[(x-1)^{2}+y^{2}\right]}  \quad 
	Q=\frac{\left(x^{2}-y^{2}-x\right) }{\left(x^{2}+y^{2}\right)\left[(x-1)^{2}+y^{2}\right]} 
	 \]
有$ \frac{\partial Q}{ \partial x} -\frac{\partial P}{\partial y}=0 $,挖点计算。
}


\item 
(本题满分 10 分)
已知$ f(x)=\sin \frac{1}{x} $,试求一个序列$ x_{n} \rightarrow 0 \ (n \rightarrow \infty) $使得$ \lim\limits _{n \rightarrow \infty} f\left(x_{n}\right) $存在,且满足
$ \lim\limits _{x \rightarrow 0}\left[\lim\limits _{n \rightarrow \infty} f\left(x_{n}\right)+x\right]^{\left[ 2/f(\frac{1}{x}) \right]}=e^{2} $

\banswer{
其中$  \lim\limits _{n \rightarrow \infty} f\left(x_{n}\right) =1 $,答案不唯一,可简单取为	$x_{n}=\frac{1}{2 n\pi +\frac{1}{2} \pi}$
}


\item 
(本题满分 10 分)
设函数$ f(x) $在$[a, b]$上$ 2 $阶可导,其中$ a<b $。且有$ f^{\prime}\left(\frac{a+b}{2}\right)=0 $。
证明:存在$ \xi	\in(a,b) $使得$\left|f^{\prime \prime}(\xi)\right| \geq \frac{4}{(b-a)^{2}}|f(b)-f(a)|$。


\banswer{
	证明略
}


\item 
(本题满分 10 分)
证明:
\begin{enumerate}
	%\renewcommand{\labelenumi}{\arabic{enumi}.}
	% A(\Alph) a(\alph) I(\Roman) i(\roman) 1(\arabic)
	%设定全局标号series=example	%引用全局变量resume=example
	%[topsep=-0.3em,parsep=-0.3em,itemsep=-0.3em,partopsep=-0.3em]
	%可使用leftmargin调整列表环境左边的空白长度 [leftmargin=0em]
	\item
闭区间$ [0,1] $上任何连续函数都有原函数。

\item
闭区间$ [0,1] $上任何连续可微函数都可以写成两个单调不减函数之差。
\end{enumerate}

\banswer{
\begin{enumerate}
	%\renewcommand{\labelenumi}{\arabic{enumi}.}
	% A(\Alph) a(\alph) I(\Roman) i(\roman) 1(\arabic)
	%设定全局标号series=example	%引用全局变量resume=example
	%[topsep=-0.3em,parsep=-0.3em,itemsep=-0.3em,partopsep=-0.3em]
	%可使用leftmargin调整列表环境左边的空白长度 [leftmargin=0em]
	\item
	证明略。原函数$ F(x)=\int_{0}^{x} f(x) dx $
	\item 
	%疑是题目有误
	存疑
\end{enumerate}

	
}


\item 
(本题满分 10 分)
求函数$ f(x, y, z)=\ln x+2 \ln y+3 \ln z $在球面$ x^{2}+y^{2}+z^{2}=6 r^{2} $上的最大值,并证明对任意正实数$ a,b,c $,不等式$ a b^{2} c^{3} \leq 108\left(\frac{a+b+c}{6}\right)^6 $恒成立。

\banswer{
证明略	。第一步易证,第二步,对任意三个正数$ a,b,c $构建一个点$ (\sqrt{a},\sqrt{b},\sqrt{c}) $,调整$ r $,让$ (\sqrt{a},\sqrt{b},\sqrt{c}) $落在球面上。然后用第一步的结论
\[ 
\ln x+2 \ln y+3 \ln z \leqslant \ln ( 6 \sqrt{3} r^{6} )
 \]
化简后平方,再利用加$ x^{2}+y^{2}+z^{2}=6 r^{2} $可证。
}
	
	
	
\end{enumerate}

