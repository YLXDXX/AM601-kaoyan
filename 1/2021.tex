\bta{2021}



\begin{enumerate}
	%\renewcommand{\labelenumi}{\arabic{enumi}.}
	% A(\Alph) a(\alph) I(\Roman) i(\roman) 1(\arabic)
	%设定全局标号series=example	%引用全局变量resume=example
	%[topsep=-0.3em,parsep=-0.3em,itemsep=-0.3em,partopsep=-0.3em]
	%可使用leftmargin调整列表环境左边的空白长度 [leftmargin=0em]
	\item
	选择题 (本题满分 50 分,每小题 5 分)
\begin{enumerate}
	%\renewcommand{\labelenumi}{\arabic{enumi}.}
	% A(\Alph) a(\alph) I(\Roman) i(\roman) 1(\arabic)
	%设定全局标号series=example	%引用全局变量resume=example
	%[topsep=-0.3em,parsep=-0.3em,itemsep=-0.3em,partopsep=-0.3em]
	%可使用leftmargin调整列表环境左边的空白长度 [leftmargin=0em]
	\item
	$a=e^{x}, \quad b=1+x, \quad c=1+x+x^{2}$, 则在 $x=0$ 的$ \epsilon $无穷小邻域内, 大小估计正确的为 \xzanswer{A} 
	
\fourchoices
{$ b \leq a \leq c$}
{$a \leq b \leq c$}
{$b \leq c \leq a$}
{无法确定}

\item 
双曲正弦, $\sin h x=\frac{e^{x}-e^{-x}}{2} $, 双曲余泫 $\cosh x=\frac{e^{x}+e^{-x}}{2}=$ 错误的是	 \xzanswer{B} 


\fourchoices
{双曲正弦为奇函数,双曲余弦为偶函数}
{双曲正弦为增函数,双曲余弦为减函数}
{双曲正弦的导为双曲余弦,双曲余弦的导为双曲正弦}
{$\cosh ^{2} x-\sinh ^{2} x=1$, 对住意 $x$ 总成立}

\item 
$n$ 为正数, $\lim\limits _{n \rightarrow \rightarrow \infty} \cos 2 \pi \sqrt{n^{2}+n}=$ \xzanswer{C} 

\fourchoices
{$ 1 $}
{$ 0 $}
{$-1$}
{不存在}

\item 
设函数 $f(x)$ 是定义在 $(-1,1)$ 内的奇函数, 且 $\lim _{x \rightarrow 0^{+}} \frac{f(x)}{x}=a \neq 0$, 则 $f(x)$ 在 $x=0$ 处导数为 \xzanswer{A} 

\fourchoices
{$a$}
{$-a$}
{$ 0 $}
{不存在}

\item 
设向是 $\vec{a}$ 和 $\vec{b}$ 满足 $(\vec{a}+\vec{b}) \perp(\vec{a}-2 \vec{b}), \quad(3 \vec{a}-\vec{b}) \perp(\vec{a}+2 \vec{a})$, 则 $|\vec{a}|$ 与 $|\vec{b}|$ \xzanswer{D} 

\fourchoices
{$|\vec{a}|=\sqrt{\frac{1}{2}} |\vec{b}|$}
{$|\vec{a}|=\sqrt{2}|\vec{b}|$}
{$|\vec{a}|=\sqrt{\frac{2}{3}} |\vec{b}|$}
{$|\vec{a}|=\sqrt{\frac{3}{2}} |\vec{b}|$}

\item 
$I=\lim\limits _{x \rightarrow 0} \dfrac{\int_{\sin ^{3} x \cos x}^{e^{x^{2}}-1} \arctan \frac{3 t}{2+t} d t}{\arcsin x^{3}}$, 则 \xzanswer{D} 

\fourchoices
{$I$ 不存在}
{$I=3 / 2$}
{$I=1 / 2$}
{$I=0$}

\item 
$a_{n}>0,\left\{a_{n}\right\}$ 单调递减值趋于$  0  $, 则 $\sum\limits_{n=1}^{\infty}(-1)^{n-1} \sqrt{a_{n} \cdot a_{n-1}}$ \xzanswer{C} 
 

\fourchoices
{发散}
{绝对收敛}
{条件收敛}
{无法判断}

\item 
设函数 $Z=x y+x F\left(\frac{\mathrm{y}}{x}\right)$, 其中 $\mathrm{F}$ 为可导函数, 则 $x \frac{\partial z}{\partial x}+y \frac{\partial z}{\partial y}$ 的表达式是 \xzanswer{C} 

\fourchoices
{$Z-x y$}
{$ 0 $}
{$Z+x y$}
{$x y$}

\item 
$y^{\prime \prime}-6 y^{\prime}+8 y=e^{x}+e^{2 x}$ 的一个特解形式 \xzanswer{B} 


\fourchoices
{$a e^{x}+b e^{2 x} $}
{$ a e^{x}+b x e^{2 x} $}
{$a x e^{x}+b e^{2x} $}
{$ a x e^{x}+b x e^{2 x}$}

\item 
正确的是 \xzanswer{A} 


\fourchoices
{$\lim\limits _{s \rightarrow 0} \iint_{s <x^{2}+y^{2}< \frac{ 1 }{ 2 } } \frac{d x d y}{\left(x^{2}+y^{2}\right)\left(\ln \sqrt{x^{2}+y^{2}}\right)^{2}}$存在}
{$\lim\limits _{s \rightarrow 1} \iint_{-\frac{1}{2}<x^{2}+y^{2}<s} \frac{d x d y}{\left(x^{2}+y^{2}\right)\left(\ln \sqrt{x^{2}+y^{2}}\right)^{2}} $存在}
{$\lim\limits _{s \rightarrow 0} \iint_{s <x^{2}+y^{2}< \frac{ 1 }{ 2 } } \frac{\left(1+x^{2}\right) d x d y}{\left(x^{2}+y^{2}\right)\left(\ln \sqrt{x^{2}+y^{2}}\right)^{2}}$存在}
{以上均不对}




\end{enumerate}

	

\item 
(本题满分 10 分)
求常微分方程 $x y^{\prime \prime}-y^{\prime} \ln y^{\prime}+y^{\prime} \ln x=0$, 满足 $y(1)=2$ 和 $y^{\prime}(1)=e^{2}$ 的特解。

\banswer{
	$y=(x-1) e^{x+1}+2$
}



\item 
(本题满分 10 分)
求过直线 $l_{1}: \frac{x-1}{2}=\frac{y}{1}=\frac{z+1}{3}$ 且平行于 $l_{2}:\left\{\begin{array}{l}2 x+y-z+1=0 \\ x-2 y+z-2=0\end{array}\right.$ 的平面方程。

\banswer{
	$4(x-1)+7 y-5(z+1)=0$
}



\item 
(本题满分 10 分)
设 $f(x)=\lim _{n \rightarrow+\infty} \frac{x^{2 n-1}+n x \sin \frac{x}{n}}{x^{2 n}+1}$, 讨论 $f(x)$ 连续性。

\banswer{
	当 $|x|<1$ 时,连续;当 $|x|=1$ 时,也连续。处处连续。
}


\item 
(本题满分 10 分)
设 $g(x)=\left\{\begin{array}{cc}\frac{e^{x}-1}{x} & x \neq 0 \\ 1 & x=0\end{array} \right.$,$f(x)=g^{\prime}(x)$。求 $f^{(n)}(0)$。

\banswer{
	$f^{(n)}(0)=\frac{1}{n+1}-\frac{1}{(n+1)(n+2)}=\frac{1}{n+2}$
}


\newpage
\item 
(本题满分 10 分)
设曲面 $\Sigma=\left\{(x, y, z) \in R^{3} \mid x^{2}+y^{2}+z^{2}=1\right\}$ 的外侧, 计算积分
\[ 
\iint_{\Sigma} y^{2} z d x d y+x^{2} y d z d x
 \]

\banswer{
	$\frac{3}{5} \pi$
}



\item 
(本题满分 10 分)
$\lim _{x \rightarrow 0} \frac{a x+\sin x}{\int_{b}^{x} \frac{\ln \left(1+t^{3}\right)}{t} d t}=c,(c \neq 0)$, 求出 $a, b, c$ 的值。

\banswer{
	$a=-1, b=0, c=-\frac{1}{2}$
}



\item 
(本题满分 10 分)
设二元函数 $f(x, y)=\left\{\begin{array}{ll}\frac{\sqrt{|x y|}}{x^{2}+y^{2}} \sin \left(x^{2}+y^{2}\right) & x^{2}+y^{2} \neq 0 \\ 0 & x^{2}+y^{2}=0\end{array}\right.$,讨论$ f(x,y) $在$ (0,0) $处的可微性。	

\banswer{
	$f(x, y)$ 在 $(0,0)$ 点处不可微
}




\item 
(本题满分 10 分)
$f(x)=\left\{\begin{array}{cc}x & -\pi \leq x \leq 0 \\ 0 & 0 \leq x \leq \pi\end{array}\right.$展开为傅里叶函数。


\banswer{
	$f(x)=\frac{\pi}{4}+\sum\limits_{n=1}^{\infty} \frac{\left[1-(-1)^{n}\right]}{n^{2} \pi} \cos x+\frac{(-1)^{n+1}}{n} \sin x$
}


\item 
(本题满分 10 分)
设函数$ f(x) $在$ [a,b] $上连续,在$ ab $内可导,证明:如果$ f(x) $为非线性函数,
则存在 $\exists \ \xi_{1}, \xi_{2} \in(a, b)$ 使$f^{\prime}(\xi_{1})=\frac{f(b)-f(a)}{b-a}$,$f^{\prime}\left(\xi_{2}\right)<\frac{f(b)-f(a)}{b-a}$。

\banswer{
	证明略
}



\item 
(本题满分 10 分)
已知函数 $z=f(x, y)$ 的全微分 $d z=2 x d x-2 y d y$ 并且 $f(1,1)=2020$; 求出
$f(x, y)$ 在椭球体域 $D=\left\{(x, y) \mid x^{2}+\frac{y^{2}}{4} \leq 1\right\}$ 上最大值和最小值。

\banswer{
	当 $y=0$ 时取,最大值 $f(\pm 1,0)=2021$;
	当 $y=\pm 2$ 时,取最小值 $f(0, \pm 2)=2016$。
}



	
	
	
\end{enumerate}


