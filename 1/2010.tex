\bta{2010}

\begin{enumerate}
	%\renewcommand{\labelenumi}{\arabic{enumi}.}
	% A(\Alph) a(\alph) I(\Roman) i(\roman) 1(\arabic)
	%设定全局标号series=example	%引用全局变量resume=example
	%[topsep=-0.3em,parsep=-0.3em,itemsep=-0.3em,partopsep=-0.3em]
	%可使用leftmargin调整列表环境左边的空白长度 [leftmargin=0em]
	\item
	选择题 (本题满分 40 分,每小题 5 分)
	
\begin{enumerate}
%\renewcommand{\labelenumi}{\arabic{enumi}.}
% A(\Alph) a(\alph) I(\Roman) i(\roman) 1(\arabic)
%设定全局标号series=example	%引用全局变量resume=example
%[topsep=-0.3em,parsep=-0.3em,itemsep=-0.3em,partopsep=-0.3em]
%可使用leftmargin调整列表环境左边的空白长度 [leftmargin=0em]
\item
已知 $\lim\limits _{x \rightarrow 0} \frac{a \tan x+b(1-\cos x)}{c \ln (1-2 x)+d\left(1-e^{-x^{2}}\right)}=2$, 其中 $a^{2}+c^{2}\neq 0$, 则必有 \xzanswer{D}

	
\fourchoices
{$b=4 d$}
{$b=-4 d$}
{$a=4 c$}
{$a=-4 c$}

\item 
$y=y(x)$ 满足 $\Delta y=\frac{1-x}{\sqrt{2 x-x^{2}}} \Delta x+o(\Delta x), y(1)=1, \int_{0}^{1} y(x)=$ \xzanswer{D} 


\fourchoices
{$2 \pi$}
{$\pi$}
{$\frac{\pi}{2}$}
{$\frac{\pi}{4}$}

\item 
设函数 $f(x)=\int_{0}^{x} t^{2}(t-1) d t$, 则 $f(x)$ 有多少个极值点 \xzanswer{B} 


\fourchoices
{$ 0 $}
{$ 1 $}
{$ 2 $}
{$ 3 $}


\item 
下列命题正确的一项是 \xzanswer{C} 


\fourchoices
{若 $\sum\limits_{n=1}^{\infty} a_{n}, \sum\limits_{n=1}^{\infty} b_{n}$ 都收敛,则 $\sum\limits_{n=1}^{\infty} a_{n} b_{n}$ 收敛}
{若 $\sum\limits_{n=1}^{\infty} a_{n}$ 收敛, $\sum\limits_{n=1}^{\infty} b_{n}$ 发散,则 $\sum\limits_{n=1}^{\infty} a_{n} b_{n}$ 发散}
{若 $\sum\limits_{n=1}^{\infty} a_{n}$ 收敛, $\sum\limits_{n=1}^{\infty} b_{n}$ 绝对收敛,则 $\sum\limits_{n=1}^{\infty} a_{n} b_{n}$ 绝对收敛}
{若 $\sum\limits_{n=1}^{\infty} a_{n}$ 收敛, $\sum\limits_{n=1}^{\infty} b_{n}$ 条件收敛,则 $\sum\limits_{n=1}^{\infty} a_{n} b_{n}$ 条件收敛}

\begin{note}
	\textcolor{red}{上述的关系可以通过举反例排除,各种关系及其严格的说明那位同学能发一下?}
\end{note}


\item 
设 $a_{0}, a_{1}, a_{2},a_{3} \ldots$ 是等差数列, 且公差 $d>0$, 则幂级数 $\sum\limits_{n=0}^{\infty} a_{n} x^{n}$ 的收敛域为 \xzanswer{C}

	

\fourchoices
{$(-d, d)$}
{$[-d, d)$}
{$(-1,1)$}
{$[-1,1)$}


\item 
设 $y_{1}(x), y_{2}(x), y_{3}(x)$ 是二阶线性非齐次微分方程 $y^{\prime \prime}+p(x) y^{\prime}+q(x) y=f(x)$ 的三 个线性无关的解, $C_{1}, C_{2}$ 是两个任意常数, 则微分方程的通解为 \xzanswer{B} 

\fourchoices
{$C_{1} y_{1}(x)+C_{2} y_{2}(x)+C_{3} y_{3}(x)$}
{$C_{1} y_{1}(x)+C_{2} y_{2}(x)+\left(1-C_{1}-C_{2}\right) y_{3}(x)$}
{$C_{1} y_{1}(x)+C_{2} y_{2}(x)-\left(C_{1}+C_{2}\right) y_{3}(x)$}
{$C_{1} y_{1}(x)+C_{2} y_{2}(x)-\left(1-C_{1}-C_{2}\right) y_{3}(x)$}

\item 
通过两个平面 $2 x+y-4=0$ 与 $y+2 z=0$ 的交线及点 $M_{0}(2,-1,-1)$ 的平面方程为 \xzanswer{A} 


\fourchoices
{$3 x+y-z=6$}
{$x+3 y-z=0$}
{$3 x-y+z=6$}
{$x-3 y-z=6$}

\item 
曲线 $y=e^{x}$ 和该曲线经过原点的切线以及 $y$ 轴所围成的面积 \xzanswer{A} 


\fourchoices
{$\frac{e}{2}-1$}
{$\frac{e}{2}+1$}
{$\frac{e}{2}$}
{$e+1$}


	
\end{enumerate}


\item 
(本题满分 12 分)
设 $g(x)$ 是以正数 $T$ 为周期的连续函数, $g(0)=1, f(x)=\int_{0}^{2 x}|x-t| g(t) d t$, 求 $f^{\prime}(T)$ 。

\banswer{
	$f^{\prime}(T)=2T$
}



\item 
(本题满分 12 分)
已知平面过点 $ (1,2,3) $,它在 $ x $ 正半轴和 $ y $ 轴正半轴上的截距相等,问当该平面在
三个坐标轴上的截距分别为何值时,它与三个坐标轴所围成的立体体积取极值,并写出此时的平面方程。


%群友反馈:原题目有问题,最小值为零,两种改法:①将$ z $轴的截距限制为正;②将最值改为极值。


\banswer{
	$ x $轴、$ y $轴、$ z $轴截距分别为$  \frac{ 2 }{ 9 }  $、$  \frac{ 2 }{ 9 }  $、$ 9 $时取极值,曲面方程为:$ \frac{x}{\frac{9}{2}}+\frac{y}{\frac{9}{2}}+\frac{z}{9}=1 $。
}


\item 
(本题满分 12 分)
求下面微分方程的通解:
\[ 
y y^{\prime \prime}-\left(y^{\prime}\right)^{2}=y^{2} \ln y
 \]
注: $\int \frac{d x}{\sqrt{x^{2} + a^{2}}}=\ln \left(x+\sqrt{x^{2} + a^{2}}\right)$+C。


\banswer{
	通解为:$\ln \left(\ln y+\sqrt{\ln ^{2} y+C_{1}}\right)=x+C_{2}$ $ (C_{1}>0) $
\begin{note}
	作代换$ y ^{\prime} =P(x) $后,得到$ yPP ^{\prime} -P^{2}=y^{2}\ln y $,若直接运用特解加通解的方法,将得到错误答案$ P=y(\ln y + C_{1}) $,需利用$ u=P^{2} $将方程化成线性微分方程。
\end{note}
}


\item 
(本题满分 12 分)
将函数 $f(x)=\frac{x-1}{(x+1)^{2}}$ 在 $x=0$ 处展开成幂级数, 并求收敛区间。


\banswer{
$f(x)=\sum\limits_{n=0}^{\infty}(-1)^{n+1} \cdot (2n+1) \cdot x^{n}$,收敛区间为$ (-1,1) $
}


\item
(本题满分 12 分)
在椭球面 $2 x^{2}+2 y^{2}+z^{2}=1$ 上求一点, 使函数 $f(x, y, z)=x^{2}+y^{2}+z^{2}$ 在该点, 沿 方向 $\vec{l}=\vec{i}-\vec{j}$ 的方向导数最大。


\banswer{
$ \frac{\partial f}{\partial \vec{l}} $在点$ ( \frac{ 1 }{ 2 } ,- \frac{ 1 }{ 2 } ,0) $处取得最大值,最大值为$\sqrt{2}$	
}

\newpage
\item 
(本题满分 14 分)
设 $f(x)$ 为定义在 $[0,+\infty)$ 上的连续函数,且满足
\[ 
f(t)=\iiint\limits_{x^{2}+y^{2}+z^{2} \leqslant t^{2}} f\left(\sqrt{x^{2}+y^{2}+z^{2}}\right) d v+t^{3}
 \]
求 $f(x)$ 的表达式。

\banswer{
	$f(x)=\frac{3}{4\pi} (e^{\frac{4}{3} \pi x^{3}} -1 )$
}



\item 
(本题满分 12 分)
设 $u=u\left(\sqrt{x^{2}+y^{2}}\right)$ 具有连续的二阶偏导, 且满足 
\[ 
\frac{\partial^{2} u}{\partial x^{2}}+\frac{\partial^{2} u}{\partial y^{2}}-\frac{1}{x} \frac{\partial u}{\partial x}+u=x^{2}+y^{2}
 \]
 试求 $u$ 的表达式。

\banswer{
	$u(t)=c_{1} \sin t+c_{2} \cos t+t^{2}-2$
}

\item 
(本题满分 12 分)
设函数 $f(x)$ 具有连续导数,在区域 $G$ 内曲线积分
\[ 
\int_{M}^{N} \frac{1}{2 x^{2}+f(y)}(y d x-x d y)
 \]
与路径无关,其中 $G$ 不包含原点的单连通区域, $M 、 N$ 是 $G$ 内的任意两点, 且 $f(1)=1$ 。
\begin{enumerate}
	%\renewcommand{\labelenumi}{\arabic{enumi}.}
	% A(\Alph) a(\alph) I(\Roman) i(\roman) 1(\arabic)
	%设定全局标号series=example	%引用全局变量resume=example
	%[topsep=-0.3em,parsep=-0.3em,itemsep=-0.3em,partopsep=-0.3em]
	%可使用leftmargin调整列表环境左边的空白长度 [leftmargin=0em]
	\item
 求 $f(x)$;
\item 
求 $\oint_{\Gamma} \frac{1}{2 x^{2}+f(y)}(y d x-x d y)$ 其中 $\Gamma$ 为闭区间 $x^{\frac{2}{3}}+y^{\frac{2}{3}}=a^{\frac{2}{3}}$, 取逆时针方向。
\end{enumerate}

\banswer{
\begin{enumerate}
	%\renewcommand{\labelenumi}{\arabic{enumi}.}
	% A(\Alph) a(\alph) I(\Roman) i(\roman) 1(\arabic)
	%设定全局标号series=example	%引用全局变量resume=example
	%[topsep=-0.3em,parsep=-0.3em,itemsep=-0.3em,partopsep=-0.3em]
	%可使用leftmargin调整列表环境左边的空白长度 [leftmargin=0em]
	\item
	$f(x)=x^{2}$
	\item 
	$-\sqrt{2} \pi$
	
\end{enumerate}


}


\item 
(本题满分 10 分)
设 $f(x)$ 在 $[0,1]$ 上具有二阶连续导数, 且 $f(0)=f(1)=0, f(x)$ 不恒为零, 求证:
$$
\int_{0}^{1}\left|f^{\prime \prime}(x)\right| d x \geqslant 4 \max _{0 \leqslant x \leqslant 1}|f(x)|
$$


\banswer{
	证明略
}


\item 
(本题满分 10 分)
一点从坐标原点出发向东移动 $ a \ \mathrm{m} $,然后左拐弯移动 $ aq \ \mathrm{m} $(其中 $ 0<q<1 $),此后反
复左拐弯前行,使得后一段移动为前一段的 $ q $ 倍,该点这样运动下去,有一极限位置,
求该极限位置距离原点距离。


\banswer{
	设$  \vv{n}_{0}=(1,q)  $,每四步下一点:\\
	$  \vv{OA}_{1}=a\cdot (q^{4}) ^{0}\cdot(1-q^{2})\cdot \vv{n}_{0}   $,\\
	$  \vv{A_{1}A_{2}}=a\cdot (q^{4}) ^{1}\cdot(1-q^{2})\cdot \vv{n}_{0}   $\\
	$  \vv{A_{2}A_{3}}=a\cdot (q^{4}) ^{2}\cdot(1-q^{2})\cdot \vv{n}_{0}   $\\
	得到$\lim\limits_{n\rightarrow\infty}  \vv{OA_{n}}=a\cdot \frac{1-q^{2}}{1-q^{4}} \cdot |\vv{n}_{0}|    =\frac{a}{\sqrt{1+q^{2}}} $
}




\end{enumerate}

