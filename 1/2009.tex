\bta{2009}

\begin{enumerate}
	%\renewcommand{\labelenumi}{\arabic{enumi}.}
	% A(\Alph) a(\alph) I(\Roman) i(\roman) 1(\arabic)
	%设定全局标号series=example	%引用全局变量resume=example
	%[topsep=-0.3em,parsep=-0.3em,itemsep=-0.3em,partopsep=-0.3em]
	%可使用leftmargin调整列表环境左边的空白长度 [leftmargin=0em]
	\item
	填空题 (本题满分 30 分,每小题 6 分)
	\begin{enumerate}
		%\renewcommand{\labelenumi}{\arabic{enumi}.}
		% A(\Alph) a(\alph) I(\Roman) i(\roman) 1(\arabic)
		%设定全局标号series=example	%引用全局变量resume=example
		%[topsep=-0.3em,parsep=-0.3em,itemsep=-0.3em,partopsep=-0.3em]
		%可使用leftmargin调整列表环境左边的空白长度 [leftmargin=0em]
		\item
		设 $y=\sqrt[4]{x \sqrt[3]{e^{x} \sqrt{\sin \frac{1}{x}}}}$, 则 $\frac{d y}{d x}=$  \tk{$y\left(\frac{1}{4 x}+\frac{1}{12}-\frac{1}{24} \cdot \frac{1}{x^{2}} \cdot \tan \frac{1}{x}\right)$} 。
        \nynote{对数求导法,这里我算的好几次都是$y\left(\frac{1}{4 x}+\frac{1}{12}-\frac{1}{24} \cdot \frac{1}{x^{2}} \cdot \cot \frac{1}{x}\right)$,不知是否答案有误。}
		\item  
设 $z=\frac{1}{x} f(x y)+y g(x+y)$, 其中 $f, g$ 具有二阶连续导数, 则 $\frac{\partial^{2} z}{\partial x^{2}}=$  \tk{$\frac{2}{x^{3}} f(x y)-\frac{2 y}{x^{2}} f^{\prime}(x y)+$ $\frac{y^{2}}{x} f^{\prime \prime}(x y)+y g^{\prime \prime}(x+y)$}  。
		
		
		\item
		函数 $f(x)=\frac{1}{x^{2}-2 x-3}$ 展开为 $x$ 的幂级数, 展开式为: \tk{$-\frac{1}{12} \sum\limits_{n=0}^{\infty}\left(\frac{x}{3}\right)^{n}-\frac{1}{4} \sum\limits_{n=0}^{\infty}(-1)^{n} x^{n}$} 。



\item
过点 $P(-1,0,4)$ 且与平面 $3 x-4 y+z+10=0$ 平行, 又与直线 $L: \frac{x+1}{1}=$ $\frac{y-3}{1}=\frac{z}{2}$ 相交的直线方程为:  \tk{$\frac{x+1}{16}=\frac{y}{19}=\frac{z-4}{28}$}  。
\begin{note}
	可能有问题,后期重算
\end{note}
\nynote{
    在L上取一点$Q(x,y,z)$,有$\mathbf{QP}$与平面法向量垂直,可解出$Q$点,即得直线方程,
    这里我算的是$\frac{x+1}{8}=\frac{y}{17}=\frac{z-4}{4}$。
}

\item
微分方程 $y y^{\prime\prime}+\left(y^{\prime}\right)^{2}=0$ 满足初始条件 $\left.y\right|_{x=0}=1,\left.y^{\prime}\right|_{x=0}=\frac{1}{2}$ 的解为:  
\tk{
$y^{2}=x+1$
} 。
		
	
		
	\end{enumerate}
	
\item 
选择题 (本题满分 30 分,每小题 6 分)
\begin{enumerate}
	%\renewcommand{\labelenumi}{\arabic{enumi}.}
	% A(\Alph) a(\alph) I(\Roman) i(\roman) 1(\arabic)
	%设定全局标号series=example	%引用全局变量resume=example
	%[topsep=-0.3em,parsep=-0.3em,itemsep=-0.3em,partopsep=-0.3em]
	%可使用leftmargin调整列表环境左边的空白长度 [leftmargin=0em]
	\item
	$ \lim\limits _{x \rightarrow 2} \frac{\int_{2}^{x}\left(\int_{t}^{2} e^{-u^{2}} du \right) d t}{(x-2)^{2}}=$ \xzanswer{D} 
	
	
\fourchoices
{$\frac{1}{e^{2}}$}
{$-\frac{1}{e^{2}}$}
{$\frac{1}{2 e^{4}}$}
{$-\frac{1}{2 e^{4}}$}

	
	
	\item
	求积分 $\int_{0}^{1} e^{\sqrt{1-x}} d x=$ \xzanswer{C} 
	
	
\fourchoices
{$ 0 $}
{$ 1 $}
{$ 2 $}
{$ 3 $}


\item
设在 $[0,+\infty)$ 上 $f^{\prime \prime}(x)>0$, 则当 $x \in(0,+\infty)$ 时, 下列不等式成立的是 \xzanswer{B} 


\fourchoices
{$f^{\prime}(0) x<f(0)-f(x)<f^{\prime}(x) x$}
{$f^{\prime}(0) x<f(x)-f(0)<f^{\prime}(x)x$}
{$f(0)-f(x)>f^{\prime}(0) x>f^{\prime}(x)x$}
{$f(0)-f(x)<f^{\prime}(0) x<f^{\prime}(x)x$}




\item
设 $L: x^{2}+(y+1)^{2}=2$ 取逆时针方向, 则 $\oint_{L} \frac{x d y-y d x}{x^{2}+(y+1)^{2}}=$ \xzanswer{B} 


\fourchoices
{$4 \pi$}
{$2 \pi$}
{$\pi^{2}$}
{$2 \pi^{2}$}



\item
设 $\Sigma$ 为曲面 $z=x^{2}+y^{2}(0 \leqslant z \leqslant 1)$ 的下侧, 则曲面积分 $I=\iint\limits_{\Sigma} y^{3} d z d x+(y+z) d x d y=$ \xzanswer{C} 


\fourchoices
{$-\frac{\pi}{2}$}
{$\frac{\pi}{2}$}
{$-\frac{\pi}{4}$}
{$\frac{\pi}{4}$}

\nynote{这里我用定义法和补顶面再高斯定理的方法算出来都是A.$-\pi/2$,不知是否答案有误。}

\end{enumerate}


\item 
(本题满分 10 分)
设 $\lim\limits _{x \rightarrow \infty}\left(\frac{x-a}{x+a}\right)^{x}=\int_{a}^{+\infty} x e^{-2 x} d x$, 求 $a$ 的值。



\banswer{
	$a=\frac{3}{2}$
}


\item 
(本题满分 10 分)
设 $f(x)$ 在 $[0,1]$ 上连续, $f(x)>0$, 求证:存在唯一的 $a \in(0,1)$, 使得 $\int_{0}^{a} f(t) d t=\int_{a}^{1} \frac{1}{f(t)} d t$。

\banswer{
证明略。
}



\item 
(本题满分 10 分)
由直线 $x=0, y=8$ 及抛物线 $y=x^{2}$ 围成一个曲边三角形, 在曲边 $y=x^{2}$ 上求一点 $M(X, Y)$, 使得在该点处的切线与直线 $x=0, y=8$ 所围成的三角形面积最小。

\banswer{
根据对称性,只需要算$ 0<x<2\sqrt{2} $一边即可。
\[ 
S_{\triangle}=\frac{16}{a} + \frac{1}{4} a^{3} +4a
 \]
 $ a^{2}=\frac{8}{3} $时取得极小值。

 $M$坐标为$( \frac{2\sqrt 6}{3}, \frac{8}{3})$或$( -\frac{2\sqrt 6}{3}, \frac{8}{3})$,
 $S_{min} = \frac{64\sqrt{6}}{9}$。
}



\item 
(本题满分 10 分)
求满足 $x=\int_{0}^{x} f(t) d t + \int_{0}^{x} t f(t-x) d t$ 的可微函数 $f(x)$ 。

\nynote{该题在2007年重复出现}
\banswer{
	$f(x)= \cos x - \sin x$。注意,这里会遇到$ f ^{\prime} (x) + f(-x)=0 $这个微分方程,解决方案有两种:①利用上一步条件得到$ f(-x) $的表达式;②利用变量替换$ t=-x $,此时有
	\[ 
	-\frac{d}{dt} f(-t)+f(t)=0
	 \]
相当于$ -\frac{d}{dx} f(-x)+f(x)=0 $。再利用最初的方程得到$ f ^{\prime\prime}(x)+f(x)=0  $,可解。
}


\item 
(本题满分 10 分)
证明方程 $F\left(x+\frac{z}{y}, y+\frac{z}{x}\right)=0$ 所确定的隐函数 $z=z(x, y)$ 满足方程 $x \frac{\partial z}{\partial x}+y \frac{\partial z}{\partial y}=z-x y_{\circ}$。


\banswer{
	证明略。设$ u=x+\frac{z}{y}, v=y+\frac{z}{x} $,求导后利用系数行列式为零可证。
}


\newpage
\item 
(本题满分 10 分)
已知函数 $y_{1}=x e^{x}+e^{2 x}, y_{2}=x e^{x}+e^{-x}, y_{3}=x e^{x}+e^{2 x}-e^{-x}$ 是某二阶常系数线性非齐次微分方程的三个解,求此微分方程的表达式。



\banswer{
	$y^{\prime \prime}-y^{\prime}-2 y=(1-2 x) e^{x}$
}


\item 
(本题满分 10 分)
设 $f(x)$ 是周期为 2 的周期函数,且 $f(x)=\left\{\begin{array}{ll}x & 0 \leqslant x \leqslant 1 \\ 0 & 1<x<2\end{array}\right.$, 试求出 $f(x)$ 的傅里叶级数,
并求数项级数 $\sum\limits_{n=0}^{\infty} \frac{1}{(2 n+1)^{2}}$ 的和。

\nynote{若代入1求解,则要注意$x=1$是一个跳跃间断点,傅里叶展开式在此处的值应该是$\frac 1 2$。}
\banswer{
在$ (-1,1) $上展开得:
\[ 
f(x)= \frac{1}{4} +\sum\limits_{n=1}^{\infty}  \frac{1}{n^{2}\pi^{2}}[ (-1)^{n}-1 ]  \cos n \pi x+\sum\limits_{n=1}^{\infty} \frac{1}{n \pi} (-1)^{n+1} \sin n \pi x
 \]
利用$f(0)$,有$\sum\limits_{n=0}^{\infty} \frac{1}{(2 n+1)^{2}}=\frac{\pi^{2}}{8}$
}



\item 
(本题满分 10 分)
设 $f(x)$ 连续, 令 $F(t)=\iint\limits_{x^{2}+y^{2} \leqslant t^{2}} f\left(x^{2}+y^{2}\right) d x d y$ $(t \geqslant 0)$ 。 求 $F^{\prime \prime}(0)$ 。

\banswer{
	$F^{\prime \prime}(0)=2 \pi f(0)$
}


\item 
(本题满分 10 分)
设 $f(x)$ 在闭区间 $[0,1]$ 上连续, 在开区间 $(0,1)$ 内可导,且 $f(0)=0$,$ f(1)=1$ 。试证明 任意给定的正数 $a, b$ 在开区间 $(0,1)$ 内存在不同的实数 $\xi$ 和 $\eta$, 使得 
\[ 
\frac{a}{f^{\prime}(\xi)}+\frac{b}{f^{\prime}(\eta)}=a+b
 \]

 \nynote{啊,我没想到,我是菜逼。$\_(:\text{3~」}\angle)\_$}
	
\banswer{
	证明略。由于$ 0 \leq \frac{a}{a+b} \leq 1 $,可令$ f(\epsilon)=\frac{a}{a+b} $,运用拉格朗日中值定理可证。
}
	






	
\end{enumerate}


