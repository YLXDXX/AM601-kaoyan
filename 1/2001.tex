\bta{2001}


\begin{enumerate}
	%\renewcommand{\labelenumi}{\arabic{enumi}.}
	% A(\Alph) a(\alph) I(\Roman) i(\roman) 1(\arabic)
	%设定全局标号series=example	%引用全局变量resume=example
	%[topsep=-0.3em,parsep=-0.3em,itemsep=-0.3em,partopsep=-0.3em]
	%可使用leftmargin调整列表环境左边的空白长度 [leftmargin=0em]
	\item
填空题 (每小题 3 分,共 15 分)
\begin{enumerate}
	%\renewcommand{\labelenumi}{\arabic{enumi}.}
	% A(\Alph) a(\alph) I(\Roman) i(\roman) 1(\arabic)
	%设定全局标号series=example	%引用全局变量resume=example
	%[topsep=-0.3em,parsep=-0.3em,itemsep=-0.3em,partopsep=-0.3em]
	%可使用leftmargin调整列表环境左边的空白长度 [leftmargin=0em]
	\item
$F(x)=\frac{x^{2}}{x-a} \int_{a}^{x} f(t) d t$, 其中 $f(x)$ 连续, 则有 $\lim\limits _{x \rightarrow a} F(x)$ \tk{} 。
\item 
通过点 $(3,1,-1)$ 及直线 $\frac{x-4}{5}=\frac{y+3}{2}=\frac{z}{1}$ 的平面方程为: \tk{} 。

\item 
$\lim\limits _{x \rightarrow 0} \frac{\sqrt{1+x}+\sqrt{1-x}-2}{x^{2}}=$ \tk{} 。

\item 
方程 $y^{\prime \prime}+y^{\prime}+y=0$ 的通解为 \tk{} 。


\item 
设方程 $f\left(\frac{z}{x}, \frac{y}{z}\right)=0$ 确立了隐函数 $z=z(x, y)$, 则 $\frac{\partial z}{\partial x}=$ \tk{} 。
	
	
	
	
\end{enumerate}




\item 
(本题满分 10 分)
设 $y=y(x), z=z(x)$ 由方程 $z=x f(x+y)$ 和 $F(x, y, z)=0$ 所确定的函数, 其中 $f$ 和 $F$ 分别具有一阶连续导数和一阶连续偏导数, 求 $\frac{d z}{d x}$。



\banswer{
	
}


\item 
(本题满分 12 分)
\begin{enumerate}
	%\renewcommand{\labelenumi}{\arabic{enumi}.}
	% A(\Alph) a(\alph) I(\Roman) i(\roman) 1(\arabic)
	%设定全局标号series=example	%引用全局变量resume=example
	%[topsep=-0.3em,parsep=-0.3em,itemsep=-0.3em,partopsep=-0.3em]
	%可使用leftmargin调整列表环境左边的空白长度 [leftmargin=0em]
	\item
(5分)
计算
	\[
	\int_{0}^{\pi} \frac{\sin \theta}{\sqrt{1-2 a \cos \theta+a^{2}}},(a>1)
	\]
	
	
	\item
	(7分)
	计算曲面积分
	\[
	I=\iint_{S}\left(x^{2}+y^{2}\right) d S
	\]
	$S$ : 曲面 $z=\sqrt{x^{2}+y^{2}}$ 及 $z=1$ 所围成立体的全表面。
	
	
	
	
\end{enumerate}

\banswer{
	
}



\item 
(本题满分 8 分)
将函数 $f(x)=\arctan \frac{1+x}{1-x}$ 展开为 $x$ 的幂级数。

\banswer{
	
}



\newpage

\item 
(本题满分 20 分)
今有常微分方程组
\[
\left\{\begin{array}{l}
	\frac{d x}{d t}=3 x+2 y-z \\
	\frac{d y}{d t}=-2 x-2 y+2 z \\
	\frac{d z}{d t}=3 x+6 z-z
\end{array}\right.
\]	
\begin{enumerate}
	%\renewcommand{\labelenumi}{\arabic{enumi}.}
	% A(\Alph) a(\alph) I(\Roman) i(\roman) 1(\arabic)
	%设定全局标号series=example	%引用全局变量resume=example
	%[topsep=-0.3em,parsep=-0.3em,itemsep=-0.3em,partopsep=-0.3em]
	%可使用leftmargin调整列表环境左边的空白长度 [leftmargin=0em]
	\item
改写方程组为 $\frac{d}{d t} X(t)=A X(t)$ 形式,其中 $X(t)=(x(t), y(t), z(t))^{\prime}$ 为列向量, $A$为 3 阶方阵, 记号 “$  {}^{\prime}  $” 表示转置。(2分)
	
	
	\item
	求矩阵 $A$ 的特征值及对应的特征向量。(10分)
	
	
	\item
	给出非退化矩阵 $T$ 和对角阵 $D ,$ 使得 $T^{-1} A T=D$。(3分)
	
	
	\item
	给出此微分方程的通解。(5分)
	
	
	
	
\end{enumerate}


\banswer{
	
}



\item 
(本题满分 5 分)
设 $A$ 为 $m \times n$ 阶矩阵, $A$ 的秩为 $r>0$ 。证明存在秩为 $r$ 的 $m \times r$ 阶矩阵 $B$ 和秩 为 $r$ 的 $r \times n$ 阶矩阵 $C$, 使得 $A=B C$。

\banswer{
	
}



\item 
(本题满分 5 分)
$f(z)=u(x, y)+i v(x, y)$ 是区域 $D$ 上的非常值解析函数, 试说明 $\overline{f(z)}=u-i v$ 和 $g(z)=v+i v$ 在 $D$ 上的解析性。

\banswer{
	
}



\newpage
\item 
(本题满分 5 分)
计算积分
\[
\int|z|=3\left[\bar{z}+(z-1)^{5} \cos \frac{1}{(z-1)^{3}}\right] d z
\]
其中 $\bar{z}$ 为复数 $z$ 的共轭复数。

\banswer{
	
}


\item 
(本题满分 12 分)
用分离变量法解定解问题。
\[
|x|=\left\{\begin{array}{ll}
	\frac{\partial^{2} u}{\partial x^{2}}+\frac{\partial^{2} u}{\partial y^{2}}=0 & (x, y) \in D, D:-\infty<x<0,0<y<\pi \\
	\left.u\right|_{x=0}=\frac{y}{\pi},\left.u\right|_{x=-\infty} \text { 有界 } & (0<y<\pi) \\
	\left.u\right|_{y=0}=0,\left.u\right|_{y=p i}=0 & (-\infty<x<0)
\end{array}\right.
\]


\banswer{
	
}


\item 
(本题满分 8 分)
\begin{enumerate}
	%\renewcommand{\labelenumi}{\arabic{enumi}.}
	% A(\Alph) a(\alph) I(\Roman) i(\roman) 1(\arabic)
	%设定全局标号series=example	%引用全局变量resume=example
	%[topsep=-0.3em,parsep=-0.3em,itemsep=-0.3em,partopsep=-0.3em]
	%可使用leftmargin调整列表环境左边的空白长度 [leftmargin=0em]
	\item
	将上题中的区域$  D  $保形变换到上半平面。(6分)
	\item 
 利用上半平面 Laplace 方程第一边值问题的 Green 函数,写出 Laplace 方程第一边值问题在区域 $D$ 的 Green 函数,其中 $\bar{z}$ 是 $z$ 的共轭复数。 (2分)
	\[
	G\left(z ; z_{0}\right)=\frac{1}{2 \pi}\left(\ln \frac{1}{\left|z-z_{0}\right|}-\ln \frac{1}{|z-\bar{z}|}\right)
	\]
	
	
	
\end{enumerate}

\banswer{
	
}

	
\end{enumerate}


