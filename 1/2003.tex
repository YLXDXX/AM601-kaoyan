\bta{2003}


\begin{enumerate}
	%\renewcommand{\labelenumi}{\arabic{enumi}.}
	% A(\Alph) a(\alph) I(\Roman) i(\roman) 1(\arabic)
	%设定全局标号series=example	%引用全局变量resume=example
	%[topsep=-0.3em,parsep=-0.3em,itemsep=-0.3em,partopsep=-0.3em]
	%可使用leftmargin调整列表环境左边的空白长度 [leftmargin=0em]
	\item
	填空题 (本题共 5 小题,每小题 5 分,满分 25 分)
\begin{enumerate}
	%\renewcommand{\labelenumi}{\arabic{enumi}.}
	% A(\Alph) a(\alph) I(\Roman) i(\roman) 1(\arabic)
	%设定全局标号series=example	%引用全局变量resume=example
	%[topsep=-0.3em,parsep=-0.3em,itemsep=-0.3em,partopsep=-0.3em]
	%可使用leftmargin调整列表环境左边的空白长度 [leftmargin=0em]
	\item
	$\lim\limits _{x \rightarrow \frac{\pi}{4}} \frac{\sqrt[3]{\tan x}-1}{2 \sin ^{2} x-1}$ \tk{$  \frac{ 1 }{ 3 }  $} 。
	\item 
	设 $\left\{\begin{aligned}x=\int_{1}^{t^{2}} u \ln u d u \\ y=\int_{t^{2}}^{1} u^{2} \ln u d u\end{aligned}\right.$ $(t>0)$, 则 $\frac{d^{2} y}{d x^{2}}=$ \tk{$ \frac{-1}{2t^{2}\ln t} $} 。
	\item 
	级数 $\sum\limits_{n=0}^{\infty} \frac{1}{2 n+1}\left(\frac{1-2 x}{1+x}\right)^{n}$ 的收敛域为 \tk{$ (0,2] $} 。
	\item 
椭球面 $x^{2}+2 y^{2}+z^{2}=1$ 上平行于平面 $x-y+2 z=0$ 的且平面方程为 \tk{$ x-y+2z=\pm \sqrt{\frac{y}{2}} $} 。
	
	
	\item
	微分方程 $y^{\prime \prime}+2 y^{\prime}-2 y=4 x e^{x}$ 的通解为 \tk{$ y=-(2x+1)e^{x}+C_{1}e^{-x}+C_{2}e^{2x} $} 。
	
	
\end{enumerate}

\item 	
选择题 (本题 5 小题,每小题 5 分,满分 25 分)	
\begin{enumerate}
	%\renewcommand{\labelenumi}{\arabic{enumi}.}
	% A(\Alph) a(\alph) I(\Roman) i(\roman) 1(\arabic)
	%设定全局标号series=example	%引用全局变量resume=example
	%[topsep=-0.3em,parsep=-0.3em,itemsep=-0.3em,partopsep=-0.3em]
	%可使用leftmargin调整列表环境左边的空白长度 [leftmargin=0em]
	\item
	设$f(x)=\left\{\begin{array}{ll}|x|^{a} \arctan \frac{1}{x} & x \neq 0 \\ x & x=0\end{array}\right.$在 $x=0$ 处连续但不可导, 则 $a$的取值范围是 \xzanswer{C} 
	
	
\fourchoices
{$a>0$}
{$0<a \leqslant 1$}
{$0<a<1$}
{$a>1$}

\item 


\item
“对任意给定的 $\epsilon>0$, 总存在正整数 $N$, 使得当 $n>N$ 时,就有 $\mid a_{N+1}+a_{N+2}+\cdots+a_{N} \mid<\epsilon $” 是级数 $\sum\limits_{n=1}^{\infty} a_{n}$ 收敛的 \xzanswer{C} 


\fourchoices
{充分条件但非必要条件}
{必要条件但非充分条件}
{充分必要条件}
{既非充分条件也非必要条件}

	
\item 
设 $s$ 是柱面 $x^{2}+y^{2}=R^{2}$ 介于平面 $z=0$ 及 $z=R$ 之间, $\iint_{S}\left(x^{2}+z^{2}\right) d S=$ \xzanswer{B} 


\fourchoices
{$\frac{8}{3} \pi R^{4}$}
{$\frac{5}{3} \pi R^{4}$}
{$\frac{4}{3} \pi R^{4}$}
{$\pi R^{4}$}

\item 
设 $L$ 是起点 $A(-1,0)$, 终点 $B(1,0)$ 的简单光滑曲线, 除 $A, B$ 外其他点都在 $x$ 轴
上方, 则曲线积分 $\int_{L} \frac{-y d x+x d y}{x^{2}+y^{2}}$ 的值为 \xzanswer{A} 


\fourchoices
{恒为 $-\pi$}
{恒为 $ 0 $}
{恒为 $\pi$}
{与曲线 $L$ 有关}

\item 
广义积分 $\int_{0}^{+\infty} \frac{\sin x^{2}}{x^{p}} d x$ 的收敛域为 \xzanswer{D} 


\fourchoices
{$p>-1$}
{$0<p<3$}
{$-1<p<1$}
{$-1<p<3$}


	
	
\end{enumerate}

\item 
(本题共 5 小题,每小题 8 分,满分 40 分)
\begin{enumerate}
	%\renewcommand{\labelenumi}{\arabic{enumi}.}
	% A(\Alph) a(\alph) I(\Roman) i(\roman) 1(\arabic)
	%设定全局标号series=example	%引用全局变量resume=example
	%[topsep=-0.3em,parsep=-0.3em,itemsep=-0.3em,partopsep=-0.3em]
	%可使用leftmargin调整列表环境左边的空白长度 [leftmargin=0em]
	\item
求 $\int \max (x, 1) d x$。
	
	
	\item
	计算无穷积分 $\int_{1}^{+\infty} \frac{1}{x\left(x^{m}+1\right)} d x$, 其中 $m$ 为正整数。
	
	
	\item
	设 $G=\sqrt[n]{(n+1)(n+2) \cdots(n+n)}$, 求 $\lim\limits _{n \rightarrow} \frac{G_{n}}{n}$ 。
	
	
	\item
	证明:当 $x>0$ 的时候,有 $\left(1+\frac{1}{x}\right)^{x}<e<\left(1+\frac{1}{x}\right)^{x+1}$ 。
	
	
	\item
	函数 $f(x)$ 在 $[0,+\infty)$ 上有一阶连续导函数, 对所有 $x \geqslant 0$, 有 $f(x) \leqslant e^{-x}$, 且 $f(0)=1$ 。 证明:存在 $\xi>0$, 使得 $f^{\prime}(\xi)=-e^{-\xi}$。
	
	
	
	
\end{enumerate}

	
\banswer{
	\begin{enumerate}
		%\renewcommand{\labelenumi}{\arabic{enumi}.}
		% A(\Alph) a(\alph) I(\Roman) i(\roman) 1(\arabic)
		%设定全局标号series=example	%引用全局变量resume=example
		%[topsep=-0.3em,parsep=-0.3em,itemsep=-0.3em,partopsep=-0.3em]
		%可使用leftmargin调整列表环境左边的空白长度 [leftmargin=0em]
		\item
		$\int \max (x, 1) d x= 
		\left\{ 
\begin{array}{lr}
	x +C & x \leq 1\\
	\frac{x^{2}+1}{2} +C &x \geq 1
\end{array}\right.
$
\item 		
$ \frac{1}{m} \ln 2 $
\item 
$ \frac{4}{e} $	
\item 
证明略
\item 
证明略
	\end{enumerate}
	
	
}
	
	
\item 
(本题共 3 小题,每小题 12 分,满分 36 分)
\begin{enumerate}
	%\renewcommand{\labelenumi}{\arabic{enumi}.}
	% A(\Alph) a(\alph) I(\Roman) i(\roman) 1(\arabic)
	%设定全局标号series=example	%引用全局变量resume=example
	%[topsep=-0.3em,parsep=-0.3em,itemsep=-0.3em,partopsep=-0.3em]
	%可使用leftmargin调整列表环境左边的空白长度 [leftmargin=0em]
	\item
 求函数 $z=x^{2} y(3-x-y)$ 在闭区域 $D: x \geqslant 0, y \geqslant 0, x+y \leqslant 4$ 上的最大值和最小值。
	
	
	\item
	设 $f(x)=\cos x+\frac{1}{4} \int_{0}^{2 \pi}(2 x-t) f\left(\frac{t}{2}\right) d t$, 其中 $f(x)$ 为连续函数,求 $f(x)$ 。
	
	
	\item
	设 $a, b, c>0$, 求曲面 $x^{2}+y^{2}+\frac{a^{2}-b^{2}}{c^{2}} z^{2}=a^{2}$ 与 $|z| \leqslant c$ 所截物体的体积。
	
	
	
	
\end{enumerate}



\banswer{
	\begin{enumerate}
		%\renewcommand{\labelenumi}{\arabic{enumi}.}
		% A(\Alph) a(\alph) I(\Roman) i(\roman) 1(\arabic)
		%设定全局标号series=example	%引用全局变量resume=example
		%[topsep=-0.3em,parsep=-0.3em,itemsep=-0.3em,partopsep=-0.3em]
		%可使用leftmargin调整列表环境左边的空白长度 [leftmargin=0em]
		\item
		$ z_{\max}=\frac{81}{64}  \quad ( \frac{ 3 }{ 2 } , \frac{ 3 }{ 4 } )  $, \quad $ z_{\min}=-\frac{256}{27}  \quad ( \frac{ 8 }{ 3 } , \frac{ 4 }{ 3 } ) $
		\item 
		$ f(x)= \frac{ 1 }{ 2 } \cos x +Ae^{x}+Be^{-x} $
		\item 
		$ V=\frac{2\pi c}{3}(2a^{2}+b^{2}) $
	\end{enumerate}
	
	
}


\newpage
\item 
(本题共 2 小题,每小题 12 分,满分 24 分)
\begin{enumerate}
	%\renewcommand{\labelenumi}{\arabic{enumi}.}
	% A(\Alph) a(\alph) I(\Roman) i(\roman) 1(\arabic)
	%设定全局标号series=example	%引用全局变量resume=example
	%[topsep=-0.3em,parsep=-0.3em,itemsep=-0.3em,partopsep=-0.3em]
	%可使用leftmargin调整列表环境左边的空白长度 [leftmargin=0em]
	\item
	设 $f(x)=\left\{\begin{aligned}&\frac{\pi-1}{2} x \ & \ 0 \leqslant x \leqslant 1 \\ &\frac{\pi-x}{2} \ & \ 1<x \leqslant \pi\end{aligned}\right.$,将 $f(x)$ 展开为周期为 $2 \pi$ 的正弦级数, 并求$\sum\limits_{n=1}^{\infty} \frac{\sin ^{2} n}{n^{2}}$ 和 $\sum\limits_{n=1}^{\infty} \frac{\sin ^{2} n}{n^{4}}$。
	
	\item 
设区域 $c$ 由曲面 $2 x^{2}+y^{2}-z^{2}=1$ 及平面 $z=1, z=-1$ 所围成, $S$ 为 $c$ 的全表 面外侧, 又设 $\vec{v}=\left(2 x^{2}+y^{2}+z^{2}\right)^{-\frac{3}{2}}(x \vec{i}+y \vec{j}+z \vec{k})$ 。
\begin{enumerate}
	%\renewcommand{\labelenumi}{\arabic{enumi}.}
	% A(\Alph) a(\alph) I(\Roman) i(\roman) 1(\arabic)
	%设定全局标号series=example	%引用全局变量resume=example
	%[topsep=-0.3em,parsep=-0.3em,itemsep=-0.3em,partopsep=-0.3em]
	%可使用leftmargin调整列表环境左边的空白长度 [leftmargin=0em]
	\item
求 $\operatorname{div} \vec{v}$。
	\item 
	求积分 $\iint_{S} \frac{a d y d z+y d z d x+z d x d y}{\left(2 x^{2}+y^{2}+z^{2}\right)^{\frac{3}{2}}}$。
\end{enumerate}
	
	
	
	
\end{enumerate}


\banswer{
	\begin{enumerate}
		%\renewcommand{\labelenumi}{\arabic{enumi}.}
		% A(\Alph) a(\alph) I(\Roman) i(\roman) 1(\arabic)
		%设定全局标号series=example	%引用全局变量resume=example
		%[topsep=-0.3em,parsep=-0.3em,itemsep=-0.3em,partopsep=-0.3em]
		%可使用leftmargin调整列表环境左边的空白长度 [leftmargin=0em]
		\item
		$ f(x)=\sum\limits_{n=1}^{\infty} \frac{\sin n}{n^{2}} \sin nx (0 \leq x \leq \pi)$,利用$ x=1 $得到$\sum\limits_{n=1}^{\infty} \frac{\sin ^{2} n}{n^{2}}=\frac{\pi -1}{2}$,再由帕塞瓦尔恒等式(Parseval's identity),得到 $\sum\limits_{n=1}^{\infty} \frac{\sin ^{2} n}{n^{4}}=\frac{1}{\pi} \int_{-\pi}^{\pi} f^{2}(x) dx = \frac{(\pi-1)^{2}}{6} $
		\item 
		\begin{enumerate}
			%\renewcommand{\labelenumi}{\arabic{enumi}.}
			% A(\Alph) a(\alph) I(\Roman) i(\roman) 1(\arabic)
			%设定全局标号series=example	%引用全局变量resume=example
			%[topsep=-0.3em,parsep=-0.3em,itemsep=-0.3em,partopsep=-0.3em]
			%可使用leftmargin调整列表环境左边的空白长度 [leftmargin=0em]
			\item
			$\operatorname{div} \vec{v}=0$
			\item 
			$ 2\sqrt{2} \pi $
			
		\end{enumerate}
		
		
		
	\end{enumerate}
	
	
}
	
	
\end{enumerate}


