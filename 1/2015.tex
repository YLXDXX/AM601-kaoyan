\bta{2015}


\begin{enumerate}
	%\renewcommand{\labelenumi}{\arabic{enumi}.}
	% A(\Alph) a(\alph) I(\Roman) i(\roman) 1(\arabic)
	%设定全局标号series=example	%引用全局变量resume=example
	%[topsep=-0.3em,parsep=-0.3em,itemsep=-0.3em,partopsep=-0.3em]
	%可使用leftmargin调整列表环境左边的空白长度 [leftmargin=0em]
	\item
	选择题 (本题满分 50 分,每小题 5 分)
	\begin{enumerate}
		%\renewcommand{\labelenumi}{\arabic{enumi}.}
		% A(\Alph) a(\alph) I(\Roman) i(\roman) 1(\arabic)
		%设定全局标号series=example	%引用全局变量resume=example
		%[topsep=-0.3em,parsep=-0.3em,itemsep=-0.3em,partopsep=-0.3em]
		%可使用leftmargin调整列表环境左边的空白长度 [leftmargin=0em]
		\item
		对于函数 $f(x)=\frac{\sin x^{2}}{x}$, 结论不正确的是 \xzanswer{B} 
		
		
\fourchoices
{在 $(0, \infty)$ 内有界}
{在 $(0, \infty)$ 内 $f(x)$ 没有最大值和最小值}
{在 $(0, \infty)$ 内处处可导}
{当 $x \rightarrow \infty, x \rightarrow 0^{+}$时, $f(x)$ 极限存在}

\item 
$ \lim\limits _{x \rightarrow 0}\left(\frac{\sin x}{x}\right)^{\frac{1}{x^{2}}}= $ \xzanswer{D} 

\fourchoices
{$ 1 $}
{$ 0 $}
{$\infty$}
{$e^{-\frac{1}{6}}$}
	
\item 
微分方程 $y^{\prime}=\frac{1}{y-x}$ 的通解为 \xzanswer{A} 


\fourchoices
{$x=y+C e^{-y}-1$}
{$y=x+C e^{-x}-1$}
{$x=\ln (x-y-1)+C$}
{$y=\ln |y-x-1|+C$}

\item 
已知 $m, n$ 为正整数,且 $m>n$ 。如果: $S=\int_{0}^{\frac{\pi}{4}} \sin ^{m} x \cos ^{n} x d x, T=\int_{0}^{\frac{\pi}{4}} \sin ^{n} x \cos ^{m} x d x$。 则下面结论正确的一个是 \xzanswer{C} 


\fourchoices
{$S>T$}
{$S=T$}
{$S<T$}
{$S, T$ 的关系不确定}

\item 
设对任意的 $x \in \mathbb{R}$, 总有 $m \leqslant f(x)<g(x)<h(x) \leqslant M$ 。 且 $g(x)$ 为连续函数, 若 $\lim\limits _{x \rightarrow \infty}[M-f(x)][h(x)-m]=0$ 。 则对于 $\lim\limits _{x \rightarrow \infty}g(x)$,则下面结论正确的一个是 \xzanswer{B} 


\fourchoices
{一定存在, 且等于 $\frac{M+m}{2}$}
{一定存在, 且只能等于 $M$ 或 $m$}
{一定不存在}
{一定存在, 且可以取到 $[m, M]$ 上的任意值}


\item 
设函数 $f(x)=x^{2} \sin x+\cos x+\frac{\pi}{2} x$, 在其定义域内零点的个数是 \xzanswer{D} 


\fourchoices
{$ 2 $}
{$ 3 $}
{$ 4 $}
{多于 $ 4 $}

\item 
设 $ \vv{a} ,  \vv{b} $ 为非零实向量, 且 $2  \vv{a} + \vv{b} $ 与 $ \vv{a} - \vv{b} $ 垂直,$  \vv{a}+2 \vv{b}   $与$  \vv{a} +  \vv{b}  $垂直。则 \xzanswer{A} 


\fourchoices
{$| \vv{b} |^{2}=7| \vv{a} |^{2}$}
{$| \vv{a} |^{2}=7| \vv{b} |^{2}$}
{$| \vv{b} |^{2}=5| \vv{a} |^{2}$}
{$| \vv{a} |^{2}=5| \vv{b} |^{2}$}

\item 
设平面$ D $由 $x+y=\frac{1}{2}, x+y=1$ 及两条坐标轴围成, $I_{1}=\iint_{D} \ln (x+y)^{3} d x d y, I_{2}=\iint_{D}(x+y)^{3} d x d y, I_{3}=\iint_{D} \sin (x+y)^{3} d x d y$ 。则下面结论正确的一个是 \xzanswer{C} 


\fourchoices
{$I_{1}<I_{2}<I_{3}$}
{$I_{3}<I_{1}<I_{2}$}
{$I_{1}<I_{3}<I_{2}$}
{$I_{3}<I_{2}<I_{1}$}


\item 
幂级数 $\sum\limits_{n=1}^{\infty} a_{n} x^{n}$ 与 $\sum\limits_{n=1}^{\infty} b_{n} x^{n}$ 的收敛半径分别为 $\frac{\sqrt{2}}{3}$ 与 $\frac{1}{3}$, 则幂级数 $\sum\limits_{n=1}^{\infty} \frac{a_{n}^{2}}{b_{n}^{2}} x^{n}$ 的 收敛半径为 \xzanswer{A} 


\fourchoices
{$ 2 $}
{$\frac{\sqrt{2}}{3}$}
{$\frac{1}{3}$}
{$\frac{1}{2}$}


\item 
已知曲面 $x^{2}+y^{2}+2 z^{2}=5$, 在其点 $\left(x_{0}, y_{0}, z_{0}\right)$ 处的切平面与平面 $x+2 y+z=0$ 平行, 则有 \xzanswer{B} 


\fourchoices
{$x_{0}: y_{0}: z_{0}=4: 2: 1$}
{$x_{0}: y_{0}: z_{0}=2: 4: 1$}
{$x_{0}: y_{0}: z_{0}=1: 4: 2$}
{$x_{0}: y_{0}: z_{0}=1: 2: 4$}



		
		
	\end{enumerate}
	


\item 
(本题满分 10 分)
计算 $\lim\limits _{n \rightarrow \infty} \frac{1}{n}\left(\sqrt{1+\sin \frac{\pi}{n}}+\sqrt{1+\sin \frac{2 \pi}{n}}+\cdots \cdots+\sqrt{1+\sin \frac{n \pi}{n}}\right)$。

\banswer{
	转换成定积分,计算可得:$ \frac{4}{\pi} $
}


\item 
(本题满分 10 分)
设 $u=e^{x^{2}} \sin \frac{x}{y}$, 计算 $\frac{\partial^{2} u}{\partial x \partial y}$ 在点 $(\pi, 2)$ 处的值。

\banswer{
	$\frac{\pi }{8} \cdot e^{\pi^{2}} $
}



\item 
(本题满分 10 分)
设 $D$ 为第一象限内由 $\left\{\begin{array}{l}y=x \\ y=2x \\ x y=1 \\ x y=2\end{array}\right.$所围成的区域, $f$ 为一元可微函数, 且 $f^{\prime}=g$ ,记 $L$ 为 $D$ 的边界。请证明:
\[
\oint_{L} x f\left(\frac{y}{x}\right) \mathrm{d} x=-\int_{1}^{2} \frac{g(u)}{2 u} \mathrm{~d} u
\]


\banswer{
	证明略。先用格林公式化成面积分,在根据积分区域换元可得。
}


\item 
(本题满分 10 分)
已知 $y_{1}=x, y_{2}=x^{2}, y_{3}=e^{x}$ 为线性非齐次微分方程: $y^{\prime \prime}+p(x) y^{\prime}+q(x) y=f(x)$ 的 三个特解,求该方程满足初始条件 $y(0)=1, y^{\prime}(0)=0$ 的特解。


\banswer{
	$ y=e^{x}+x^{2}-x $
}


\item 
(本题满分 10 分)
设 $f(x)=4 x+\cos \pi x+\frac{1}{1+x^{2}}-x^{2} e^{x}+x e^{x} \int_{x}^{1} f(t) d t$ 。 求 $\int_{0}^{1}(1-x) e^{x} f(x) d x$。

\banswer{
	先对题中等式两边从$ 0 \rightarrow 1 $积分。此后的解法有两种:①交换积分后的顺序;②构造函数,对被积函数进行分部积分。
	$\int_{0}^{1}(1-x) e^{x} f(x) d x= 4+ \frac{\pi}{4} -e $
}


\newpage
\item 
(本题满分 10 分)
计算曲面积分 $I=\iint_{\Sigma} \frac{x d y d z+y d z d x+z d x d y}{\left(x^{2}+y^{2}+z^{2}\right)^{\frac{3}{2}}}$, 其中 $\Sigma$ 是曲面 $2 x^{2}+2 y^{2}+z^{2}=4$ 的外侧。

\banswer{
	积分曲面包含奇点$ (0,0,0) $,为了好计算,找一个好的位置用球坐标系。\\
	设球面$ \Sigma ^{\prime}  $为$ x^{2}+y^{2}+z^{2}=\epsilon^{2}  $,$ \epsilon \ll 1 $
	\[ 
		\begin{aligned}
			I&=\iint_{\Sigma}  \vv{a} \cdot d \vv{S} \\
			& =\iiint_{V} \triangledown  \cdot  \vv{a} dv +	\iint_{\Sigma ^{\prime} }  \vv{a} \cdot d \vv{S}  \\
			&=0+\iiint_{V ^{\prime} } 4\pi \cdot \frac{1}{4\pi r^{2}} \delta(r) \cdot r^{2} \sin \theta  dr d\theta d\varphi\\
			&=4\pi
	\end{aligned}
	 \]
}


\item 
(本题满分 10 分)
将函数 $f(x)=x-1$ $(0 \leqslant x \leqslant 2)$ 展开成周期为$  4  $的余弦级数。

\banswer{
	$ f(x)=\left(\frac{2}{\pi}\right)^{2} \sum\limits_{n=1}^{\infty}  \frac{(-1)^{n}-1}{n^{2}}  \cos (\frac{\pi}{2}nx) $
}


\item 
(本题满分 10 分)
若 $g(x)$ 为单调增加的可微函数, 且当 $x \geqslant a$ 时, $\left|f^{\prime }(x)\right| \leqslant g^{\prime }(x)$ 。 证明:当 $x \geqslant a$ 时, $|f(x)-f(a)| \leqslant g(x)-g(a)$。


\banswer{
	证明略。构造函数$ f(x)-f(0)=\int_{a}^{x} f ^{\prime} (x) dx $,可证。注意,这里如果用柯西中值定理的话需要对导数为零的点单独讨论。
}


\item 
(本题满分 10 分)
函数 $f(x)$ 在 $[0,2]$ 上二阶可导, 且 $f^{\prime}(0)=f^{\prime}(2)=0$ 。 证明:在区间 $(0,2)$ 内至少存 在一点 $\xi$, 满足 $\left|f^{\prime \prime}(\xi)\right| \geqslant|f(2)-f(0)|$。

\banswer{
	证明略。\\
	可以用泰勒展开来做,也可以构造函数$ f(2)-f(0)=\int_{0}^{1}\int_{0}^{x} f ^{\prime\prime}(t)dtdx  - \int_{1}^{2} \int_{x}^{2} f ^{\prime\prime}(t)dtdx  $
}



\item 
(本题满分 10 分)
设 $0<a<1$,$ x \geqslant 0$ 且 $y \geqslant 0$ 。 证明:
\begin{enumerate}
	%\renewcommand{\labelenumi}{\arabic{enumi}.}
	% A(\Alph) a(\alph) I(\Roman) i(\roman) 1(\arabic)
	%设定全局标号series=example	%引用全局变量resume=example
	%[topsep=-0.3em,parsep=-0.3em,itemsep=-0.3em,partopsep=-0.3em]
	%可使用leftmargin调整列表环境左边的空白长度 [leftmargin=0em]
	\item
$x^{a} y^{1-a} \leqslant a x+(1-a) y$
\item 
设 $x_{1}, x_{2}, x_{3}, \cdots, x_{n}, y_{1}, y_{2}, \cdots, y_{n}$, 利用 (1) 中的结果证明:
\[
x_{1} y_{1}+x_{2} y_{2}+\cdots+x_{n} y_{n} \leqslant\left(x_{1}^{2}+x_{2}^{2}+\ldots+x_{n}^{2}\right)^{\frac{1}{2}}\left(y_{1}^{2}+y_{2}^{2}+\ldots+y_{n}^{2}\right)^{\frac{1}{2}}
\]
\end{enumerate}


\banswer{
	\begin{enumerate}
		%\renewcommand{\labelenumi}{\arabic{enumi}.}
		% A(\Alph) a(\alph) I(\Roman) i(\roman) 1(\arabic)
		%设定全局标号series=example	%引用全局变量resume=example
		%[topsep=-0.3em,parsep=-0.3em,itemsep=-0.3em,partopsep=-0.3em]
		%可使用leftmargin调整列表环境左边的空白长度 [leftmargin=0em]
		\item
		证明略。变形$ e^{a\ln x} e^{b \ln y} \leq ae^{\ln x} + be^{\ln y} $,跟上一年的题就差不多了。这里,有个通用结论:只要$ f ^{\prime\prime} (x)\leq 0 $,则有(其中$ a+b=1 $)
		\[ 
		f(ax+by)\leq af(x)+bf(y)
		 \]
		\item 
		证明略。把不等式右边化成$ 1 $的形式即可证。
		
		
	\end{enumerate}
	
	
}
	
	
	
\end{enumerate}

