\bta{2012}


\begin{enumerate}
	%\renewcommand{\labelenumi}{\arabic{enumi}.}
	% A(\Alph) a(\alph) I(\Roman) i(\roman) 1(\arabic)
	%设定全局标号series=example	%引用全局变量resume=example
	%[topsep=-0.3em,parsep=-0.3em,itemsep=-0.3em,partopsep=-0.3em]
	%可使用leftmargin调整列表环境左边的空白长度 [leftmargin=0em]
	\item
	选择题 (本题满分 50 分,每小题 5 分)
	\begin{enumerate}
		%\renewcommand{\labelenumi}{\arabic{enumi}.}
		% A(\Alph) a(\alph) I(\Roman) i(\roman) 1(\arabic)
		%设定全局标号series=example	%引用全局变量resume=example
		%[topsep=-0.3em,parsep=-0.3em,itemsep=-0.3em,partopsep=-0.3em]
		%可使用leftmargin调整列表环境左边的空白长度 [leftmargin=0em]
		\item
		函数 $f(x)=x \cos x^{2}$, 正确结论是 \xzanswer{C} 
		
	\fourchoices
	{在 $(-\infty, 0)$ 内有界}
	{当 $x \rightarrow \infty$ 时, $f(x)$ 为无穷大}
	{在 $(-\infty, 0)$ 内无界}
	{当 $x \rightarrow \infty$ 时, $f(x)$ 极限存在}
	
\item 
函数 $f(x)$ 上是连续函数, 且 $0<m<f(x)<M<\infty$, 则 $\frac{1}{m} \int_{-m}^{m}(f(t)-M) d t$ 的最大取值区间是 \xzanswer{B} 


\fourchoices
{$(-M-m, m-M)$}
{$(2 m-2 M, 0)$}
{$(m-M, 0)$}
{$(0, M+m)$}


\item 
微分方程 $y y^{\prime \prime}-\left(y^{\prime}\right)^{2}=0$ 的一个特解是 \xzanswer{D} 


\fourchoices
{$y=x e^{x}$}
{$y=x \ln x$}
{$y=\ln x$}
{$y=e^{x}$}

\item 
已知 $n, m$ 是正整数, 且 $n<m$, 如果 $A=\int_{0}^{1} x^{m}(1-x)^{n} d x, B=\int_{0}^{1} x^{n}(1-x)^{m+1} d x$, 则下面结论正确的一个是 \xzanswer{A} 


\fourchoices
{$A>B$}
{$A=B$}
{$A<B$}
{$A, B$ 的大小关系不确定}

\item 
函数 $f(x)=e^{x}-x^{2}-4 x-3$ 在其定义域内零点的个数是 \xzanswer{C} 



\fourchoices
{$ 1 $}
{$ 2 $}
{$ 3 $}
{多余$  3 $}

\item 
若函数 $f(x)= \left\{\begin{aligned} &e^{x}(\sin x+\cos x) &   x \geqslant 0 \\ &a b x^{2}+a x+2 a+b &   x<0\end{aligned}\right.$ 的导函数在 $(-\infty,+\infty)$ 上连续, 则 \xzanswer{B} 


\fourchoices
{$a=2, b=-1$}
{$a=2, b=-3$}
{$a=1, b=-3$}
{$a=1, b=-1$}

\item 
若幂级数 $\sum\limits_{n=1}^{\infty} a_{n}(x-1)^{n}$ 在 $x=4$ 处条件收敛, 则级数 $\sum\limits_{n=1}^{\infty}(-1)^{n}\left(1+2^{n}\right) a_{n}$ \xzanswer{C} 


\fourchoices
{条件收敛}
{发散}
{绝对收敛}
{不能确定}

\item 
设 $S$ 为螺旋面 $\left\{\begin{array}{l}x=u \cos v \\ y=u \sin v   \\ z=v\end{array} \right.$的一部份,$ 0\leq u \leq \sqrt{15} $,$ 0 \leq v \leq \pi $ 则 $\iint_{S} \sqrt{x^{2}+y^{2}} d S$的值为 \xzanswer{C} 


\fourchoices
{$17 \pi$}
{$19 \pi$}
{$21 \pi$}
{$23 \pi$}

\item 
$\lim\limits _{x \rightarrow 0}\left(\frac{x}{\sin x}\right)^{\frac{1}{1-\cos x}}$ 的值为 \xzanswer{A} 


\fourchoices
{$e^{\frac{1}{3}}$}
{$e^{\frac{-1}{3}}$}
{$e^{\frac{1}{2}}$}
{$e^{-\frac{1}{2}}$}


\item 
一平面过点 $M(1,1,-1)$ 且与直线 $L: \frac{x}{2}=\frac{y+1}{1}=\frac{z-3}{-1}$ 垂直, 则该平面与平 面 $x-2 y-z+1=0$ 的交线的方向数是 \xzanswer{D} 


\fourchoices
{$(-5,1,3)$}
{$(1,-3,5)$}
{$(1,-5,3)$}
{$(3,-1,5)$}


		
		
	\end{enumerate}




\item 
(本题满分 10 分)	
证明极限 $\lim\limits _{n \rightarrow \infty}\left(\frac{1}{n}+\frac{1}{n+1}+\cdots+\frac{1}{3 n}\right)$ 存在, 并求出极限值。

\banswer{
	证明略,单调递减,有下界。可利用定积分的定义,算得极限值为$ \ln 3 $
}


\item 
(本题满分 10 分)	
求微分方程 $y^{\prime \prime}-3 y^{\prime}+2 y=e^{x}(2 x+1)$ 的通解。

\banswer{
	$y=C_{1} e^{x}+C_{2} e^{2 x}-\left(x^{2}+3 x\right)e^{x}$
}



\item 
(本题满分 10 分)	
计算 $\iint_{D}(x|y|+x y) d x d y$, 其中 $D$ 是由抛物线 $5 y=x^{2}-6$ 和抛物线 $y^{2}=x$ 围成的 闭区域。


\banswer{
求交点时的因式分解为$ (y+1)(y-2)(y^{2}+y+3)=0 $,最终结果为$\frac{44}{3}$
}


\item 
(本题满分 10 分)	
将函数 $f(x)=|x-1|(0 \leqslant x \leqslant \pi)$ 展开成正弦级数。


\banswer{
	作奇延拓,
\[ 
f(x)= \sum\limits_{n=1}^{\infty}  \frac{2}{\pi} \left(
\frac{1}{n}+ \frac{(-1)^{2}}{n}\cdot (1-\pi) -\frac{2}{n^{2}} \sin n 
\right) \sin n x   \quad  (0<x<\pi) 
 \]
$ f(x) $的傅氏级数在$ x=0 $处收敛于$ 0 $,在$ x=\pi $处收敛于$ 0 $
}


\item 
(本题满分 10 分)	
设函数 $f(x)=\int_{x}^{1} e^{-t^{2}} d t$, 求 $\int_{0}^{1} x^{2} f(x) d x$ 的值。

\banswer{
	$\frac{1}{6}\left(1-\frac{2}{e}\right)$,用分部积分或交换积分顺序都可
}


\newpage
\item 
(本题满分 10 分)	
计算曲线积分 $I=\oint_{L} \frac{x d y-y d x}{x^{2}+2y^{2}}$, 其中 $L$ 是由直线 $x+y=1, y=x-1$ 和半圆周 $x^{2}+y^{2}=1, x \leqslant 0$ 所围成的闭曲线, 方向为逆时针方向。


\banswer{
与路径无关,包含奇点,	$I=\sqrt{2} \pi$
}


\item 
(本题满分 10 分)	
设函数 $f(x)$ 连续, 且 $f^{2}(x) \leqslant|x|^{3}$, 记 $F(x)=\int_{0}^{1} f(x t) d t$, 求 $F^{\prime}(x)$, 并讨论 $F^{\prime}(x)$ 的连续性。

\banswer{
用导数定义,简单放缩后得到$ F ^{\prime} (0)=0 $,\\
\[ 
F^{\prime}(x)=\left\{\begin{aligned}
	&-\frac{1}{x^{2}} \int_{0}^{x} f(u) d u+\frac{f(x)}{x}, &x \neq 0 \\
	&0, &x=0\end{aligned}\right.
 \]
$ \lim\limits_{x\rightarrow0}F ^{\prime} (x)=0 $,故$ F ^{\prime}(x) $在整个实轴上连续
}



\item 
(本题满分 10 分)	
函数 $f(x)$ 在 $[a, b]$ 上连续, 在 $(a, b)$ 内可导, 并且有 $0<a<b$ 。 证明:
存在 $\xi \in(a, b)$, 使得 $\frac{a+b}{2 \xi} f^{\prime}(\xi)=\frac{f(b)-f(a)}{b-a}$。


\banswer{
	证明略,利用柯西中值定理即可。
}


\item 
(本题满分 10 分)	
函数 $f(x)$ 在 $(-\infty,+\infty)$ 上满足 $f^{\prime \prime}(x)>0$ 。 证明:
\[
f\left(\frac{x_{1}+x_{2}+\cdots+x_{n}}{n}\right) \leq \frac{f\left(x_{1}\right)+f\left(x_{2}\right)+\cdots+f\left(x_{n}\right)}{n}
\]


\banswer{
	证明略。利用二阶泰勒展开,求和,令特殊值即可。
}



\item 
(本题满分 10 分)	
设 $a<b$, 函数 $f(x)$ 在 $[a, b]$ 上连续, 且
\[
\int_{a}^{b} f(x) d x=\int_{a}^{b} x f(x) d x=\int_{a}^{b} x^{2} f(x) d x=0
\]
证明:在 $(a, b)$ 上至少存在三个不同点 $x_{1}, x_{2}, x_{3}$, 使得 $f\left(x_{1}\right)=f\left(x_{2}\right)=f\left(x_{3}\right)=0$。


\banswer{
	证明略。这里可以采取两种方法来证明:
	\begin{enumerate}
		%\renewcommand{\labelenumi}{\arabic{enumi}.}
		% A(\Alph) a(\alph) I(\Roman) i(\roman) 1(\arabic)
		%设定全局标号series=example	%引用全局变量resume=example
		%[topsep=-0.3em,parsep=-0.3em,itemsep=-0.3em,partopsep=-0.3em]
		%可使用leftmargin调整列表环境左边的空白长度 [leftmargin=0em]
		\item
		使用罗尔定理,$ F(x)=\int_{a}^{x} f(x)dx $,$ G (x)=\int_{a}^{x} F(x) dx $,再用分部积分可证
		\item 
		使用反证法,①不存在;②值存在一个;③只存在两个;配合积分中值定理可证
	\end{enumerate}
	
	
}




	
\end{enumerate}

