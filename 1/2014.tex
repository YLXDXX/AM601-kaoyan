\bta{2014}


\begin{enumerate}
	%\renewcommand{\labelenumi}{\arabic{enumi}.}
	% A(\Alph) a(\alph) I(\Roman) i(\roman) 1(\arabic)
	%设定全局标号series=example	%引用全局变量resume=example
	%[topsep=-0.3em,parsep=-0.3em,itemsep=-0.3em,partopsep=-0.3em]
	%可使用leftmargin调整列表环境左边的空白长度 [leftmargin=0em]
	\item
	选择题 (本题满分 50 分,每小题 5 分)
	\begin{enumerate}
		%\renewcommand{\labelenumi}{\arabic{enumi}.}
		% A(\Alph) a(\alph) I(\Roman) i(\roman) 1(\arabic)
		%设定全局标号series=example	%引用全局变量resume=example
		%[topsep=-0.3em,parsep=-0.3em,itemsep=-0.3em,partopsep=-0.3em]
		%可使用leftmargin调整列表环境左边的空白长度 [leftmargin=0em]
		\item
		设数列 $\left\{a_{n}\right\}$ 收敛,其极限为 $a \geqslant 2$, 则 $\lim\limits _{n \rightarrow \infty} \frac{1}{n} \sum\limits_{k=1}^{n} a_{k}=$ \xzanswer{D} 
		
		
\fourchoices
{$\infty$}
{$ 0 $}
{$ 1 $}
{$a$}

\item 
求 $\lim\limits _{x \rightarrow 0}\left(\frac{1}{x \sin x}-\frac{1}{x^{2}}\right)=$ \xzanswer{C} 


\fourchoices
{$ 0 $}
{$\infty$}
{$\frac{1}{6}$}
{$\frac{1}{3}$}

\item 
$M=\int_{0}^{\frac{\pi}{2}} \frac{\sin ^{2} x}{1+x^{k}} d x, N=\int_{0}^{\frac{\pi}{2}} \frac{\cos ^{2} x}{1+x^{k}} d x, k$ 为正整数,下面结论正确的是 \xzanswer{A} 


\fourchoices
{$M<N$}
{$M=n$}
{$M>N$}
{$M, N$ 的大小关系不确定}

\item 
设数列 $a_{n}$ 单调减少, 且 $\lim\limits _{n \rightarrow \infty} a_{n}=0$, 令 $S_{n}=\sum\limits_{k=1}^{n} a_{k} $,$ \left\{ S_{n} \right\} $ 是无界数列, 则幂级数 $\sum\limits_{n=1}^{\infty} a_{n}(x-~1)^{n}$ 的收敛域为 \xzanswer{C} 

\fourchoices
{$[-1,1]$}
{$[-1,1)$}
{$[0,2)$}
{$(0,2]$}

\item 
设函数 $u(x, y)=\varphi(x)-\varphi(x-y)+\int_{x-y}^{x} \phi(t) d t$, 其中函数 $\varphi$ 具有二阶导数, $\phi$ 具 有一阶导数, 则必有 \xzanswer{D} 


\fourchoices
{$\frac{\partial^{2} u}{\partial x \partial y}=\frac{\partial^{2} u}{\partial x^{2}}$}
{$\frac{\partial^{2} u}{\partial x \partial y}=-\frac{\partial^{2} u}{\partial x^{2}}$}
{$\frac{\partial^{2} u}{\partial x \partial y}=\frac{\partial^{2} u}{\partial y^{2}}$}
{$\frac{\partial^{2} u}{\partial x \partial y}=-\frac{\partial^{2} u}{\partial y^{2}}$}
	
\item 
设 $y=f(x)$ 是方程 $y^{\prime \prime}-2 y^{\prime}+4 y=0$ 的一个解, 且 $f(x_{0})>0, f^{\prime}\left(x_{0}\right)=0$, 则函 数 $f(x)$ 在点 $x_{0}$ 处 \xzanswer{A} 
	
\fourchoices
{取得极大值}
{取得极小值}
{某邻域内单调增加}
{某邻域内单调减少}


\item 
设 $ \Sigma $ 为 $z o y$ 平面内曲线 $z=y^{2}(0 \leqslant y \leqslant 2)$绕 $z$ 轴旋转生成的旋转曲面与平 面 $z=4$ 所围成的区域边界的外侧, $\cos \alpha, \cos \beta, \cos \gamma$ 为其外法线向量的方向余弦, 则 $\iint\limits_{\Sigma}\left[\left(\frac{x^{2} y}{2}+2 x-z\right) \cos \alpha+(3 y+z) \cos \beta-x y z \cos \gamma\right] d S$ 的值为 \xzanswer{B} 


\fourchoices
{$ 40 $}
{$40 \pi$}
{$ 20 $}
{$20 \pi$}

\item 
已知 $x e^{-2 x}, e^{x}, 3 x$ 是 $n$ 阶常系数微分方程 $y^{(n)}+a_{1} y^{(n-1)}+\ldots \ldots+a_{n-1} y^{\prime}+a_{n} y=0$ 的三个解, 而 $e^{-x}$ 不是该微分方程的解, 则下述结论中成立的是 \xzanswer{B} 


\fourchoices
{$n=6, a_{1}=4, a_{2}=3, a_{3}=a_{4}=-4, a_{5}=a_{6}=0$}
{$n=5, a_{1}=3, a_{2}=0, a_{3}=-4, a_{5}=a_{4}=0$}
{$n=4, a_{1}=1, a_{2}=-3, a_{3}=a_{4}=0$}
{$n=3, a_{1}=-1, a_{3}=a_{2}=0$}

\item 
已知直线 $L_{1}:\left\{\begin{array}{l}2 x+y-z+1=0 \\ x-2 y+2 z-3=0\end{array} \quad\right.$ 与直线 $L_{2}:\left\{\begin{array}{l}y+z+5=0 \\ 2 x-z+1=0\end{array}\right.$ 。则 $L_{1}$ 和$L_{2}$ 的夹角为 \xzanswer{D} 


\fourchoices
{$\frac{\pi}{6}$}
{$\frac{\pi}{4}$}
{$\frac{\pi}{3}$}
{$\frac{\pi}{2}$}

\item 
积分 $\int_{0}^{\pi} \sqrt{1-\sin x} d x=$ \xzanswer{D} 


\fourchoices
{$\pi$}
{$\frac{\pi}{2}$}
{$\sqrt{2}+1$}
{$4(\sqrt{2}-1)$}


	
		
	\end{enumerate}



\item 	
(本题满分 10 分)	
设 $a \geqslant-12$ 是一个给定的实数, 已知 $x_{1}=a$, 且 $x_{n+1}=\sqrt{12+x_{n}}$, 判断 $\left\{x_{n}\right\}$ 的 极限是否存在; 若存在, 求出 $\left\{x_{n}\right\}$ 的极限。

\banswer{
	除$ x_{1} $外$ x_{n} \geq 0 $,求大小关系
\[ 
\begin{WithArrows}
	x_{n+1}-x_{n}&=\sqrt{12+x_{n}} -x_{n}\Arrow{相当于} \\
	&=\sqrt{12+x}-x
\end{WithArrows} 
\]	 
分类讨论$ x_{n}=4 $、$ x_{n}<4 $、$ x_{n}>4 $。极限存在,且为$ 4 $。
}



\item 	
(本题满分 10 分)	
设 $L$ 为圆 $x^{2}+y^{2}=2$ 位于第一象限的一段, 方向为逆时针方向, $f(x)$ 为 $R$ 上的正
值连续函数,证明:
\[
\int_{L} y\left(f(x)-\frac{1}{f(x)}\right) d x+\left(x^{2} y+2 x f(y)\right) d y \geq 1+\pi
\]

\banswer{
把路径补全为四分之一个半圆,有
\[ 
\begin{WithArrows}[ll]
 \text{	原式} &= \iiint _{S} \left( \frac{\partial Q}{\partial x} - \frac{\partial P}{\partial y} \right)\\
 &=\int_{0}^{\sqrt{2}} \left[  x(2-x^{2}) +\frac{1}{f(x)} -f(x) \right]  dx  + \int_{0}^{\sqrt{2}} 2f(y)\sqrt{2-y^{2}} dy  \Arrow{将$ y $替换成$ x $可得} \\
&=\int_{0}^{\sqrt{2}} \left[  x(2-x^{2}) +\frac{1}{f(x)} + f(x) \right]  dx   \\
&\geq \int_{0}^{\sqrt{2}}x(2-x^{2}) + \pi   \\
	&\geq 1+ \pi
\end{WithArrows} 
\]	

}


\item 	
(本题满分 10 分)	
求微分方程的通解
\[
\left[2 x+e^{x} \sin (x y)+y e^{x} \cos (x y)\right] d x+\left[x e^{x} \cos (x y)+3 y^{2}\right] d y=0
\]

\banswer{
	$ x^{2}+ e^{x}\sin(xy) +y^{3}=C $
}


\item 	
(本题满分 10 分)	
将函数 $f(x)=1-x^{2} \ (0 \leqslant x \leqslant \pi)$ 展开成余弦级数,并求 $\sum\limits_{n=1}^{\infty} \frac{(-1)^{n+1}}{n^{2}}$。

\banswer{
\[ 
f(x)=1-\frac{\pi^{2}}{3} + 4 \sum\limits_{n=1}^{\infty} \cos (nx)
 \]
利用$ f(0) $可得$ \sum\limits_{n=1}^{\infty} \frac{(-1)^{n+1}}{n^{2}}=\frac{\pi^{2}}{12}$
}

\newpage
\item 	
(本题满分 10 分)	
已知 $2 \mu+e^{\mu}=x y$, 求 $\frac{\partial^{2} u}{\partial x \partial y}$。


\banswer{
	$\frac{\partial^{2} u}{\partial x \partial y}=\frac{1}{ \left(2+e^{u}\right) } + \frac{x y}{\left(2+e^{u}\right)^{3}} \cdot e^{u}$
}


\item 	
(本题满分 10 分)	
设函数 $f(x)$ 在 $(0,+\infty)$ 内有界可微, 则:
\begin{enumerate}
	%\renewcommand{\labelenumi}{\arabic{enumi}.}
	% A(\Alph) a(\alph) I(\Roman) i(\roman) 1(\arabic)
	%设定全局标号series=example	%引用全局变量resume=example
	%[topsep=-0.3em,parsep=-0.3em,itemsep=-0.3em,partopsep=-0.3em]
	%可使用leftmargin调整列表环境左边的空白长度 [leftmargin=0em]
	\item
举例说明 $\lim\limits _{x \rightarrow+\infty} f^{\prime}(x)$ 不一定存在
\item 
对于极限 $\lim\limits _{x \rightarrow+\infty} f^{\prime}(x)$ 存在的函数, 证明 $\lim\limits _{x \rightarrow+\infty} f^{\prime}(x)=0$。
\end{enumerate}

\banswer{
	\begin{enumerate}
		%\renewcommand{\labelenumi}{\arabic{enumi}.}
		% A(\Alph) a(\alph) I(\Roman) i(\roman) 1(\arabic)
		%设定全局标号series=example	%引用全局变量resume=example
		%[topsep=-0.3em,parsep=-0.3em,itemsep=-0.3em,partopsep=-0.3em]
		%可使用leftmargin调整列表环境左边的空白长度 [leftmargin=0em]
		\item
		例:$ e^{-x} $导数的极限存在,$ \sin x $导数的极限不存在
		\item 
		证明略。设极限存在为$ A $,$ A $可为正,也可为负,分情况讨论,再利用$ \epsilon - \delta $语言推出矛盾即可证。
		
	\end{enumerate}
	
	
}


\item 	
(本题满分 10 分)	
$f(x)$ 在 $[0,2]$ 上是单调减少的连续函数,证明:
\[
\int_{0}^{2} x f(x) d x \leqslant \int_{0}^{2} f(x) d x
\]

\banswer{
	证明略。这里证明的方法有很多。例如:$ \int_{0}^{2} f(x)(1-x)dx $ \\
	①考虑$ 1-x $的对称性,将积分区间分为$ (0,1) $和$ (1,2) $\\
	②直接变量替换$ t=x-1 $,再用定积分定义即可\\
	③构造函数$ F(x)=\frac{a+x}{2}\int_{a}^{x} f(x)dx - \int_{a}^{x}xf(x)dx $,证明$ \int_{a}^{b} xf(x) \leq \frac{a+b}{2} \int_{a}^{b} f(x) dx $
}


\item 	
(本题满分 10 分)	
已知 $0 \leqslant a, b \leqslant 1$, 且 $a+b=1$, 证明:对任意实数 $x, y$ 有 $e^{a x+b y} \leq a e^{x}+b e^{y}$。

\banswer{
	证明略。证明方法也有多种,例如:\\
	①利用$ e^{x} $在$ x_{0} $处的二阶展开式,取$ x_{0}=ax+by $\\
	②用二元函数先求无条件极值得到驻点线$ y=x $,再以此求条件极值
}


\item 	
(本题满分 10 分)	
设函数 $f(x)$ 在 $[0, \pi]$ 上连续, 且 $\int_{0}^{\pi} f(x) d x=0, \int_{0}^{\pi} f(x) \cos x d x=0$ 。 证明: 在 $(0, \pi)$ 内至少存在两个不同的点 $\xi_{1}, \xi_{2}$, 使得 $f\left(\xi_{1}\right)=f\left(\xi_{2}\right)=0$。

\banswer{
	证明略。\\
	可构造函数
	\[ 
	F(x)=\int_{0}^{x} f(t)dt
	 \]
	再找到$ F(x) $异于$ 0 $、$ \pi $的一个零点。
}


\item 	
(本题满分 10 分)	
设 $f(x)$ 是 $[0,1]$ 上的连续函数,已知 $\int_{0}^{\infty} e^{-x^{2}} d x=\frac{\sqrt{\pi}}{2}$ 。证明:
\[
\lim\limits _{t \rightarrow+\infty} \int_{0}^{1} t e^{-t^{2} x^{2}} f(x) d x=\frac{\sqrt{\pi}}{2} f(0)
\]


\banswer{
	证明略。
	\[ 
	\lim\limits _{t \rightarrow+\infty} \int_{0}^{1} t e^{-t^{2} x^{2}} f(x) d x= \int_{0}^{t} e^{-u^{2}} f(\frac{u}{t}) du
	 \]
	证明方法有多种,例如:\\
①最暴力:
\[ 
\lim_{t \rightarrow \infty } \int_{0}^{t} e^{-u^{2}} f(\frac{u}{t}) du =\int_{0}^{\infty} e^{-u^{2}} f(0) du  = f(0) \int_{0}^{\infty} e^{-u^{2}}  du 
 \]
这里的做法还需要证明一系列的加强条件,详情参见数学分析广义积分部分。\\
②巧妙束缚住应用中值定理时点的范围\\
$ 
\begin{WithArrows}[ll]
	\lim\limits_{t \rightarrow \infty } \int_{0}^{t} e^{-u^{2}} f(\frac{u}{t}) du &=\lim\limits_{t \rightarrow \infty }  \left(  \int_{0}^{\sqrt{t}} e^{-u^{2}} f(\frac{u}{t}) du +\int_{\sqrt{t}}^{t} e^{-u^{2}} f(\frac{u}{t}) du \right) \Arrow{$ \epsilon \in (0,\frac{1}{\sqrt{t}}) $} \\
	&=\lim\limits_{t \rightarrow \infty }  \left( f(\epsilon)  \int_{0}^{\sqrt{t}} e^{-u^{2}}  du +\int_{\sqrt{t}}^{t} e^{-u^{2}} f(\frac{u}{t}) du \right)  \Arrow{$ f(\frac{u}{t})   $整体自变量的范围是\\ $ (\frac{1}{\sqrt{t}},1) $,$ f(x) $在其上有界} \\
	&=f(0) \frac{\sqrt{\pi}}{2} + \lim\limits_{t \rightarrow \infty } M \int_{\sqrt{t}}^{t} e^{-u^{2}}  du\\
	&=f(0) \frac{\sqrt{\pi}}{2}
\end{WithArrows} 
 $
\\
\\
③构造函数$ F(t)=\int_{0}^{t} e^{-u^{2}} f(\frac{u}{t}) du $,通过
\[ 
F(\infty) -F(0) =\int_{0}^{\infty} F ^{\prime} (t) dt
 \]
在积分中交换顺序可证。\\
④凑$ \delta(x) $函数。
\begin{note}
	方法③存在问题,不做变量替换时,求得结果为零。方法④存在问题,差一个常数。需要严格的数学分析,后期再议。
\end{note}
}
	
	
\end{enumerate}

