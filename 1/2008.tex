\bta{2008}


\begin{enumerate}
	%\renewcommand{\labelenumi}{\arabic{enumi}.}
	% A(\Alph) a(\alph) I(\Roman) i(\roman) 1(\arabic)
	%设定全局标号series=example	%引用全局变量resume=example
	%[topsep=-0.3em,parsep=-0.3em,itemsep=-0.3em,partopsep=-0.3em]
	%可使用leftmargin调整列表环境左边的空白长度 [leftmargin=0em]
	\item
填空题 (本题满分 30 分,每小题 6 分)	
\begin{enumerate}
	%\renewcommand{\labelenumi}{\arabic{enumi}.}
	% A(\Alph) a(\alph) I(\Roman) i(\roman) 1(\arabic)
	%设定全局标号series=example	%引用全局变量resume=example
	%[topsep=-0.3em,parsep=-0.3em,itemsep=-0.3em,partopsep=-0.3em]
	%可使用leftmargin调整列表环境左边的空白长度 [leftmargin=0em]
	\item
 $\lim\limits _{n \rightarrow \infty} \sum\limits_{k=1}^{n} \frac{k}{n^{2}+k}$ \tk{$  \frac{ 1 }{ 2 }  $} 。


\item
设 $y=y(x)$ 是二阶线性常系数非齐次微分方程 $y^{\prime \prime}+2 y^{\prime}+y=e^{3 x}$ 满足条件 $y(0)=0$
的解,且 $y=y(x)$ 有连续的二阶导数,则 $\lim\limits _{x \rightarrow 0} \frac{\ln \left(1+x^{2}\right)}{y(x)}=$ \tk{$ 2 $} 。


\item
已知 $f^{\prime}(x) \cdot \int_{0}^{2} f(x) d x=8$, 且 $f(0)=0$, 则 $\int_{0}^{2} f(x) d x=$ \tk{$\pm 4$} 。


\item
设 $\int_{0}^{\pi}\left[f(x)+f^{\prime \prime}(x)\right] \sin x d x=3, f(\pi)=2$, 则 $f(0)=$ \tk{$ 1 $} 。


\item
在过点 $(0,1)$ 的直线 $y=f(x)$ 中,使得积分值 $\int_{0}^{2}\left[x^{2}-(f(x))^{2}\right] d x$ 达到最大的直线 方程为 \tk{$y=-\frac{3}{4} x+1$} 。
	
	
	
	
\end{enumerate}

\item 
单选题 (本题满分 30 分,每小题 6 分)
\begin{enumerate}
	%\renewcommand{\labelenumi}{\arabic{enumi}.}
	% A(\Alph) a(\alph) I(\Roman) i(\roman) 1(\arabic)
	%设定全局标号series=example	%引用全局变量resume=example
	%[topsep=-0.3em,parsep=-0.3em,itemsep=-0.3em,partopsep=-0.3em]
	%可使用leftmargin调整列表环境左边的空白长度 [leftmargin=0em]
	\item
	二元函数 $f(x, y)=\left\{\begin{array}{ll}\frac{x y}{x^{2}+y^{2}} & (x, y) \neq(0,0) \\ 0 & (x, y)=(0,0)\end{array}\right.$ 在点 $(0,0)$ 处 \xzanswer{B} 
	
\fourchoices
{连续且两个偏导数都存在}
{不连续但两个偏导数都存在}
{连续但至少有一个偏导数不存在}
{不连续且至少有一个偏导数不存在}

\item 
如图, $f(x), g(x)$ 是两个逐段线性的连续函数, 设 $u(x)=f(g(x))$, 则 $u^{\prime}(1)$ 的值为 \xzanswer{A} 
\begin{figure}[h!]
	\centering
	\includesvg[width=0.23\linewidth]{picture/svg/601-006}
\end{figure}

\fourchoices
{$\frac{3}{4}$}
{$-\frac{3}{4}$}
{$-\frac{1}{12}$}
{$\frac{1}{12}$}



\item
方程 $x e^{-x}=\frac{1}{2 e}$ 的实根数为 \xzanswer{C} 


\fourchoices
{$ 0 $}
{$ 1 $}
{$ 2 $}
{$ 3 $}

\item 
与直线 $L_{1}\left\{\begin{array}{l}x=1 \\ y=-2+t \text { 与直线 } L_{2}: \frac{x+1}{1}=\frac{y+1}{2}=\frac{z-1}{1} \text { 都平行,且过原点 } \\ z=1+t\end{array}\right.$
的平面 $\pi$ 方程是 \xzanswer{B} 


\fourchoices
{$x+y+z=0$}
{$x-y+z=0$}
{$x+y-z=0$}
{$z-y-z=0$}

\item 
设 $x^{2}=\sum\limits_{n=0}^{\infty} a_{n} \cos n x \quad(-\pi \leqslant x \leqslant \pi)$, 则傅里叶系数 $a_{2}=$ \xzanswer{C} 


\fourchoices
{$\frac{-2}{\pi}$}
{$\frac{2}{\pi}$}
{$ 1 $}
{$-1$}




	
	
	
\end{enumerate}



\item 
(本题满分 10 分)
求 $\int_{-1}^{2}\left(|x|+2 x^{2}\right) d x$。
\banswer{
	$\frac{17}{2}$
}


\item 
(本题满分 10 分)
设曲线 $y=x^{2}$ 和直线 $y=t(0<t<1)$ 分别于 $x=0, x=1$ 所围成的面积之和为 $S(t)$, 试判断 $S(t)$ 是否存在最小值, 若存在, 求出其最小值点。

\banswer{
	$S(t)_{\min }=\frac{1}{4}$
}


\item 
(本题满分 10 分)
设 $f(x)$ 在 $[0,1]$ 上可微, $f(1)=2 \int_{0}^{\frac{1}{2}} x f(x) d x$, 求证:
存在 $\xi \in(0,1)$, 使得 $\xi f(\xi)+f(\xi)=0$。

\banswer{
	证明略
}


\item 
(本题满分 10 分)
设函数 $u=f\left(\ln \sqrt{x^{2}+y^{2}}\right)$, 满足 $\frac{\partial^{2} u}{\partial x^{2}}+\frac{\partial^{2} u}{\partial y^{2}}=\left(x^{2}+y^{2}\right)^{\frac{3}{2}}$, 试求函数 $f$ 的表达式。


\banswer{
	令 $\sqrt{x^{2}+y^{2}}=e^{t}$,得$u(t)=\frac{1}{25} e^{5 t}+C_{1} t+C_{2}$
}


\newpage
\item 
(本题满分 10 分)
计算二重积分 $I=\iint_{D} \sqrt{1-y^{2}} d x d y$, 其中 $D$ 为 $x^{2}+y^{2}=1(y>0)$ 与 $y=|x|$ 围成区域。

\banswer{
	$2-\sqrt{2}$
}



\item 
(本题满分 10 分)
求微分方程 $y^{\prime}+y=y^{2}(\cos x-\sin x)$ 的通解。

\banswer{
	$u(x)=-\sin x e^{x}+C$
}



\item 
(本题满分 10 分)
求级数 $\sum\limits_{n=1}^{\infty}(-1)^{n-1} n(n+1) x^{n}$ 在收敛区间 $(-1,1)$ 内的核函数 $S(x)$, 并求数项级数 $\sum\limits_{n=1}^{\infty}(-1)^{n-1} \frac{n(n+1)}{3^{n}}$ 的和。

\banswer{
	$ S(x) =\frac{2 x}{(1+x)^{3}}$,$\sum_{n=1}^{\infty}(-1)^{n-1} n(n+1)\left(\frac{1}{3}\right)^{n}=\frac{9}{32}$
}


\item 
(本题满分 10 分)
已知函数 $f(x)$ 在 $[0,+\infty)$ 上可导, $f(0)=1$, 且满足不等式 $f^{\prime}(x)+f(x)-\frac{1}{1+x} \int_{0}^{x} f(t) d t=$ 0 , 求 $f^{\prime}(x)$ 的函数表达式, 并证明不等式:
\[
e^{-x} \leqslant f(x) \leqslant 1 \quad(x \geqslant 0)
\]


\banswer{
	证明略
}



\item 
(本题满分 10 分)
设函数 $f(x), g(x)$ 具有二阶连续导数,曲线积分
\[ 
\oint_{C}\left[y^{2} f(x)+2 y e^{x}+2 y g(x)\right] d x+2[y g(x)+f(x)] d y=0
 \]
其中 $C$ 为平面上任一简单封闭曲线。求上式中的 $f(x)$ 和 $g(x)$ 使得 $f(0)=g(0)=0$。


\banswer{
$g(x)=C_{1} e^{x}+C_{2} e^{-x}+\frac{1}{2} x e^{x}$,$f(x)=g^{\prime}(x)=C_{1} e^{x}-C_{2} e^{-x}+\frac{1}{2}\left(x e^{x}+e^{x}\right)$	
}

	
	
\end{enumerate}


