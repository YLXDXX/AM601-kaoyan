\bta{2002}

\begin{enumerate}
	%\renewcommand{\labelenumi}{\arabic{enumi}.}
	% A(\Alph) a(\alph) I(\Roman) i(\roman) 1(\arabic)
	%设定全局标号series=example	%引用全局变量resume=example
	%[topsep=-0.3em,parsep=-0.3em,itemsep=-0.3em,partopsep=-0.3em]
	%可使用leftmargin调整列表环境左边的空白长度 [leftmargin=0em]
	\item
	填空题(本题共 5 小题,每小题 3 分,满分 15 分)
	\begin{enumerate}
		%\renewcommand{\labelenumi}{\arabic{enumi}.}
		% A(\Alph) a(\alph) I(\Roman) i(\roman) 1(\arabic)
		%设定全局标号series=example	%引用全局变量resume=example
		%[topsep=-0.3em,parsep=-0.3em,itemsep=-0.3em,partopsep=-0.3em]
		%可使用leftmargin调整列表环境左边的空白长度 [leftmargin=0em]
		\item
	 $\lim\limits _{n \rightarrow \infty} \cos \frac{1}{2} \cos \frac{1}{4} \cdots \cos \frac{1}{2^{n}}=$ \tk{} 。
		
		
		\item
		 $\int \frac{\cos x d x}{1+e^{\sin x}}$ \tk{} 。
		
		
		\item
		设 $z=z(x, y)$, 且有 $y z+z x+x y=1$, 则 $d z=$ \tk{} 。
		
		
		\item
		含参变量广义积分 $\int_{0}^{+\infty} \frac{\sin \frac{1}{x}}{x^{p}} d x$ 的收敛域为 \tk{} 。
		
		
		\item
		微分方程 $y^{\prime \prime}+4 y^{\prime}+4 y=e^{x}$ 的通解为 \tk{} 。
		
		
		
		
	\end{enumerate}

\item 	
选择题 (本题共 5 小题,每小题 3 分,满分 15 分)	
\begin{enumerate}
	%\renewcommand{\labelenumi}{\arabic{enumi}.}
	% A(\Alph) a(\alph) I(\Roman) i(\roman) 1(\arabic)
	%设定全局标号series=example	%引用全局变量resume=example
	%[topsep=-0.3em,parsep=-0.3em,itemsep=-0.3em,partopsep=-0.3em]
	%可使用leftmargin调整列表环境左边的空白长度 [leftmargin=0em]
	\item
设 $\left\{\begin{array}{ll}a x+b & x \leqslant 9 \\ x^{2} \cos \frac{1}{x} & x>0\end{array}\right.$ 在 $x=0$ 处可导, 则 \xzanswer{} 

	
\fourchoices
{$a=1, b=0$}
{$a=0, b=0$}
{$a=1, b=1$}
{$a=0, b=1$}

	
\item 
$f(x)$ 在区间 $[-L, L]$ 内具有二阶导数,且 $f^{\prime \prime}(x)>0, \lim\limits _{x \rightarrow 0} \frac{f(x)}{x}=1$, 则 \xzanswer{} 


\fourchoices
{在 $(-L, 0)$ 和 $0, L$ 内均有 $f(x)>x$}
{在 $(-L, 0)$ 和 $0, L$ 内均有 $f(x)<x$}
{在 $(-L, 0)$ 内, $f(x)>x$; 在 $0, L$ 内, $f(x)<x$}
{在 $(-L, 0)$ 内, $f(x)<x$; 在 $0, L$ 内, $f(x)>x$}



\item
设 $L$ 为圆周 $\left\{\begin{array}{l}x^{2}+y^{2}+z^{2}=a^{2} \quad, \text { 则曲线积分 } \int_{L}\left(x^{4}+2 y^{2} z^{2}\right) d L= \ x+y+y=0\end{array}\right.$ \xzanswer{} 


\fourchoices
{$\frac{\pi a^{5}}{3}$}
{$\frac{2 \pi a 5}{3}$}
{$\pi a^{5}$}
{$2 \pi a^{5}$}



\item
下列级数中, 绝对收敛的级数是 \xzanswer{} 


\fourchoices
{$\sum\limits_{n=1}^{\infty}\left(e^{\frac{1}{n}}-1\right)$}
{$\sum\limits_{n=1}^{\infty} \frac{\sin n}{n}$}
{$\sum\limits_{n=1}^{\infty} \frac{(-1)^{n}}{e^{\sqrt{n}}+1}$}
{$\sum\limits_{n=1}^{\infty} \frac{(-1)^{n}}{\sqrt[n]{e}+1}$}


\item 
$ a+|x|=\pi-\frac{4}{\pi} \sum\limits_{n=1}^{\infty} \frac{\cos (2 n-1) x}{(2 n-1)^{2}}$, 其中 $-\pi \leqslant x \leqslant \pi, a$ 为常数, 则 $a=$ \xzanswer{} 


\fourchoices
{$\frac{\pi}{2}$}
{$-\frac{\pi}{2}$}
{$\pi$}
{$-\pi$}



	
\end{enumerate}


\newpage
\item 
(3 小题,每小题 6 分,共 18 分)
\begin{enumerate}
	%\renewcommand{\labelenumi}{\arabic{enumi}.}
	% A(\Alph) a(\alph) I(\Roman) i(\roman) 1(\arabic)
	%设定全局标号series=example	%引用全局变量resume=example
	%[topsep=-0.3em,parsep=-0.3em,itemsep=-0.3em,partopsep=-0.3em]
	%可使用leftmargin调整列表环境左边的空白长度 [leftmargin=0em]
	\item
计算定积分 $\int_{-\frac{\pi}{2}}^{\frac{\pi}{2}} \frac{\cos x}{1+e^{x}} d x$。
	
	
	\item
	计算积分 $\int_{0}^{1}(1-\sqrt{x})^{n} d x$。
	
	
	\item
	设函数 $f(x)$ 满足 $f^{\prime \prime}(x)+\left[f^{\prime}(x)\right]^{2}=\sin x$, 且 $f^{\prime}(0)=0$ 。证明: $x=0$ 是 $f(x)$ 的拐点。
	
	
	
	
\end{enumerate}


\banswer{
	
}



\item 
(4 小题,每小题 7 分,共 28 分)
\begin{enumerate}
	%\renewcommand{\labelenumi}{\arabic{enumi}.}
	% A(\Alph) a(\alph) I(\Roman) i(\roman) 1(\arabic)
	%设定全局标号series=example	%引用全局变量resume=example
	%[topsep=-0.3em,parsep=-0.3em,itemsep=-0.3em,partopsep=-0.3em]
	%可使用leftmargin调整列表环境左边的空白长度 [leftmargin=0em]
	\item
计算极限 $\lim\limits _{n \rightarrow \infty} \sqrt[n]{1+e^{n x}+e^{-n x}}$。
	
	
	\item
	计算曲线积分 $I=\oint_{L} z^{2} d x+\left(x^{2}+x y-x\right) d y+2 x z d z$, 其中 $L$ 是抛物面 $z=x^{2}+y^{2}$ 与圆柱面 $x^{2}+4 y^{2}=1$ 的交线。从 $z$ 轴正方向向下看, $L$ 为顺时针方向。
	
	
	\item
	把 $y=\arctan \frac{3+x}{3-x}$ 展为 $x$ 的幂级数, 并求收敛域。
	
	
	\item
	求微分方程 $\left(x-x^{3} y^{2} \ln y\right) y^{\prime}=2 y$ 的通解。
	
	
	
	
\end{enumerate}

\banswer{
	
}


\newpage
\item 
(3 小题,每小题 8 分,共 24 分)
\begin{enumerate}
	%\renewcommand{\labelenumi}{\arabic{enumi}.}
	% A(\Alph) a(\alph) I(\Roman) i(\roman) 1(\arabic)
	%设定全局标号series=example	%引用全局变量resume=example
	%[topsep=-0.3em,parsep=-0.3em,itemsep=-0.3em,partopsep=-0.3em]
	%可使用leftmargin调整列表环境左边的空白长度 [leftmargin=0em]
	\item
设曲面 $S: \sqrt{x}+\sqrt{y}+\sqrt{z}=\sqrt{a}(a>0)$, 在 $S$ 上求一切平面, 使此切面与三坐标 所围成的四面体体积最大,并求四面体体积的最大值。
	
\item
设 $f(x)$ 是以 $2 \pi$ 为周期的函数, 在 $(\pi, \pi]$ 内的表达式为	
\[
f(x)=\left\{\begin{array}{ll}
	0 & -\pi<x<0 \\
	1 & 0 \leqslant x \leqslant 1 \\
	0 & 1<x \leqslant \pi
\end{array}\right.
\]	
将 $f(x)$ 展为傅里叶级数 (说明收敛情况), 并求 $\sum\limits_{n=1}^{\infty} \frac{\sin n}{n}$ 与 $\sum\limits_{n=1}^{\infty} \frac{1-\cos n}{n^{2}}$。 

\item
设区域 $\Omega$ 由曲面 $x=0, y=0, x+y=1, z(x+y)=1$ 及 $z=1$ 围成。
\begin{enumerate}
	%\renewcommand{\labelenumi}{\arabic{enumi}.}
	% A(\Alph) a(\alph) I(\Roman) i(\roman) 1(\arabic)
	%设定全局标号series=example	%引用全局变量resume=example
	%[topsep=-0.3em,parsep=-0.3em,itemsep=-0.3em,partopsep=-0.3em]
	%可使用leftmargin调整列表环境左边的空白长度 [leftmargin=0em]
	\item
求 $\Omega$ 的体积 $V$。
\item 
证明 $\iiint_{\Omega} \frac{d x d y d z}{x^{2}+y^{2}+z^{2}} \leqslant \frac{V}{2}$。
	
	
	
\end{enumerate}




	
\end{enumerate}

\banswer{
	
}


	
\end{enumerate}


