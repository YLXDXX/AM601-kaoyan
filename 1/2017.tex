\bta{2017}


\begin{enumerate}
	%\renewcommand{\labelenumi}{\arabic{enumi}.}
	% A(\Alph) a(\alph) I(\Roman) i(\roman) 1(\arabic)
	%设定全局标号series=example	%引用全局变量resume=example
	%[topsep=-0.3em,parsep=-0.3em,itemsep=-0.3em,partopsep=-0.3em]
	%可使用leftmargin调整列表环境左边的空白长度 [leftmargin=0em]
	\item
	选择题 (本题满分 50 分,每小题 5 分)
\begin{enumerate}
	%\renewcommand{\labelenumi}{\arabic{enumi}.}
	% A(\Alph) a(\alph) I(\Roman) i(\roman) 1(\arabic)
	%设定全局标号series=example	%引用全局变量resume=example
	%[topsep=-0.3em,parsep=-0.3em,itemsep=-0.3em,partopsep=-0.3em]
	%可使用leftmargin调整列表环境左边的空白长度 [leftmargin=0em]
	\item
	函数 $f(x)=|x| \sin x^{2}$, 正确的结论是 \xzanswer{C} 
	
	
	



\fourchoices
{在 $(-\infty,+\infty)$ 内有界}
{当 $x \rightarrow \infty$ 时, $f(x)$ 为无穷大}
{在 $(-\infty,+\infty)$ 内处处可导}
{当 $x \rightarrow \infty$ 时, $f(x)$ 极限存在}


\item 
$ \lim\limits _{n \rightarrow \infty}\left(\frac{1}{n^{4}+n^{3}+n^{2}+1}+\frac{2}{n^{4}+n^{3}+n^{2}+2}+\cdots+\frac{n^{2}}{n^{4}+n^{3}+n^{2}+n^{2}}\right)= $ \xzanswer{B} 


\fourchoices
{$ 0 $}
{$\frac{1}{2}$}
{$ 1 $}
{$\infty$}

\item 
设 $f(x)=(x-1)(x-2)^{2}(x-3)^{3}$, 则 $f^{\prime}(1)+f^{\prime \prime}(2)+f^{\prime \prime \prime}(3)=$ \xzanswer{D} 


\fourchoices
{$ 0 $}
{$-2$}
{$-8$}
{$-10$}

\item 
设 $f(x)=x \sin x+\cos x-\frac{\pi}{2}$, 则下列命题中正确的是 \xzanswer{B} 


\fourchoices
{$f(0)$ 是极大值, $f\left(\frac{\pi}{2}\right)$ 是极小值}
{$f(0)$ 是极小值, $f\left(\frac{\pi}{2}\right)$ 是极大值}
{$f(0)$ 是极大值, $f\left(\frac{\pi}{2}\right)$ 是极大值}
{$f(0)$ 是极小值, $f\left(\frac{\pi}{2}\right)$ 是极小值}


\item 
求极限 $\lim\limits _{x \rightarrow 0}(1-3 x)^{\frac{\cos x^{2}}{\sin x}}=$ \xzanswer{C} 


\fourchoices
{$e^{-1}$}
{$e^{-2}$}
{$e^{-3}$}
{$ 1 $}

\item 
定积分 $\int_{0}^{\pi}(x \sin x)^{2} d x=$ \xzanswer{B} 


\fourchoices
{$\frac{\pi^{2}}{3}-\frac{\pi}{4}$}
{$\frac{\pi^{3}}{6}-\frac{\pi}{4}$}
{$\frac{\pi^{3}}{6}-\frac{\pi^{2}}{4}$}
{$\frac{\pi^{3}}{6}+\frac{\pi}{4}$}

\item 
设 $D_{K}$ 是圆域 $D=\left\{(x, y) \mid x^{2}+y^{2} \leqslant 1\right\}$ 在第 $k$ 象限的部分, $k=1,2,3,4$ 。 记 $I_{k}=\iint_{D_{k}}(y-x) d x d y$, 则正确的是 \xzanswer{D} 


\fourchoices
{$I_{1}<0$}
{$I_{2}<0$}
{$I_{3}<0$}
{$I_{4}<0$}

\item 
已知 $y_{1}(x)$ 和 $y_{2}(x)$ 是微分方程 $y^{\prime}+p(x) y=0$ 的两个不同的特解,则方程的通 解一定为 \xzanswer{D} 


\fourchoices
{$y=C y_{1}(x)$}
{$y=Cy_{2}(x)$}
{$y=C\left(y_{1}(x)+y_{2}(x)\right)$}
{$y=C\left(y_{1}(x)-y_{2}(x)\right)$}

\item 
设级数 $\sum\limits_{n=1}^{+\infty} u_{n}$ 收敛,则必收敛的级数为 \xzanswer{D} 


\fourchoices
{$\sum\limits_{n=1}^{+\infty}(-1)^{n} \frac{u_{n}}{n}$}
{$\sum\limits_{n=1}^{+\infty} u_{n}^{2}$}
{$\sum\limits_{n=1}^{+\infty}\left(u_{2 n-1}-u_{2 n}\right)$}
{$\sum\limits_{n=1}^{+\infty}\left(u_{n}+u_{n+1}\right)$}

\item 
设 $f^{\prime}\left(\sin ^{2} x\right)=\cos 2 x$, 且 $f(0)=1$, 则 $\int_{0}^{1} f(x) d x=$ \xzanswer{C} 


\fourchoices
{$ 1 $}
{$\frac{1}{6}$}
{$\frac{7}{6}$}
{$\frac{1}{2}$}



\end{enumerate}


\item 
(本题满分 10 分)
计算 $\lim\limits _{x \rightarrow 0} \frac{[\sin x-\sin (\sin x)] \sin x}{x^{4} \cos x^{2}}$。

\banswer{
	$  \frac{ 1 }{ 6 }  $
}



\item 
(本题满分 10 分)
证明: $\lim\limits _{n \rightarrow \infty} \int_{0}^{\pi} \sin ^{n} t d t=0$。


\banswer{
	证明略。构造函数$ F(n,\epsilon)=\int_{0}^{\epsilon} \sin^{n} t dt + \frac{\pi}{2} -\epsilon $,有
\[ 
\begin{WithArrows}[displaystyle]
	\lim\limits _{n \rightarrow \infty} \int_{0}^{\pi} \sin ^{n} t d t &=2 \cdot \lim\limits _{n \rightarrow \infty} \lim\limits _{\epsilon \rightarrow \frac{\pi}{2}}  F(n,\epsilon) \Arrow{交换极限顺序可证}\\
	&=2 \cdot \lim\limits _{\epsilon \rightarrow \frac{\pi}{2}}  \lim\limits _{n \rightarrow \infty}  F(n,\epsilon)
\end{WithArrows} 
\]	
也可以利用$ F(n,\epsilon)=\int_{0}^{\epsilon} \sin^{n} t dt + \int_{\epsilon}^{1} \sin^{n} t dt   $来证明。
}


\item 
(本题满分 10 分)
把函数 $f(x)=(x-1)^{2}$ 在 $(0,1)$ 上展成余弦级数,并求 $\sum\limits_{n=1}^{+\infty} \frac{1}{n^{2}}$。


\banswer{
$f(x)=\frac{a_{0}}{2}+\sum\limits_{n=1}^{\infty} a_{n} \cos n \pi x=\frac{1}{3}+\frac{4}{\pi^{2}} \sum\limits_{n=1}^{\infty} \frac{\cos n \pi x}{n^{2}}$,利用$ f(0)=1 $,得
$\sum\limits_{n=1}^{\infty} \frac{1}{n^{2}}=\frac{\pi^{2}}{6} $
}


\item 
(本题满分 10 分)
计算曲面积分
\[
I=\iint_{\Sigma} 2 x^{3} d y d z+2 y^{3} d z d x+3\left(z^{2}-2\right) d x d y
\]
其中 $\Sigma$ 为曲面 $z=1-x^{2}-y^{2}(z \geqslant 0)$ 的上侧。


\banswer{
	$ I=-4\pi $
}



\item 
(本题满分 10 分)
计算 $\iint_{\Omega} \frac{(x+y) \ln \left(1+\frac{y}{x}\right)}{\sqrt{1-x-y}} d x d y$, 这里 $\Omega$ 是由直线 $x+y=1, x=0, y=0$ 所围成的 三角形区域。

\banswer{
	$ I=\frac{16}{15} $。采取特殊路径,或等效使用变换$ \left\{  
	\begin{array}{l}
		t=x+y\\
		s=x
	\end{array}
	\right. $
}


\newpage
\item 
(本题满分 10 分)
求微分方程 $y^{\prime \prime}\left(x+\left(y^{\prime}\right)^{2} \right)=y^{\prime}$ 满足初始条件 $y(1)=y^{\prime}(1)=1$ 的特解。


\banswer{
	$ y= \frac{ 2 }{ 3 } x^{3/2} + \frac{ 1 }{ 3 }  $
}


\item 
(本题满分 10 分)
计算曲线积分 $I=\oint_{L} \frac{x d y-y d x}{4 x^{2}+y^{2}}$, 其中 $L$ 是以点 $(1,0)$ 为中心, $ 2  $为半径的圆周, 取 逆时针方向。


\banswer{
	$ I=\pi $
}


\item 
(本题满分 10 分)
设 $x \geqslant 0, a>0$, 若 $\sqrt{x+a}-\sqrt{x}=\frac{a}{2 \sqrt{x+\phi_{a}(x)}}$ 成立,证明: $\frac{a}{4} \leqslant \phi_{a}(x) \leqslant \frac{a}{2}$ 且 $\lim\limits _{x \rightarrow 0} \phi_{a}(x)=\frac{a}{4}$ 和 $\lim\limits _{x \rightarrow \infty} \phi_{a}(x)=\frac{a}{2}$。

\banswer{
	证明略。可化简为$ \phi_{a}(x)=\frac{a}{4} +\frac{a}{2} \left( \sqrt{(1+\frac{x}{a} ) \frac{x}{a}} -\frac{x}{a} \right) $,设$ f(t)=\sqrt{(1+t)t}-t $,可证。
}


\item 
(本题满分 10 分)
函数 $f(x)$ 在 $[0,2]$ 上连续, 且为增函数,证明:
\[
\int_{0}^{2} f(x) d x \leq \int_{0}^{2} x f(x) d x
\]

\banswer{
	证明略。利用$ x-1 $的性质,区间分成$ (0,1) $和$ (1,2) $,可证。
}

\item 
(本题满分 10 分)
设 $f(x)$ 在 $[a, b]$ 上可导且 $f^{\prime}(x)$ 连续, $f(a)=0$, 这里 $a<b$。 证明:
\[
\int_{a}^{b} \left[f(x)\right] ^{2} d x \leq \frac{(b-a)^{2}}{2} \int_{a}^{b}\left[f^{\prime}(x)\right]^{2} d x
\]
	
\banswer{
	证明略。利用柯西不等式的积分形式
	\[ 
	f^{2}(x)=\left( \int_{a}^{x} f ^{\prime} (x)dx \right)^{2} \leqslant \int_{a}^{x} f ^{\prime}{}^{2} (x)dx \int_{a}^{x} 1 ^{2}dx
	 \]
}
	
\end{enumerate}

