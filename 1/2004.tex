\bta{2004}


\begin{enumerate}
	%\renewcommand{\labelenumi}{\arabic{enumi}.}
	% A(\Alph) a(\alph) I(\Roman) i(\roman) 1(\arabic)
	%设定全局标号series=example	%引用全局变量resume=example
	%[topsep=-0.3em,parsep=-0.3em,itemsep=-0.3em,partopsep=-0.3em]
	%可使用leftmargin调整列表环境左边的空白长度 [leftmargin=0em]
	\item
	填空题 (本题共 5 题,每小题 5 分,满分 25 分)
	\begin{enumerate}
		%\renewcommand{\labelenumi}{\arabic{enumi}.}
		% A(\Alph) a(\alph) I(\Roman) i(\roman) 1(\arabic)
		%设定全局标号series=example	%引用全局变量resume=example
		%[topsep=-0.3em,parsep=-0.3em,itemsep=-0.3em,partopsep=-0.3em]
		%可使用leftmargin调整列表环境左边的空白长度 [leftmargin=0em]
		\item
	$\lim\limits _{n \rightarrow \infty} \sqrt[n]{\sin 1+\sin \frac{1}{2}+\cdots+\sin \frac{1}{n}}=$ \tk{$ 1 $} 。
		
		
		\item
		设 $y-\epsilon \sin y=x$ (常数 $\epsilon \in(0,1))$, 则 $\frac{d^{2} y}{d x^{2}}=$ \tk{$ -\frac{\epsilon \sin y}{(1-\epsilon \cos y)^{3}} $} 。
		
		
		\item
		积分 $\int_{0}^{+\infty} \frac{\ln \left(1+x^{2}\right)}{x^{\alpha}} d x$ 的收敛域为 \tk{$ 1< \alpha <3 $} 。
		
		
		\item
		 曲面 $z=\arctan \frac{y}{x}$ 在点 $\left(1,1, \frac{\pi}{4}\right)$ 处的切平面方程为 \tk{$ x-y+2z=\frac{\pi}{2} $} 。
		
		
		\item
		微分方程 $y^{\prime \prime}-3 y^{\prime}+2 y=\cos x$ 的通解为 \tk{$ y= \frac{1}{10}(\cos x -3\sin x )+C_{1}e^{x}+C_{2}e^{2x} $} 。
		
		
		
		
	\end{enumerate}
	
\item 
单项选择题 (本题共 5 题,每小题 5,满分 25 分)	
\begin{enumerate}
	%\renewcommand{\labelenumi}{\arabic{enumi}.}
	% A(\Alph) a(\alph) I(\Roman) i(\roman) 1(\arabic)
	%设定全局标号series=example	%引用全局变量resume=example
	%[topsep=-0.3em,parsep=-0.3em,itemsep=-0.3em,partopsep=-0.3em]
	%可使用leftmargin调整列表环境左边的空白长度 [leftmargin=0em]
	\item
设 $S$ 为球面 $x^{2}+y^{2}+z^{2}=R^{2}$ 外侧,则 $\oiint_{S} x^{2} d y d z+y^{2} d z d x+z d x d y=$ \xzanswer{A} 

	
\fourchoices
{$ 0 $}
{$\pi R^{4}$}
{$2 \pi R^{4}$}
{$4 \pi R^{4}$}

	
	
	\item
	曲线 $y=\sqrt{x^{2}+1}-x-1$ 的渐进线条数为 \xzanswer{C} 
	
	
\fourchoices
{$ 0 $}
{$ 1 $}
{$ 2 $}
{$ 3 $}


\item 
给定严格递增数列 $\left\{A_{n}\right\}$, 且 $A_{1}=a, \lim\limits _{n \rightarrow \infty} A_{n}=+\infty$, 函数 $f(x)$ 在 $[a,+\infty)$ 上连
续且非负,则积分 $\int_{a}^{+\infty} f(x) d x$ 收敛是级数 $\sum\limits_{n=1}^{\infty} \int_{A_{n}}^{A_{n+1}} d x$ 收敛的 \xzanswer{C} 



\fourchoices
{充分条件但非必要条件}
{必要条件但非充分条件}
{充分必要条件}
{既非充分条件也非必要条件}

\item 
如果级数 $\sum\limits_{n=2}^{\infty} \ln \left[1+\frac{(-1)^{n}}{n^{p}}\right](p>0)$ 条件收敛,则 \xzanswer{D} 


\fourchoices
{$0<\leqslant 1$}
{$p>1$}
{$\frac{1}{3}<p \leqslant 1$}
{$\frac{1}{2}<p<\leqslant 1$}


\item 
设$f(x, y)=\left\{\begin{array}{ll}\left(x^{2}+y^{2}\right) \sin \frac{1}{x^{2}+y^{2}} & (x, y) \neq(0,0) \\ 0 & (x, y)=(0,0)\end{array}\right.$,则下列选项正确的是 \xzanswer{D} 

\fourchoices
{$f(x, y)$ 在 $(0,0)$ 处不可微, $\frac{\partial f}{\partial x}, \frac{\partial f}{\partial y}$ 在 $(0,0)$ 处连续}
{$f(x, y)$ 在 $(0,0)$ 处不可微, $\frac{\partial f}{\partial x}, \frac{\partial f}{\partial y}$ 在 $(0,0)$ 处不连续}
{$f(x, y)$ 在 $(0,0)$ 处可微, $\frac{\partial f}{\partial x}, \frac{\partial f}{\partial y}$ 在 $(0,0)$ 处连续}
{$f(x, y)$ 在 $(0,0)$ 处可微, $\frac{\partial f}{\partial x}, \frac{\partial f}{\partial y}$ 在 $(0,0)$ 处不连续}

	
	
\end{enumerate}


\item 
(本题共 5 题,每小题 8 分,满分 40 分)	
\begin{enumerate}
	%\renewcommand{\labelenumi}{\arabic{enumi}.}
	% A(\Alph) a(\alph) I(\Roman) i(\roman) 1(\arabic)
	%设定全局标号series=example	%引用全局变量resume=example
	%[topsep=-0.3em,parsep=-0.3em,itemsep=-0.3em,partopsep=-0.3em]
	%可使用leftmargin调整列表环境左边的空白长度 [leftmargin=0em]
	\item
	计算极限 $\lim\limits _{x \rightarrow 0} \frac{\int_{0}^{\tan x} t(\tan t-t) d t}{\int_{0}^{\sin ^{2} x} \sin ^{\frac{3}{2}} t d t}$。
	\item 
计算积分 $\int_{0}^{\ln 2} \sqrt{e^{x}-1} d x$。
	
	
	\item
	利用欧拉积分计算 $\int_{0}^{\frac{\pi}{2}}(\tan x)^{\frac{2}{3}} d x$。
	
	
	



\item
利用 Stokes 公式计算
\[ 
\oint_{L}(y-z) d x+(z-x) d y+(x-y) d z
 \]
 其中$L:\left\{\begin{array}{l}x^{2}+y^{2}+z^{2}=a^{2} \\ x+y+z=0\end{array}\right.$ $ (a>0) $,从 $x$ 轴正向看 $L$ 为逆时针走向。


\item
 设 $a, b>0$ 。 证明:当 $y>x>0$ 时,有 $\left(a^{x}+b^{x}\right)^{\frac{1}{x}}>\left(a^{y}+b^{y}\right)^{\frac{1}{y}}$。

\end{enumerate}


\banswer{
	\begin{enumerate}
		%\renewcommand{\labelenumi}{\arabic{enumi}.}
		% A(\Alph) a(\alph) I(\Roman) i(\roman) 1(\arabic)
		%设定全局标号series=example	%引用全局变量resume=example
		%[topsep=-0.3em,parsep=-0.3em,itemsep=-0.3em,partopsep=-0.3em]
		%可使用leftmargin调整列表环境左边的空白长度 [leftmargin=0em]
		\item
		$  \frac{ 1 }{ 6 }  $
		\item 
		$ 2-\frac{\pi}{2} $		
		\item 
		$ \pi $
		\item 
		$ -2\sqrt{3}\pi a^{2} $
		\item 
		证明略
	\end{enumerate}
	
	
}


\item 
(本题共 3 题,每小题 12 分,满分 36 分)
\begin{enumerate}
	%\renewcommand{\labelenumi}{\arabic{enumi}.}
	% A(\Alph) a(\alph) I(\Roman) i(\roman) 1(\arabic)
	%设定全局标号series=example	%引用全局变量resume=example
	%[topsep=-0.3em,parsep=-0.3em,itemsep=-0.3em,partopsep=-0.3em]
	%可使用leftmargin调整列表环境左边的空白长度 [leftmargin=0em]
	\item
求曲面 $x^{2}+y^{2}=a z$ 和 $z=2 a-\sqrt{x^{2}+y^{2}}(a>0)$ 所围成的立体体积。
	
	
	\item
	求级数 $\sum\limits_{n=1}^{\infty}\left[n(n+1)-\frac{1}{n(n+1)}\right] x^{n}$ 的和函数, 并求收敛域。
	
	
	\item
	求 $k$ 的取值范围, 使得方程 $\frac{k}{r}+x^{2}=1$ 有唯一正根。
	
	
	
	
\end{enumerate}



\banswer{
	\begin{enumerate}
		%\renewcommand{\labelenumi}{\arabic{enumi}.}
		% A(\Alph) a(\alph) I(\Roman) i(\roman) 1(\arabic)
		%设定全局标号series=example	%引用全局变量resume=example
		%[topsep=-0.3em,parsep=-0.3em,itemsep=-0.3em,partopsep=-0.3em]
		%可使用leftmargin调整列表环境左边的空白长度 [leftmargin=0em]
		\item
		$ V= \frac{ 5 }{ 6 } \pi a^{3} $
		\item 
		$\sum\limits_{n=1}^{\infty}\left[n(n+1)-\frac{1}{n(n+1)}\right] x^{n}=
		\left\{
		\begin{aligned}
			&0&(x=0)\\
			&\frac{2x}{(1-x)^{3}}-(1-x)\frac{\ln (1-x)}{x} -1 & (-1<x<1)
		\end{aligned}\right.
		$	
		\item 
		$ (-\infty, 0] \cup \{ \frac{2\sqrt{3}}{9} \} $
	\end{enumerate}
	
	
}


\newpage
\item 
(本题共 2 题,每题 12 分,满分 24 分)
\begin{enumerate}
	%\renewcommand{\labelenumi}{\arabic{enumi}.}
	% A(\Alph) a(\alph) I(\Roman) i(\roman) 1(\arabic)
	%设定全局标号series=example	%引用全局变量resume=example
	%[topsep=-0.3em,parsep=-0.3em,itemsep=-0.3em,partopsep=-0.3em]
	%可使用leftmargin调整列表环境左边的空白长度 [leftmargin=0em]
	\item
	将函数 $f(x)=\left\{\begin{array}{lc}x & -\pi \leqslant x \leqslant 0 \\ \pi x & 0<x \leqslant \pi\end{array}\right.$ 展开成傅里叶级数 (说明收敛情况), 并求$\sum\limits_{n=1}^{\infty} \frac{1}{n^{2}}$。
	
	
	\item 
确定常数 $\lambda$, 使得 $\frac{x}{y}\left(x^{2}+y^{2}\right)^{\lambda} d x-\frac{x^{2}}{y^{2}}\left(x^{2}+y^{2}\right)^{\lambda} d y=0$ 在 $D=\{(x, y) \mid y>0\}$ 内 为一全微分方程, 并利用曲线积分求此全微分方程的通解。
	
	
\end{enumerate}

\begin{enumerate}
	%\renewcommand{\labelenumi}{\arabic{enumi}.}
	% A(\Alph) a(\alph) I(\Roman) i(\roman) 1(\arabic)
	%设定全局标号series=example	%引用全局变量resume=example
	%[topsep=-0.3em,parsep=-0.3em,itemsep=-0.3em,partopsep=-0.3em]
	%可使用leftmargin调整列表环境左边的空白长度 [leftmargin=0em]
	\item
	$ f(x) $可展开为
\[ 
	\begin{aligned}
	f(x)&=\frac{\pi}{4}(\pi-1)+\sum\limits_{n=1}^{\infty}\left\{-\frac{\pi-1}{n^{2} \pi}\left[\left(1-(-1)^{n}\right] \cos n x+\frac{\pi+1}{n}(-1)^{n+1} \sin n x\right\}\right. \\
	&=\left\{\begin{array}{ll}x, & -\pi<x \leqslant 0 \\ \pi x, & 0 \leqslant x<\pi \\ \frac{\pi(\pi-1)}{2},  & x=\pm \pi\end{array}\right.
\end{aligned}
 \]
利用$ x=0 $的值有$\sum\limits_{n=1}^{\infty} \frac{1}{n^{2}}=\frac{\pi^{2}}{8}$	
	
\item 
$ \lambda= \frac{ 1 }{ 2 }  $,通解为$ \frac{\sqrt{x^{2}+y^{2}}}{y} =C $
\end{enumerate}


	
\end{enumerate}


