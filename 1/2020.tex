\bta{2020}

\begin{enumerate}
	%\renewcommand{\labelenumi}{\arabic{enumi}.}
	% A(\Alph) a(\alph) I(\Roman) i(\roman) 1(\arabic)
	%设定全局标号series=example	%引用全局变量resume=example
	%[topsep=-0.3em,parsep=-0.3em,itemsep=-0.3em,partopsep=-0.3em]
	%可使用leftmargin调整列表环境左边的空白长度 [leftmargin=0em]
	\item
	选择题(本题满分 50 分,每小题 5 分)
	\begin{enumerate}
		%\renewcommand{\labelenumi}{\arabic{enumi}.}
		% A(\Alph) a(\alph) I(\Roman) i(\roman) 1(\arabic)
		%设定全局标号series=example	%引用全局变量resume=example
		%[topsep=-0.3em,parsep=-0.3em,itemsep=-0.3em,partopsep=-0.3em]
		%可使用leftmargin调整列表环境左边的空白长度 [leftmargin=0em]
		\item
		极限 $\lim\limits _{n \rightarrow+\infty}\left(\frac{2}{n^{2}}+\frac{4}{n^{2}+1}+\cdots+\frac{2 n}{n^{2}+n-1}\right)$ 的值为 \xzanswer{B} 
		
		
\fourchoices
{$ 0 $}
{$ 1 $}
{$ 2 $}
{$+\infty$}

\item 
设 $f(x)$ 在 $x=a$ 处连续, $F(x)=f(x)|x-a|$, 则 $f(a)=0$ 是 $F(x)$ 在 $x=a$ 处可导的 \xzanswer{A} 


\fourchoices
{充要条件}
{充分非必要条件}
{必要非充分条件}
{既非充分又非必要条件}

		
\item 
极限$\lim\limits _{x \rightarrow \frac{\pi}{2}}(\sin x)^{\frac{1}{1-\sin x}}$的值为 \xzanswer{D}



\fourchoices
{$e^{-1}$}
{$ 1 $}
{$e$}
{$+\infty$}

\item 
设周期函数 $f(x)$ 在 $(-\infty,+\infty)$ 内可导, 周期为 4, 又 $\lim\limits _{x \rightarrow 0} \frac{f(1)-f(1-x)}{2 x}=-1$, 则曲线 $y=f(x)$ 在点 $(5, f(5))$ 处的切线斜率为 \xzanswer{D} 


\fourchoices
{$\frac{1}{2}$}
{$ 2 $}
{$-1$}
{$-2$}

\item 
设向量 $\vec{a}=(1,2,2),  \vec{b}=(0,1,2)$, 则向量 $\vec{b}$ 在向量 $\vec{a}$ 方向上的投影向量为 \xzanswer{D} 


\fourchoices
{$\left(0, \frac{6}{5}, \frac{12}{5}\right)$}
{$\left(\frac{1}{3}, \frac{2}{3}, \frac{2}{3}\right)$}
{$\left(0, \frac{1}{\sqrt{5}}, \frac{2}{\sqrt{5}}\right)$}
{$\left(\frac{2}{3}, \frac{4}{3}, \frac{4}{3}\right)$}

\item 
二元函数 $f(x, y)$ 在点 $(0,0)$ 处可微的一个充分条件是 \xzanswer{C} 


\fourchoices
{$\lim\limits _{(x, y) \rightarrow(0,0)}[f(x, y)-f(0,0)]=0$.}
{$\lim\limits _{x \rightarrow 0} \frac{f(x, 0)-f(0,0)}{x}=0$, 且 $\lim\limits _{y \rightarrow 0} \frac{f(0, y)-f(0,0)}{y}=0$.}
{$\lim\limits _{(x, y) \rightarrow(0,0)} \frac{f(x, y)-f(0,0)}{\sqrt{x^{2}+y^{2}}}=0$}
{$\lim\limits _{x \rightarrow 0}\left[f_{x}^{\prime}(x, 0)-f_{x}^{\prime}(0,0)\right]=0$, 且 $\lim\limits _{y \rightarrow 0}\left[f_{y}^{\prime}(0, y)-f_{y}^{\prime}(0,0)\right]=0$.}
		
\item 
级数 $\sum\limits_{n=1}^{\infty} \frac{1}{n(n+1)(n+2)}=$  \xzanswer{B} 


\fourchoices
{$ 0 $}
{$\frac{1}{4}$}
{$\frac{1}{3}$}
{$ 1 $}

\item 
设方程 $x+y+z=e^{x y}$, 则 $\frac{\partial^{2} z}{\partial x^{2}}+\frac{\partial^{2} z}{\partial y^{2}}$ 的表达式是 \xzanswer{A} 


\fourchoices
{$\left(x^{2}+y^{2}\right) e^{x y}$}
{$(x+y) e^{x y}$}
{$2 x y e^{x y}$}
{$(1+x y) e^{x y}$}

\item 
设 $a_{0}=3, a_{1}=5$, 且对任何自然数 $n>1$ 有 $n a_{n}=\frac{2}{3} a_{n-1}-(n-1) a_{n-1}$, 则幂级数 $\sum\limits_{n=0}^{\infty} a_{n} x^{n}$ 的收敛半径为 \xzanswer{B} 


\fourchoices
{$\frac{2}{3}$}
{$ 1 $}
{$\frac{3}{2}$}
{$ 2 $}

\item 
下列反常积分发散的是 \xzanswer{C} 


\fourchoices
{$\int_{0}^{+\infty} \frac{x^{6}}{1+e^{x}} d x$}
{$\int_{0}^{+\infty} \frac{1}{\sqrt{x}(1+x)} d x$}
{$\int_{1}^{+\infty} \frac{1}{x^{4} \ln x} d x$}
{$\int_{1}^{+\infty} \frac{2 \sin ^{2} x}{1+x^{2}} d x$}

		
	\end{enumerate}



\item 
(本题满分10分)
已知 $\lim\limits _{x \rightarrow 0} f(x)$ 存在,而 $\lim\limits _{x \rightarrow 0} f^{\prime}(x)$ 不存在,并且 $\lim\limits _{x \rightarrow 0} \frac{\sqrt[3]{1+x f(x)}-1}{\sin x}=3$,求 $\lim\limits _{x \rightarrow 0} f(x)$。

\banswer{
$\lim\limits _{x \rightarrow 0} f(x)=9$	
}


\item 
(本题满分10分)
两平面均通过点 $A(-2,1,-1)$, 其中一个平面通过 $x$ 轴, 另一个平面通过直线 $\frac{x-1}{2}=\frac{y+1}{1}=\frac{z}{-1}$ ,求两平面夹角的余弦。

\banswer{
	$\cos \theta=\frac{2 \sqrt{6}}{5}$
}



\item 
(本题满分10分)
设函数 $y=f(x)$ 由 $\left\{\begin{array}{c}x^{x}+t x-t^{2}=0, \\ \arctan (t y)=\ln \left(1+t^{2} y^{2}\right)\end{array}\right.$确定,求 $\frac{d y}{d x}$ 。

\banswer{
	$ \frac{dy}{dx}=\frac{t+x^{x}(1+\ln x)}{2 t^{2}(x-2 t)} $
}



\item 
(本题满分10分)
已 知 函 数 $u=f(r), r=\ln \sqrt{x^{2}+y^{2}+z^{2}}$ 满 足 方 程$\frac{\partial^{2} u}{\partial x^{2}}+\frac{\partial^{2} u}{\partial y^{2}}+\frac{\partial^{2} u}{\partial z^{2}}=\left(x^{2}+y^{2}+z^{2}\right)^{-3 / 2}$, 求 $f(x)$ 的表达式。


\banswer{
待定?$f(x)=1+C_{2} e^{x}+\frac{1}{2} \frac{1}{x^{2}+y^{2}+z^{2}}-\frac{1}{\sqrt{x^{2}+y^{2}+z^{2}}}$	
}


\item 
(本题满分10分)
设曲线 $C: y=x^{3}+2 x$ 与其在(1,3)点处的切线以及 $x$ 轴围成的区 域落在第一象限中的部分为 $D$, 计算:
\begin{enumerate}
	%\renewcommand{\labelenumi}{\arabic{enumi}.}
	% A(\Alph) a(\alph) I(\Roman) i(\roman) 1(\arabic)
	%设定全局标号series=example	%引用全局变量resume=example
	%[topsep=-0.3em,parsep=-0.3em,itemsep=-0.3em,partopsep=-0.3em]
	%可使用leftmargin调整列表环境左边的空白长度 [leftmargin=0em]
	\item
$  D  $的面积。
\item 
$D$ 绕 $x$ 轴旋转一周所得旋转体的体积。
\end{enumerate}

\banswer{
	\begin{enumerate}
		%\renewcommand{\labelenumi}{\arabic{enumi}.}
		% A(\Alph) a(\alph) I(\Roman) i(\roman) 1(\arabic)
		%设定全局标号series=example	%引用全局变量resume=example
		%[topsep=-0.3em,parsep=-0.3em,itemsep=-0.3em,partopsep=-0.3em]
		%可使用leftmargin调整列表环境左边的空白长度 [leftmargin=0em]
		\item
		$ S_{D}=\frac{3}{20} $
		\item 
		$ V_{D} =0.11\pi $ (保留二位小数)
		
	\end{enumerate}
}



\newpage
\item 
(本题满分10分)
计算下列第二型曲面积分:
\[
I=\iint_{S} x d y d z+2 y^{4} d x d z+3 z^{6} d x d y
\]
其中 $S$ 是椭球面: $x^{2}+4 y^{2}+9 z^{2}=1$ 。

\banswer{
	$ I=\frac{2}{9} \pi $
}




\item 
(本题满分10分)
设 $f(x)$ 是周期为$  3  $的连续函数, 证明:在任意长度为$  2  $的闭区间
$[a, a+2]$ 上至少存在一点 $\theta$, 使得 $f(\theta)=f(\theta+1)$ 。


\banswer{
	证明略
}


\item 
(本题满分 10 分)
设 $f(x), g(x)$ 在 $[a, b]$ 上二阶可导, 且 $f(a)=f(b)=g(a)=0$. 证明: 存在 $\xi \in(a, b)$, 使得 $f^{\prime \prime}(\xi) g(\xi)+2 f^{\prime}(\xi) g^{\prime}(\xi)+f(\xi) g^{\prime \prime}(\xi)=0$ 。


\banswer{
	证明略
}


\item 
(本题满分10分)
设 $f$ 是 $[0,1]$ 上的连续函数,满足 $\int_{0}^{x} f(t) d t \geq 0$ 对所有的 $x \in[0,1]$ 成立且 $\int_{0}^{1} f(t) d t=0$ 。证明: $\int_{0}^{1} x f(x) d x \leq 0$ 。


\banswer{
	证明略
}



\item 
(本题满分10分)
求证:若正数 $x, y, z$ 满足 $x^{2}+y^{2}+z^{2}=a$, 其中 $a>0$, 则有不等
式 $x^{3}+y^{3}+z^{3} \geq \frac{a \sqrt{3 a}}{3}$ 恒成立。

\banswer{
	证明略
}

	
\end{enumerate}


