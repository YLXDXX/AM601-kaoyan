\bta{2011}



\begin{enumerate}
	%\renewcommand{\labelenumi}{\arabic{enumi}.}
	% A(\Alph) a(\alph) I(\Roman) i(\roman) 1(\arabic)
	%设定全局标号series=example	%引用全局变量resume=example
	%[topsep=-0.3em,parsep=-0.3em,itemsep=-0.3em,partopsep=-0.3em]
	%可使用leftmargin调整列表环境左边的空白长度 [leftmargin=0em]
	\item
	选择题 (本题满分 40 分,每小题 5 分)
	\begin{enumerate}
		%\renewcommand{\labelenumi}{\arabic{enumi}.}
		% A(\Alph) a(\alph) I(\Roman) i(\roman) 1(\arabic)
		%设定全局标号series=example	%引用全局变量resume=example
		%[topsep=-0.3em,parsep=-0.3em,itemsep=-0.3em,partopsep=-0.3em]
		%可使用leftmargin调整列表环境左边的空白长度 [leftmargin=0em]
		\item
		极限 $\lim\limits _{x \rightarrow + \infty}x\left[\left(1+\frac{1}{x}\right)^{x}-e\right]=$ \xzanswer{D} 
		
		
	\fourchoices
	{$ 0 $}
	{$\infty$}
	{$\frac{e}{2}$}
	{$-\frac{e}{2}$}

%	$ (1+\frac{1}{x})^{x}-e=e(e^{x\ln(1+\frac{1}{x} ) -1} -1) $,此时再用等价替换就好


	
	\item 
	若函数 $\left\{\begin{array}{ll}x^{2}+2 x \quad(x \geqslant 0) \\ \ln (1+a x) & (x<0)\end{array}\right.$ 在 $x=0$ 处可导, 则 $a$ 等于	 \xzanswer{B} 
	
	
\fourchoices
{$-2$}
{$ 2 $}
{$-1$}
{$ 1 $}

\item 
设极限 $\lim\limits _{x \rightarrow a} \frac{f(x)-f(a)}{(x-a)^{4}}=-2$, 则函数 $f(x)$ 在 $x=a$ 处 \xzanswer{C} 


\fourchoices
{导数值不等于$ 0 $}
{导数值等于$ 0 $ 但 $x=a$ 不是极值点}
{取得极大值}
{取得极小值}

\item 
设函数 $f(x)$ 在定义域内可导, $f(x)$ 的图像如下图所示, 则 $f^{\prime}(x)$ 的图像最有可能是 \xzanswer{D} 
\begin{figure}[h!]
	\centering
	\includesvg[width=0.23\linewidth]{picture/svg/601-001}
\end{figure}
\pfourchoices
{\includesvg[width=3.3cm]{picture/svg/601-002}}
{\includesvg[width=3.3cm]{picture/svg/601-003}}
{\includesvg[width=3.3cm]{picture/svg/601-004}}
{\includesvg[width=3.3cm]{picture/svg/601-005}}

\item 
设 $x^{2} \ln x$ 是函数 $f(x)$ 的一个原函数, 则不定积分 $\int x f^{\prime}(x) d x=$ \xzanswer{C} 


\fourchoices
{$\frac{2}{3} x^{3} \ln x+\frac{1}{9} x^{3}+C$}
{$2 x-x^{2} \ln x+C$}
{$x^{2} \ln x+x^{2}+C$}
{$3 x^{2} \ln x+x^{2}+C$}

\item 
设 $y_{1}, y_{2}$ 是二阶线性函数齐次微分方程 $y^{\prime \prime}+p(x) y^{\prime}+q(x) y=0$ 的两个特解, $C_{1}, C_{2}$ 是两个任意常数, 则 \xzanswer{B} 


\fourchoices
{$C_{1} y_{1}+C_{2} y_{2}$ 不一定是该微分方程的解}
{$C_{1} y_{1}+C_{2} y_{2}$ 是该微分方程的解, 但不一定是通解}
{$C_{1} y_{1}+C_{2} y_{2}$ 是该微分方程的解,但不是通解}
{$C_{1} y_{1}+C_{2} y_{2}$ 不是该微分方程的解}

\item 
若级数 $\sum\limits_{n=1}^{\infty} u_{n}$ 和 $\sum\limits_{n=1}^{\infty} v_{n}$ 都发散,则下列级数一定发散的是 \xzanswer{C} 


\fourchoices
{$\sum\limits_{n=1}^{\infty}\left(u_{n}+v_{n}\right)$}
{$\sum\limits_{n=1}^{\infty} u_{n} v_{n}$}
{$\sum\limits_{n=1}^{\infty}\left(\left|u_{n}\right|+\left|v_{n}\right|\right)$}
{$\sum\limits_{n=1}^{\infty}\left(u_{n}^{2}+v_{n}^{2}\right)$}

\item 
设两条抛物线 $y=n x^{2}+\frac{1}{n}$ 和 $y=(n+1) x^{2}+\frac{1}{n+1}$ 所围成的面积为 $A_{n}$, 则 $\lim\limits _{n \rightarrow \infty} A_{n}=$ \xzanswer{A}



\fourchoices
{$ 0 $}
{$ 1 $}
{$ 2 $}
{$ 3 $}


		
	\end{enumerate}
	
\item 
(本题满分 10 分)	
设函数 $f(x)$ 在 $[0,2 \pi]$ 上有连续导数, 且 $f^{\prime}(x) \geqslant 0$, 求证:对任何整数 $n$, 都有
\[
\left|\int_{0}^{2 \pi} f(x) \sin (nx) d x\right| \leqslant \frac{2}{n} \left[ f(2 \pi)-f(0)\right]
\]


\banswer{
	证明略,分部积分即可
}



\item 
(本题满分 10 分)
设 $f(x)$ 具有连续导数, 且满足 $f(x)=x+\int_{0}^{x} t f^{\prime}(x-t) d t$, 求极限 $\lim\limits _{x \rightarrow-\infty} f(x)$。


\banswer{
$ f(x)=e^{x}-1 $, $\lim\limits _{x \rightarrow -\infty} f(x)=-1$
}


\item 
(本题满分 10 分)
设函数 $z=f(x, y)$ 满足 $\frac{\partial^{2} z}{\partial x \partial y}=x+y, f(x, 0)=x, f(0, y)=y^{2}$, 求 $f(x, y)$。

\banswer{
	$f(x, y)=\frac{1}{2}x^{2}y + \frac{1}{2} x y^{2}+x+y^{2}$
}



\item 
(本题满分 10 分)
求函数 $f(x, y)=(x-6)^{2}+(y+8)^{2}$ 在区域 $D=\left\{(x, y) \mid x^{2}+y^{2} \leqslant 25\right\}$ 上的最大值 和最小值。

\banswer{
	 $f(x, y) \Big|_{max}= 225 $, $f(x, y) \Big|_{min}=  25 $
}


\newpage

\item 
(本题满分 10 分)
计算二重积分 $\iint_{D}(|x|+|y|) d x d y$, 其中积分区域 $D$ 由直线 $x=0, x+y=3, y=x-1$ 及 $y=x+1$ 所围成的区域。



\banswer{
	%$3$
	$ I=\frac{14}{3} $ 
}


\item 
(本题满分 10 分)
计算曲线积分 $I=\oint_{L} \frac{x d y-y d x}{4 x^{2}+y^{2}}$, 其中 $L$ 是以 $(1,0)$ 为中心, 半径为 $R$ 的圆周 $(R>0, R \neq 1)$, 取逆时针方向。

\banswer{
	$0<R<1,I=0$;$ R>1,I=\pi$
}



\item 
(本题满分 10 分)
设函数 $f(x)$ 在 $(-\infty,+\infty)$ 上有定义,在 $x=0$ 的某个领域内有一阶连续的导数, 且 $\lim\limits _{x \rightarrow 0} \frac{f(x)}{x}=a>0$, 证明:
$\sum\limits_{n=1}^{\infty}(-1)^{n} f\left(\frac{1}{n}\right)$ 收敛,而 $\sum\limits_{n=1}^{\infty} f\left(\frac{1}{n}\right)$ 发散。



\banswer{
	证明略,$ \lim\limits_{n\rightarrow \infty}f(\frac{1}{n}) \sim \frac{a}{n} $
}


\item 
(本题满分 10 分)
将函数 $f(x)=x^{2}(-\pi \leqslant x \leqslant \pi)$ 展开成以 $2 \pi$ 为周期的傅里叶级数。


\banswer{
$ 	
f(x)=\frac{\pi^{2}}{3}+\sum\limits_{n=1}^{\infty}(-1)^{n} \frac{4}{n^{2}} \cos n x 
$
}


\item 
(本题满分 10 分)
已知函数 $f(x)$ 具有连续的二阶导数, $f(0)=0, f^{\prime}(0)=\frac{1}{3}$, 且对任意光滑有向封闭 曲面 $\Sigma$都有 $\oint_{\Sigma} e^{x}\left(f^{\prime}(x) d y d z-2 y f(x) d z d x-z d x d y\right)=0$, 求函数 $f(x)$ 的表达式。


\banswer{
	$f(x)=C_{1}e^{-2x}+C_{2}e^{x}- \frac{ 1 }{ 2 } $,再利用初始条件得$f(x)=
	\frac{1}{18} e^{-2 x}+\frac{4}{9} e^{x}-\frac{1}{2} $
}



\item 
(本题满分 10 分)
已知平面过点 $ (1,2,3) $,它在 $ x $ 正半轴和 $ y $ 轴正半轴上的截距相等,问当该平面在
三个坐标轴上的截距分别为何值时,它与三个坐标轴所围成的立体体积取极值,并写出此时的平面方程。


%群友反馈:原题目有问题,最小值为零,两种改法:①将$ z $轴的截距限制为正;②将最值改为极值。


\banswer{
	$ x $轴、$ y $轴、$ z $轴截距分别为$  \frac{ 2 }{ 9 }  $、$  \frac{ 2 }{ 9 }  $、$ 9 $时取极值,平面方程为:$ \frac{x}{\frac{9}{2}}+\frac{y}{\frac{9}{2}}+\frac{z}{9}=1 $。
}


\item 
(本题满分 10 分)
设函数 $f(x)$ 在 $[0, c]$ 上可导, $f^{\prime}$ 单调递减且 $f(0)=0$, 求证: 对于 $0 \leqslant a \leqslant b \leqslant a+b \leqslant c$ ,都有 $f(a+b) \leqslant f(a)+f(b)$。

\banswer{
	证明略,利用三次拉格朗日中值定理即可
}




\end{enumerate}


