\bta{2006}

\begin{enumerate}
	%\renewcommand{\labelenumi}{\arabic{enumi}.}
	% A(\Alph) a(\alph) I(\Roman) i(\roman) 1(\arabic)
	%设定全局标号series=example	%引用全局变量resume=example
	%[topsep=-0.3em,parsep=-0.3em,itemsep=-0.3em,partopsep=-0.3em]
	%可使用leftmargin调整列表环境左边的空白长度 [leftmargin=0em]
	\item
	(本题满分 10 分)
求 $\lim\limits _{x \rightarrow \infty}\left(\frac{x+1}{x-1}\right)^{x}$。

\banswer{
	$e^{2}$
}
	

\item
(本题满分 10 分)	
求 $a, b$ 的值, 使得 $f(x)=\left\{\begin{array}{ll}\sin a(x-1) & x \leqslant 1 \\ \ln x+b & x>1\end{array}\right.$在$  x = 1  $处可导。
	

\banswer{
	$ a=1,b=0 $
}


\item
(本题满分 10 分)	
求 $y=(x-1)\left(\frac{(1-2 x) \ln x}{1+x^{2}}\right)^{\frac{1}{3}}$ 的导数。

\banswer{
	$y^{\prime}=\left[\frac{1}{x-1}+\frac{1}{3}\left(\frac{2}{2 x-1}+\frac{1}{x \ln x}-\frac{2 x}{1+x^{2}}\right)\right] y$
}

	

\item
(本题满分 10 分)	
设 $z=f(u)$, 方程 $u=q(u)+\int_{y}^{x} p(t) d t$ 确定 $u$ 是 $x, y$ 的函数, 其中 $f(u), q(u)$ 可 微, $p(t)$ 连续, 且 $q(u) \neq 1$, 求 $p(y) \frac{\partial z}{\partial x}+p(x) \frac{\partial z}{\partial y}$。

\banswer{
	$p(y) \frac{\partial z}{\partial x}+p(x) \frac{\partial z}{\partial y}= 0 $
}
	

\item
(本题满分 10 分)	
设函数 
\[ 
	F(x)=\left\{\begin{aligned}
	&\frac{\int_{0}^{x} t f(t) d t}{x^{2}} & x \neq 0 \\
	&A & x=0
\end{aligned}\right.
 \]
 连续, 且 $f(0)=0$ 。
\begin{enumerate}
	%\renewcommand{\labelenumi}{\arabic{enumi}.}
	% A(\Alph) a(\alph) I(\Roman) i(\roman) 1(\arabic)
	%设定全局标号series=example	%引用全局变量resume=example
	%[topsep=-0.3em,parsep=-0.3em,itemsep=-0.3em,partopsep=-0.3em]
	%可使用leftmargin调整列表环境左边的空白长度 [leftmargin=0em]
	\item
	求 $A$ 的值, 使 $F(x)$ 在 $x=0$ 处连续。
	\item 
	研究 $F^{\prime}(x)$ 在 $x=0$ 处的连续性。
	
	
	
	
\end{enumerate}

\banswer{
	\begin{enumerate}
		%\renewcommand{\labelenumi}{\arabic{enumi}.}
		% A(\Alph) a(\alph) I(\Roman) i(\roman) 1(\arabic)
		%设定全局标号series=example	%引用全局变量resume=example
		%[topsep=-0.3em,parsep=-0.3em,itemsep=-0.3em,partopsep=-0.3em]
		%可使用leftmargin调整列表环境左边的空白长度 [leftmargin=0em]
		\item
		$ A=0 $
		\item 
		$F^{\prime}(x)$ 在 $x=0$ 处连续
		
	\end{enumerate}
	
	
}

	
\newpage
\item
(本题满分 10 分)	
设 $f(x)$ 满足方程 $\int \sqrt{x} f(x) d x=\frac{1}{\sqrt{x}}+\int x^{\frac{3}{2}} \sin x d x+C$, 求 $\int f(x) d x$。

\banswer{
	$\frac{1}{2} x^{-1}-x \cos x+\sin x+\mathrm{C}$
}




\item
(本题满分 10 分)	
当 $x>0$ 时, $f(\ln x)=\frac{1}{\sqrt{x}}$ , 求 $\int_{-2}^{2} x f^{\prime}(x) d x$。
	

\banswer{
	$\frac{4}{e}$
}



\item
(本题满分 10 分)	
求经过原点且垂直于平面 $\pi_{1}: x+2 y+3 z-2=0$ 及 $\pi_{2}: 6 x-y-5 z+23=0$ 的平面方程。

\banswer{
	$-7 x+23 y-13 z=0$
}

	
	

\item
(本题满分 10 分)	
将函数 $f(x)=\ln \left(\frac{x}{1-x}\right)$ 展开成 $x-1$ 的幂级数。


\banswer{
	$f(x)=\sum\limits_{n=1}^{\infty}(-1)^{n-1} \frac{(x-1)^{n}}{n}-\sum\limits_{n=1}^{\infty}(-1)^{n-1} \frac{(x-1)^{n}}{n \cdot 2^{n}} x^{n-1}-\ln 2$
}



\item
(本题满分 10 分)	
求级数 $\sum\limits_{n=1}^{\infty} \frac{1}{n 2^{n}} x^{n-1}$ 的核函数。

\banswer{
收敛区间为 $[-2,2]$, 令 $S_{n}=\sum\limits_{n=1}^{\infty} \frac{1}{n 2^{n}} x^{n-1}$,$T_{n}=\sum\limits_{n=1}^{\infty} \frac{1}{n 2^{n}} x^{n}$。$S_{n} \frac{1}{x} T_{n}=[\ln 2-\ln (2 - x)]$
}
	

\newpage
\item
(本题满分 10 分)	
计算 $I=\int_{L} \frac{(x-y) d x+(x+y) d y}{x^{2}+y^{2}}$, 其中 $L$ 是抛物线 $y=2 x^{2}-1$ 从点 $A(-1,1)$ 到 点 $B(1,1)$ 的一段。

\banswer{
	$I=\frac{3 \pi}{2}$
}


\item
(本题满分 10 分)	
设曲线积分 $\int_{L}\left(f^{\prime}(x)+2 f(x)+e^{x}\right) y d x+f^{\prime}(x) d y$ 与路径无关,且 $f(0)=0, f^{\prime}(0)=1$, 计算 $I=\int_{(0,0)}^{(1,1)}\left(f^{\prime}(x)+2 f(x)+e^{x}\right) y d x+f^{\prime}(x) d y$。

\banswer{
	$I=\frac{4}{3} e^{2}+\frac{1}{6 e}-\frac{1}{2} e$
}


\item
(本题满分 10 分)	
设 $f(x)$ 在 $[a, b]$ 上连续在 $(a, b)$ 内可导, 试证存在 $\xi \in(a, b)$ 使得 $2 \xi[f(b)-f(a)]=\left(b^{2}-a^{2}\right) f^{\prime}(\xi)$。

\banswer{
	证明略
}

	

\item
(本题满分 10 分)	
设 $f(x)$ 为可导且周期为 2 的函数,满足 $f(1+x)+2 f(1-x)=2 x+\sin ^{2} x$, 求曲 线 $y=f(x)$ 在 $x=3$ 处的切线斜率。
\banswer{
	$ f(x) $在$ x=3 $处的切线方程为$y=-2 x$
}
	

\item
(本题满分 10 分)	
如图, 某公园有一座高为 $a$ 的塑像, 其基座高为 $b$ 米。今有一观赏者高为 $h$ 米 $h<b$, 问他离基底多远时, 其视线对塑像张成的角最大?
\begin{figure}[h!]
	\flushright
	\includesvg[width=0.25\linewidth]{picture/svg/601-007}
\end{figure}

\banswer{
张角$\tan \alpha=\frac{a x}{(a+b-h)(b-h)+x^{2}}$,当$x=\sqrt{(a+b-h)(b-h)} $时,$\tan  \alpha $ 取极大值
}
	





\end{enumerate}


