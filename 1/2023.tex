\bta{2023}

\begin{enumerate}
	%\renewcommand{\labelenumi}{\arabic{enumi}.}
	% A(\Alph) a(\alph) I(\Roman) i(\roman) 1(\arabic)
	%设定全局标号series=example	%引用全局变量resume=example
	%[topsep=-0.3em,parsep=-0.3em,itemsep=-0.3em,partopsep=-0.3em]
	%可使用leftmargin调整列表环境左边的空白长度 [leftmargin=0em]
	\item
\begin{enumerate}
%\renewcommand{\labelenumi}{\arabic{enumi}.}
% A(\Alph) a(\alph) I(\Roman) i(\roman) 1(\arabic)
%设定全局标号series=example	%引用全局变量resume=example
%[topsep=-0.3em,parsep=-0.3em,itemsep=-0.3em,partopsep=-0.3em]
%可使用leftmargin调整列表环境左边的空白长度 [leftmargin=0em]
\item
已知 $f\left(\ln \left(\sqrt{x^2+1}+x\right)\right)=\sqrt{x^2+1}-x$, 其反函数 $f^{-1}(x)$ 是 \xzanswer{A} 

		
\fourchoices
{$\ln \frac{1}{x} $}
{$\ln \frac{1}{\sqrt{x}} $}
{$e^{\frac{1}{x}} $}
{$ e^{\frac{1}{\sqrt{x}}}$}

		
\item 
极限 $\lim\limits _{x \rightarrow 0}\left(\frac{2^x+3^x}{2}\right)^{2 / x}$的结果为 \xzanswer{B} 

		
\fourchoices
{$ e $}
{$ 6 $}
{$ \sqrt{6} $}
{$ \sqrt{e} $}



\item 
已知对数螺旋线$\rho=e^\theta$,在$\theta=\frac{\pi}{2} $,$ (\rho, \theta)=\left(e^{\frac{\pi}{2}}, \frac{\pi}{2}\right)$处切线的
直角坐标方程是 \xzanswer{C} 


\fourchoices
{$x-y=0 $}
{$ x-y=1 $}
{$x+y=e^{\frac{\pi}{2}}$}
{$x+y=e^{2 \pi}$}



\item 
已知方程$ xe^{x}=1 $,求实数根的个数 \xzanswer{B} 

\fourchoices
{$ 0 $}
{$ 1 $}
{$ 2 $}
{$ 4 $}


\item 
已知$ \bm{a} =(0,1,2) $,而$  \bm{b}$平行于$\bm{a} $,且$  \bm{a} \cdot \bm{b}=2$,则$  \bm{b}$为 \xzanswer{B} 


\fourchoices
{$ \cdot\left(0, \frac{1}{3}, \frac{2}{3}\right) $}
{$ \left(0, \frac{2}{5}, \frac{4}{5}\right) $}
{$ \left(0, \frac{2}{7}, \frac{4}{7}\right) $}
{$ \left(0, \frac{2}{9}, \frac{4}{9}\right)$}


\item 
已知$f(x, y)=\sqrt{|x y|}$,则$ f(x,y) $在$ (0,0) $处 \xzanswer{A} 


\fourchoices
{可导但不可微}
{不可导}
{可导也可微}
{不连续也不可导}


\item 
已知$y=e^{\frac{1}{x^2}} \arctan \frac{x^2+x+1}{(x+1)(x+2)}$,其渐近线的条数为 \xzanswer{B} 

\fourchoices
{$ 1 $}
{$ 2 $}
{$ 3 $}
{$ 4 $}

\item 
已知$z=z(x, y)$是由$F\left(x+\frac{z}{y}, y+\frac{z}{x}\right)=0$确定的二元函数,
$ F $和$ z $均为可微函数,求$x \frac{\partial z}{\partial x}+y \frac{\partial z}{\partial y}$的值 \xzanswer{B} 


\fourchoices
{$z+x y $}
{$ z-x y $}
{$ 0$}
{$ 1 $}


\item 
已知$\sum\limits_{n=0}^{\infty} \frac{n^n x^n}{(n !)^2}$,收敛半径$ r $为 \xzanswer{D} 

\fourchoices
{$ 1 $}
{$ 0 $}
{$ e $}
{$ \infty $}

\item 
已知$y=c_1 e^x+c_2 e^{-2 x}+x e^x$,其满足的一个微分方程是 \xzanswer{C} 

\fourchoices
{$y^{\prime \prime}-y^{\prime}-2 y=3 x e^x$}
{$y^{\prime \prime}-y^{\prime}-2 y=3 e^x$}
{$y^{\prime \prime}+y^{\prime}-2 y=3 e^x$}
{$y^{\prime \prime}+y^{\prime}-2 y=3 x e^x$}




		
\end{enumerate}



	
\item 
函数
$ 
	f(x)=\left\{\begin{array}{cc}
	\frac{\ln \left(1+x^2\right)}{x}+a, & x<0 \\
	1 / 2, &  x=0 \\
	b \cdot \frac{\sin ^3 x}{\ln \left(1+x^3\right)}, &  x>0
\end{array}\right.
 $
在$ x=0 $处连续,求$ a $,$ b $的值。

\banswer{
	$ a=b= \frac{1}{2}  $
}


\newpage
\item 
已知微分方程$y^{\prime \prime}+x y^{\prime}-2 y=0$。
\begin{enumerate}
	%\renewcommand{\labelenumi}{\arabic{enumi}.}
	% A(\Alph) a(\alph) I(\Roman) i(\roman) 1(\arabic)
	%设定全局标号series=example	%引用全局变量resume=example
	%[topsep=-0.3em,parsep=-0.3em,itemsep=-0.3em,partopsep=-0.3em]
	%可使用leftmargin调整列表环境左边的空白长度 [leftmargin=0em]
	\item
	利用级数$\sum\limits_{n=0}^{\infty} a_n x^n$的形式,求一个特解;
	\item 
	求其通解。
	
	
	
\end{enumerate}


\banswer{
\begin{enumerate}
	%\renewcommand{\labelenumi}{\arabic{enumi}.}
	% A(\Alph) a(\alph) I(\Roman) i(\roman) 1(\arabic)
	%设定全局标号series=example	%引用全局变量resume=example
	%[topsep=-0.3em,parsep=-0.3em,itemsep=-0.3em,partopsep=-0.3em]
	%可使用leftmargin调整列表环境左边的空白长度 [leftmargin=0em]
	\item
	级数系数关系为$a_0=a_2$,$a_{n+2}=-\frac{n-2}{(n+1)(n+2)} a_n$,一个特解为$ y=0 $。
	
	\item 
	通解为$y=c_1\left(x^2+1\right)+c_2 \sum\limits_{n=0}^{\infty} \frac{(-1)^n}{\left(4 n^2-1\right) 2^n n !} x^{2 n+1}$。
	
\end{enumerate}


}


\item 
已知$\sqrt[x]{y}=\sqrt[y]{x} $,$  x>0 $,$  y>0 $,$  y=f(x)$,求$\frac{d^2 y}{d x^2}$。

\banswer{
	一阶导为$y^{\prime}=\frac{1+\ln x}{1+\ln y}$,二阶导为
	\begin{equation*}
		y^{\prime \prime}=\frac{\left(\frac{1}{x}-\frac{\left(y^{\prime}\right)^2}{y}\right)}{1+\ln y}=\frac{1}{x(1+\ln y)}-\frac{(\ln x+1)^2}{y(1+\ln y)^3}
	\end{equation*}
}


\item 
对二元函数$ u $和一元函数$ f $,
求证$ u $满足$y \frac{\partial u}{\partial x}+x \frac{\partial u}{\partial y}=0 $的充要条件是$  u(x, y)=f\left(x^2-y^2\right)  $。

\banswer{
	易证略。
}


\item 
区域$D: x^2+y^2 \leqslant 2 $,$  x \geqslant 0 $,$  y \geqslant 0 $上,求二重积分
\[
\iint_D x^{\left[1+x^2+y^2\right]} \cdot y\left[1+x^2+y^2\right] d x d y
\]
其中$ \left[x\right]$为向下取整函数,表示取不大于$ x $的最大整数。

\banswer{
	$I=\frac{1}{8}+\frac{2}{15}(4 \sqrt{2}-1)$
}



\item 
\begin{enumerate}
	%\renewcommand{\labelenumi}{\arabic{enumi}.}
	% A(\Alph) a(\alph) I(\Roman) i(\roman) 1(\arabic)
	%设定全局标号series=example	%引用全局变量resume=example
	%[topsep=-0.3em,parsep=-0.3em,itemsep=-0.3em,partopsep=-0.3em]
	%可使用leftmargin调整列表环境左边的空白长度 [leftmargin=0em]
	\item
求$y=\ln x$上过原点的切线和$y=\ln x$和$ x $轴围成的面积;

\item 
求(1)中的封闭曲线绕$ x $轴旋转一周形成的体积。

	
	
	
\end{enumerate}

\banswer{
\begin{enumerate}
	%\renewcommand{\labelenumi}{\arabic{enumi}.}
	% A(\Alph) a(\alph) I(\Roman) i(\roman) 1(\arabic)
	%设定全局标号series=example	%引用全局变量resume=example
	%[topsep=-0.3em,parsep=-0.3em,itemsep=-0.3em,partopsep=-0.3em]
	%可使用leftmargin调整列表环境左边的空白长度 [leftmargin=0em]
	\item
	切线为$y=\frac{x}{e}$,面积$S=\frac{e}{2}-1$。
	
	\item 
	$V=\pi\left(2-\frac{2 e}{3}\right)$
\end{enumerate}

	
}



\newpage
\item 
计算曲面积分
\[
I=\iint_S x^2 y d y d z+x y^2 d z d x+\left(x^2+2 y^2\right) d x d y
\]
其中$ S $为$x^2+y^2+z^2=4$的上则。



\banswer{
	$I=3 \pi$
}

\item 
函数$f(x)$在$[0,1]$上二阶可导,$f(0)=f(1)=0 $,$  f(x)$的最小值为$ -1 $,
求证$\exists \xi \in(0,1) $,使得$  f^{\prime \prime}(\xi) \geqslant 8$。

\banswer{
易证略。	
}


\item 
直线$l: \left\{\begin{array}{l}x^2+y+2 z=1 \\ 2 x+y+3 z=4\end{array}\right.$,
线上有一点$P(x, y, z)$,求当$ P $点到原点的距离最小时$ P $点的坐标。

\banswer{
	距离取最小时$x=\frac{8}{3}, y=-\frac{7}{3}, z=\frac{1}{3}$,$|O P| _{\min} =\sqrt{x^2+y^2+z^2}=\frac{\sqrt{114}}{3}$。
}







\item 
函数$f(x)$在闭区间$[2,4]$上连续,在开区间$(2,4)$上可导且导数取值大于$ 0 $,假设极限$\lim\limits _{x \rightarrow 2} \frac{f(2 x-2)}{x-2}$存在,试证明:
\begin{enumerate}
	%\renewcommand{\labelenumi}{\arabic{enumi}.}
	% A(\Alph) a(\alph) I(\Roman) i(\roman) 1(\arabic)
	%设定全局标号series=example	%引用全局变量resume=example
	%[topsep=-0.3em,parsep=-0.3em,itemsep=-0.3em,partopsep=-0.3em]
	%可使用leftmargin调整列表环境左边的空白长度 [leftmargin=0em]
	\item
	$f(x)$在$(2,4)$上取值大于$ 0 $;
	
	\item 
	$\exists \xi \in(2,4) $,使得$  \frac{6}{\int_2^4 f(x) d x}=\frac{\xi}{f(\xi)}$;
	
	\item 
	对上述中的$\xi \in(2,4) $,$  \exists \eta \neq \xi $,$  \eta \in(2,4)$,使得
	\[
	6 f^{\prime}(\eta)=\frac{\xi}{\xi-2} \int_2^4 f(x) d x
	\]
	
\end{enumerate}


\banswer{
	易证略。
}



\end{enumerate}

















