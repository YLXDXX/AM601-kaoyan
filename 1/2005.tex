\bta{2005}


\begin{enumerate}
	%\renewcommand{\labelenumi}{\arabic{enumi}.}
	% A(\Alph) a(\alph) I(\Roman) i(\roman) 1(\arabic)
	%设定全局标号series=example	%引用全局变量resume=example
	%[topsep=-0.3em,parsep=-0.3em,itemsep=-0.3em,partopsep=-0.3em]
	%可使用leftmargin调整列表环境左边的空白长度 [leftmargin=0em]
	\item
	填空题 (本题共 5 小题,每小题 5 分,满分 25 分)
	\begin{enumerate}
		%\renewcommand{\labelenumi}{\arabic{enumi}.}
		% A(\Alph) a(\alph) I(\Roman) i(\roman) 1(\arabic)
		%设定全局标号series=example	%引用全局变量resume=example
		%[topsep=-0.3em,parsep=-0.3em,itemsep=-0.3em,partopsep=-0.3em]
		%可使用leftmargin调整列表环境左边的空白长度 [leftmargin=0em]
		\item
		已知 $f^{\prime}\left(x_{0}\right)=3$, 则 $\lim\limits _{x \rightarrow 0} \frac{f\left(x_{0}\right)-f\left(x_{0}-2 x\right)}{x}=$ \tk{$ 6 $} 。
		
		
		\item
		 设 $f(x)$ 的一个原函数为 $e^{x^{2}}$, 则 $\int x f^{\prime}(x) d x=$ \tk{$ (2x^{2}-1)e^{x^{2}} +C $} 。
		
		
		
		
		\item
		数量场 $u=x^{2}+2 y^{2}+3 z^{2}$ 在点 $(1,1,-1)$ 的最大方向微商值为 \tk{$ 2\sqrt{14} $} 。
		
		
		\item
		级数 $\sum\limits_{n=1}^{\infty} \frac{x^{2 n}}{2^{n}+3^{n}}$ 的收敛半径为 \tk{$\sqrt{3}$} 。
		
		
		\item
		微分方程 $y^{\prime}+\frac{1}{x}=1$ 的通解为 \tk{$ y=x(c+ \ln |x|) $} 。
		
		
	\end{enumerate}
	
\item 
单项选择题 (本题共 5 小题,每小题 5 分,满分 25 分)
\begin{enumerate}
	%\renewcommand{\labelenumi}{\arabic{enumi}.}
	% A(\Alph) a(\alph) I(\Roman) i(\roman) 1(\arabic)
	%设定全局标号series=example	%引用全局变量resume=example
	%[topsep=-0.3em,parsep=-0.3em,itemsep=-0.3em,partopsep=-0.3em]
	%可使用leftmargin调整列表环境左边的空白长度 [leftmargin=0em]
	\item
	设 $f(0)=0$, 则 $f(x)$ 在点 $x=0$ 可导的充要条件为 \xzanswer{C} 
	
	
\fourchoices
{$\lim\limits _{t \rightarrow 0} \frac{1}{t^{2}} f\left(t^{2}\right)$ 存在}
{$\lim\limits _{t \rightarrow 0} \frac{1}{t^{2}} f(t-\sin t)$ 存在}
{$\lim\limits _{t \rightarrow 0} \frac{1}{t} f(\ln (1+t))$ 存在}
{$\lim\limits _{t \rightarrow 0} \frac{1}{t^{2}}[f(2 t)-f(t)]$ 存在}

\item 
设曲面 $x^{2}+y^{2}-\frac{z^{2}}{4}=1$ 在点 $(1,1,2)$ 处的法线为 $L$, 又设 $L_{1}: \frac{x-1}{2}=\frac{y-2}{1}=$ $\frac{z-1}{0} ; \quad \pi: x+y+4 z=1$, 则 \xzanswer{A} 


\fourchoices
{$L$ 与 $L_{1}$ 相交, 且 $L$ 平行于 $\pi$}
{$L$ 与 $L_{1}$ 相交, 且 $L$ 垂直于 $\pi$}
{$L$ 与 $L_{1}$ 异面, 且 $L$ 平行于 $\pi$}
{$L$ 与 $L_{1}$ 异面, 且 $L$ 垂直于 $\pi$}

\item 
设 $S$ 是柱面 $x^{2}+y^{2}=R^{2}(0 \leqslant z \leqslant R)$ 的外侧, 则 $\iint_{S}\left(x^{2}+y^{2}\right) d x d y$ 的值为 \xzanswer{D} 


\fourchoices
{$2 \pi R^{3}$}
{$2 \pi R^{4}$}
{$\pi R^{4}$}
{$ 0 $}

\item 
设级数 $\sum\limits_{n=1}^{\infty} a_{n}$ 收敛, 则下列结论中正确的是 \xzanswer{B} 


\fourchoices
{级数 $\sum\limits_{n=1}^{\infty} a_{n}$ 收敛}
{级数 $\sum\limits_{n=1}^{\infty} \sqrt[n]{n} a_{n}$ 收敛}
{级数 $\sum\limits_{n=1}^{\infty} \frac{(-1)^{n}}{\sqrt{n}} a_{n}$ 收敛}
{级数 $\sum\limits_{n=1}^{\infty} \frac{a_{n}}{n}$ 收敛}

\item 
设 $f(x)=x-L(0 \leqslant x \leqslant 2 L)$, 则其以 $2 L$ 为周期的傅里叶级数在点 $x=-\frac{L}{2}$ 处收 敛于 \xzanswer{C} 


\fourchoices
{$-\frac{L}{2}$}
{$-\frac{3 L}{2}$}
{$\frac{L}{2}$}
{$\frac{3 L}{2}$}


	
	
\end{enumerate}

\newpage
\item 
(5 小题,每小题 8 分,满分 40 分)
\begin{enumerate}
	%\renewcommand{\labelenumi}{\arabic{enumi}.}
	% A(\Alph) a(\alph) I(\Roman) i(\roman) 1(\arabic)
	%设定全局标号series=example	%引用全局变量resume=example
	%[topsep=-0.3em,parsep=-0.3em,itemsep=-0.3em,partopsep=-0.3em]
	%可使用leftmargin调整列表环境左边的空白长度 [leftmargin=0em]
	\item
计算极限 $\lim\limits _{x \rightarrow 0} \frac{\sin x-\sin (\sin x)}{x^{3}}$。
	
	
	\item
	计算广义积分 $\int_{0}^{+\infty} \frac{x}{\left(2+x^{2}\right) \sqrt{1+x^{2}}} d x$。
	


\item
 利用欧拉积分计算 $\int_{0}^{1} \frac{x^{6}}{\sqrt[6]{1-x^{6}}} d x$。


\item
$f(u, v)$ 具有二阶连续偏导数, $z=f\left(x y^{2}, \frac{y}{x}\right)$, 求 $\frac{\partial z}{\partial x}, \frac{\partial^{2} z}{\partial x \partial y}$。


\item
计算二重积分 $\int_{0}^{1} d x \int_{\sqrt[3]{x}}^{1} \cos y^{2} d y$。
	
	
	
\end{enumerate}



\banswer{
	\begin{enumerate}
		%\renewcommand{\labelenumi}{\arabic{enumi}.}
		% A(\Alph) a(\alph) I(\Roman) i(\roman) 1(\arabic)
		%设定全局标号series=example	%引用全局变量resume=example
		%[topsep=-0.3em,parsep=-0.3em,itemsep=-0.3em,partopsep=-0.3em]
		%可使用leftmargin调整列表环境左边的空白长度 [leftmargin=0em]
		\item
		$  \frac{ 1 }{ 6 }  $
		\item 
		$ \frac{\pi}{4} $
		\item 
		$ \frac{\pi}{18} $
		\item 
		$\frac{\partial z}{\partial x}=y^{2} \frac{\partial f}{\partial u}-\frac{y}{x^{2}} \frac{\partial f}{\partial v}$,
		$\frac{\partial^{2} z}{\partial x \partial y}=2 y \frac{\partial f}{\partial u}-\frac{1}{x^{2}} \frac{\partial t}{\partial v}+2 x y^{3} \frac{\partial^{2} f}{\partial u^{2}}-\frac{y^{2}}{x} \frac{\partial^{2} f}{\partial u \partial v}-\frac{y}{x^{3}} \frac{\partial^{2} f}{\partial v^{2}}$
		\item 
	$=\frac{1}{2}(\sin 1+\cos 1-1)$
	
	\end{enumerate}
	
	
}


\item 
(3 小题,每小题 12 分,满分 36 分)
\begin{enumerate}
	%\renewcommand{\labelenumi}{\arabic{enumi}.}
	% A(\Alph) a(\alph) I(\Roman) i(\roman) 1(\arabic)
	%设定全局标号series=example	%引用全局变量resume=example
	%[topsep=-0.3em,parsep=-0.3em,itemsep=-0.3em,partopsep=-0.3em]
	%可使用leftmargin调整列表环境左边的空白长度 [leftmargin=0em]
	\item
设 $f(x)$ 具有二阶连续导数, $f(x)=1, f^{\prime}(0)=1$, 且曲线积分 $\int_{L}\left(e^{x} \sin y+2 y f^{\prime}(x)+2 x y\right) d x+\left(f^{\prime}(x)+f(x)+2 x+e^{x} \cos y\right) d y$ 与路径无关。
\begin{enumerate}
	%\renewcommand{\labelenumi}{\arabic{enumi}.}
	% A(\Alph) a(\alph) I(\Roman) i(\roman) 1(\arabic)
	%设定全局标号series=example	%引用全局变量resume=example
	%[topsep=-0.3em,parsep=-0.3em,itemsep=-0.3em,partopsep=-0.3em]
	%可使用leftmargin调整列表环境左边的空白长度 [leftmargin=0em]
	\item
	求 $f(x)$。
	\item 
	当 $L$ 是从 $(0,0)$ 沿曲线 $y=x^{4}$ 到 $(1,1)$ 的有向曲线段时, 求该曲线积分的值。
	
	
	
	
\end{enumerate}

\item 
将函数 $y=\arctan x-\frac{1}{2} \ln \left(1+x^{2}\right)$ 在 $x=0$ 处展成泰勒级数, 并求收敛域及$ \sum\limits_{n=0}^{\infty} \frac{(-1)^{n}}{(2 n+1)(2 n+2)} $。
	
\item 
将函数 $f(x)=\left\{\begin{array}{ll}\pi & -\pi \leqslant x \leqslant 0 \\ \pi-x & 0 \leqslant x \leqslant \pi\end{array}\right.$展开成傅里叶级数 (说明收敛情况),并求$\sum\limits_{n=1}^{\infty} \frac{1}{(2 n-1)^{2}}$。


	
\end{enumerate}


\banswer{
	\begin{enumerate}
		%\renewcommand{\labelenumi}{\arabic{enumi}.}
		% A(\Alph) a(\alph) I(\Roman) i(\roman) 1(\arabic)
		%设定全局标号series=example	%引用全局变量resume=example
		%[topsep=-0.3em,parsep=-0.3em,itemsep=-0.3em,partopsep=-0.3em]
		%可使用leftmargin调整列表环境左边的空白长度 [leftmargin=0em]
		\item
		\begin{enumerate}
			%\renewcommand{\labelenumi}{\arabic{enumi}.}
			% A(\Alph) a(\alph) I(\Roman) i(\roman) 1(\arabic)
			%设定全局标号series=example	%引用全局变量resume=example
			%[topsep=-0.3em,parsep=-0.3em,itemsep=-0.3em,partopsep=-0.3em]
			%可使用leftmargin调整列表环境左边的空白长度 [leftmargin=0em]
			\item
			$ f(x)=e^{x}-x^{2} $
			\item 
			$ e(2+\sin 1)-1 $
			
		\end{enumerate}
		\item 
		$y(x)=\sum\limits_{n=0}^{\infty} \frac{(-1)^{n} x^{2 n+2}}{(2 n+1)(2 n+2)}$,收敛域为$ |x| \leq 1 $,$ \sum\limits_{n=0}^{\infty} \frac{(-1)^{n}}{(2 n+1)(2 n+2)} =\frac{\pi}{4}- \frac{ 1 }{ 2 } \ln 2$
		
		\item 
		$ f(x)=\frac{3}{4} \pi+\sum\limits_{n=1}^{\infty}\left[\frac{1-(-1)^{n}}{n^{2} \pi} \cos n x+\frac{(-1)^{n}}{n} \sin n x\right] \quad(-\pi<x<\pi)$,在$ x=\pm \pi $处,级数收敛于$ \frac{\pi}{2} $,$\sum\limits_{n=1}^{\infty} \frac{1}{(2 n-1)^{2}}=\frac{\pi^{2}}{8}$
		
		
		
		
	\end{enumerate}
	
	
}

\newpage
\item 
(2 小题,每小题 12 分,共 24 分)
\begin{enumerate}
	%\renewcommand{\labelenumi}{\arabic{enumi}.}
	% A(\Alph) a(\alph) I(\Roman) i(\roman) 1(\arabic)
	%设定全局标号series=example	%引用全局变量resume=example
	%[topsep=-0.3em,parsep=-0.3em,itemsep=-0.3em,partopsep=-0.3em]
	%可使用leftmargin调整列表环境左边的空白长度 [leftmargin=0em]
	\item
设 $f(x)$ 在 $[0,1]$ 上连续,在 $(0,1)$ 上可导, $f(0)=0, f(1)=2$, 证明:
\begin{enumerate}
	%\renewcommand{\labelenumi}{\arabic{enumi}.}
	% A(\Alph) a(\alph) I(\Roman) i(\roman) 1(\arabic)
	%设定全局标号series=example	%引用全局变量resume=example
	%[topsep=-0.3em,parsep=-0.3em,itemsep=-0.3em,partopsep=-0.3em]
	%可使用leftmargin调整列表环境左边的空白长度 [leftmargin=0em]
	\item
 存在 $\xi \in(0,1)$ 使 $f(\xi)=1$。
\item 
存在 $0<x_{1}<x_{2}<1$, 使 $\frac{1}{f^{\prime}\left(x_{1}\right)}+\frac{1}{f^{\prime}\left(x_{2}\right)}=1$。
\end{enumerate}
	
	
\item 
\begin{enumerate}
	%\renewcommand{\labelenumi}{\arabic{enumi}.}
	% A(\Alph) a(\alph) I(\Roman) i(\roman) 1(\arabic)
	%设定全局标号series=example	%引用全局变量resume=example
	%[topsep=-0.3em,parsep=-0.3em,itemsep=-0.3em,partopsep=-0.3em]
	%可使用leftmargin调整列表环境左边的空白长度 [leftmargin=0em]
	\item
	求 $F(x)=\int_{0}^{a}|t-x| d t$ (常数 $a>0$ ) 在 $[0, a]$ 上的最小值。
	\item 
	设 $f(x)$ 在 $[0, a](a>0)$ 连续,并且 $\int_{0}^{a} f(x) d x=0, \int_{0}^{u} x f(x) d x=1$, 求证: 存 在一点 $x_{0} \in[0, a]$ 使得 $\left|f\left(x_{0}\right)\right| \geqslant \frac{4}{a^{2}}$。
	
	
\end{enumerate}


	
	
	
	
\end{enumerate}


\banswer{
	\begin{enumerate}
		%\renewcommand{\labelenumi}{\arabic{enumi}.}
		% A(\Alph) a(\alph) I(\Roman) i(\roman) 1(\arabic)
		%设定全局标号series=example	%引用全局变量resume=example
		%[topsep=-0.3em,parsep=-0.3em,itemsep=-0.3em,partopsep=-0.3em]
		%可使用leftmargin调整列表环境左边的空白长度 [leftmargin=0em]
		\item
		\begin{enumerate}
			%\renewcommand{\labelenumi}{\arabic{enumi}.}
			% A(\Alph) a(\alph) I(\Roman) i(\roman) 1(\arabic)
			%设定全局标号series=example	%引用全局变量resume=example
			%[topsep=-0.3em,parsep=-0.3em,itemsep=-0.3em,partopsep=-0.3em]
			%可使用leftmargin调整列表环境左边的空白长度 [leftmargin=0em]
			\item
			证明略
			\item 
			证明略
			
		\end{enumerate}
		\item 
		\begin{enumerate}
			%\renewcommand{\labelenumi}{\arabic{enumi}.}
			% A(\Alph) a(\alph) I(\Roman) i(\roman) 1(\arabic)
			%设定全局标号series=example	%引用全局变量resume=example
			%[topsep=-0.3em,parsep=-0.3em,itemsep=-0.3em,partopsep=-0.3em]
			%可使用leftmargin调整列表环境左边的空白长度 [leftmargin=0em]
			\item
			$ x=\frac{a}{2} $时,$ F(x) $有最小值$ \frac{a^{2}}{4} $
			\item 
			证明略
			
		\end{enumerate}
		
		
		
		
		
	\end{enumerate}
	
	
}


	
\end{enumerate}


