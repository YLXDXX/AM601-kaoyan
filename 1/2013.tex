\bta{2013}


\begin{enumerate}
	%\renewcommand{\labelenumi}{\arabic{enumi}.}
	% A(\Alph) a(\alph) I(\Roman) i(\roman) 1(\arabic)
	%设定全局标号series=example	%引用全局变量resume=example
	%[topsep=-0.3em,parsep=-0.3em,itemsep=-0.3em,partopsep=-0.3em]
	%可使用leftmargin调整列表环境左边的空白长度 [leftmargin=0em]
	\item
	(本题满分 10 分,每小题 5 分)
	\begin{enumerate}
		%\renewcommand{\labelenumi}{\arabic{enumi}.}
		% A(\Alph) a(\alph) I(\Roman) i(\roman) 1(\arabic)
		%设定全局标号series=example	%引用全局变量resume=example
		%[topsep=-0.3em,parsep=-0.3em,itemsep=-0.3em,partopsep=-0.3em]
		%可使用leftmargin调整列表环境左边的空白长度 [leftmargin=0em]
		\item
		函数 $f(x)$ 的导数 $f^{\prime}(x)$ 在 $(-\infty,+\infty)$ 上是连续函数, $a>0$, 则函数$ F(x) $
		\[
		F(x)=\left\{\begin{aligned}
			&a && f(x) \geqslant a \\
			&f(x) && -a<f(x)<a \\
			&-a && f(x) \leqslant-a
		\end{aligned}\right.
		\]
		一定是 \xzanswer{B} 
		
		
	\fourchoices
	{有界可微函数}
	{有界连续函数}
	{连续可微函数}
	{以上结论都不正确}

\item 	
$ 	\lim\limits _{n \rightarrow \infty}\left(\frac{1}{n^{2}+2 n+1}+\frac{2}{n^{2}+2 n+2}+\cdots+\frac{n}{n^{2}+2 n+n}\right)= $ \xzanswer{C} 

		
\fourchoices
{$ 1 $}
{$\infty$}
{$\frac{1}{2}$}
{$ 0 $}


\item 
函数 $f(x)=(x+2 \cos x)^{2}$ 在区间 $[0, \pi / 2]$ 上的最大值是 \xzanswer{C} 


\fourchoices
{$\frac{\pi^{2}}{36}+\frac{\sqrt{3} \pi}{3}+1$}
{$\frac{\pi^{2}}{36}+\frac{\sqrt{3} \pi}{3}+2$}
{$\frac{\pi^{2}}{36}+\frac{\sqrt{3} \pi}{3}+3$}
{$\frac{\pi^{2}}{4}$}





\item 
函数 $f(x)=x(x+1) \cdots(x+20)$, 下面四个结论正确的是 \xzanswer{D} 


\fourchoices
{$f^{\prime}(-1)>0, f^{\prime}(-2)>0$}
{$f^{\prime}(-1)>0, f^{\prime}(-2)<0$}
{$f^{\prime}(-1)<0, f^{\prime}(-2)<0$}
{$f^{\prime}(-1)<0, f^{\prime}(-2)>0$}

\item 
已知 $g(x) \cdot \int_{0}^{2} f(x) d x=10$, 则 $\int_{0}^{2} f(x) d x \cdot \int_{0}^{2} g(x) d x=$ \xzanswer{A} 


\fourchoices
{$ 20 $}
{$ 10 $}
{$ 5 $}
{不能确定}

\item 
$ \lim\limits _{\substack{ x \rightarrow 0 \\ y \rightarrow 0}} \frac{x y}{\sqrt[3]{x^{4}+y^{12}}}= $ \xzanswer{D} 


\fourchoices
{$ 0 $}
{$\frac{1}{\sqrt{2}}$}
{$\frac{1}{\sqrt[3]{2}}$}
{不存在}


\item 
$f(u)$ 区间内可导且 $f^{\prime}(u)>0, f(0)=0, L$ 为单位圆周 $x^{2}+y^{2}=1$ 被 $y=x$ 和 $y$ 轴所夹弧段, 则弧长的曲线积分 $c_{1}=\int_{L} f(2 x y) d s$ 和 $c_{2}=\int_{L} f\left(2 x^{2}-1\right) d s$ 满足 \xzanswer{B} 

\fourchoices
{$c_{1}>0, c_{2}>0$}
{$c_{1}>0, c_{2}<0$}
{$c_{1}<0, c_{2}>0$}
{$c_{1}<0, c_{2}<0$}

\item 
设二阶齐次常系数微分方程 $y^{\prime \prime}+a y^{\prime}+b y=0$ 的任一解 $y(x)$ 满足当 $x \rightarrow+\infty, y \rightarrow 0 $  的时候, 则实数 $a, b$ 满足 \xzanswer{A} 


\fourchoices
{$a>0, b>0$}
{$a>0, b<0$}
{$a<0, b>0$}
{$a<0, b<0$}

\item 
幂级数 $\sum\limits_{n=1}^{\infty} \frac{(x+1)^{n}}{\sqrt{n}}$ 的收敛域是 \xzanswer{A} 


\fourchoices
{$[-2,0)$}
{$(-2,0)$}
{$(-2,0]$}
{$[-2,0]$}

\item 
过点 $(0,0,1)$ 且与直线$\left\{\begin{array}{l}x=t+1 \\ y=-t-4 \\ z=2 t\end{array}\right.$及 $\frac{x-1}{-1}=\frac{y}{2}=\frac{z}{-1}$ 都平行的平面方程
为 \xzanswer{C} 


\fourchoices
{$5 x+2 y-z+1=0$}
{$5 x-y-3 z+3=0$}
{$3 x+y-z+1=0$}
{$-3 x-y+z+1=0$}


		
		
	\end{enumerate}



\item 	
(本题满分 10 分)	
计算 $\lim\limits _{x \rightarrow 0^{+}} x^{ (x^{x}-1)}$。


\banswer{
	$\lim\limits _{x \rightarrow 0^{+}} x^{ (x^{x}-1)}= e^{0}=1 $
}


\item 	
(本题满分 10 分)	
求微分方程 $y^{\prime \prime}=y^{\prime}(y-3)$ 满足初始条件 $y(0)=1, y^{\prime}(0)=-\frac{5}{2}$ 的解。

\banswer{
$ y=\frac{6}{1+5e^{3}} $,初次降阶后利用初始条件确定第一个积分常数。
}



\item 	
(本题满分 10 分)	
求函数 $f(x)=\pi^{2}-x^{2}$ 在区间 $[-\pi, \pi)$ 上的傅里叶级数。


\banswer{
	\[ 
	f(x)= \frac{ 2 }{ 3 } \pi^{2}+\sum\limits_{n=1}^{\infty} (-1)^{n+1}\frac{4}{n^{2}}   \cos nx 
	 \]
	在$ x=-\pi $处收敛于$ 0 $,在$ -\pi<x<\pi $处收敛于$ \pi^{2}-x^{2} $
}


\item 	
(本题满分 10 分)	
求曲面积分 $\iint_{S} x y dy d z+z^{2} d x d y$, 其中 $S$ 由 $z=\sqrt{x^{2}+y^{2}}$ $(0 \leqslant z \leqslant 1)$ 的上侧 (法向量与 $z$ 轴正向量夹角为锐角的一侧) 及 $z=1$ 的下侧围成的有向曲面。


\banswer{
	$ \iint_{S} x y dy d z+z^{2} d x d y=-\frac{\pi}{2} $
}



\item 	
(本题满分 10 分)	
假设函数 $f(x)$ 满足 $f(1)=1$ 且对于 $x \geqslant 1$,
\[ 
f^{\prime}(x)=\frac{1}{x^{2}+f^{2}(x)}
 \]
证明: $\lim\limits _{x \rightarrow +\infty} f(x)$ 存在,且不大于 $1+\frac{\pi}{4}$。

\banswer{
	证明略,构造函数$ g(x)=1-\frac{\pi}{4} + \arctan x $
}


\newpage
\item 	
(本题满分 10 分)	
设两个连续函数 $f, g$ 满足: 当 $x \in[0,1]$ 时, $f(x)+g(x) \neq 0$ 。证明存在唯一的数$ a $ $(0 \leqslant a \leqslant 1)$ 使得
\[
\int_{a}^{1}|f(x)| d x=\int_{0}^{a} g^{2}(x) d x
\]



\banswer{
	证明略,构造函数$ F(t)=\int_{0}^{t} g^{2}(x)dx-\int_{a}^{1} \left| f(x)\right| dx $
}


\item 	
(本题满分 10 分)	
证明:
\[
\lim _{x \rightarrow+\infty} \frac{1}{x} \int_{x}^{2 x}|\cos t| d t=\frac{2}{\pi}
\]


\banswer{
	证明略,利用夹逼法$ k\pi\leq x \leq (k+1)\pi $
}


\item 	
(本题满分 10 分)	
设 $f(x)=\frac{1}{1+x^{2}}-x e^{x} \int_{0}^{1} f(x) d x$, 求 $f(x)$ 和 $f^{\prime}(x)$ 。


\banswer{
	$ f(x)=\frac{1}{1+x^{2}} -\frac{\pi}{8} xe^{x}$,$ f(x) ^{\prime} =-\frac{2x}{(1+x^{2})^{2}}-\frac{\pi}{8}(1+x)e^{x} $
}


\item 	
(本题满分 10 分)	
函数 $f(x)$ 在 $[a, b]$ 上连续, 在 $(a, b)$ 内可导。证明:存在 $\xi, \eta \in(a, b)$, 使得
\[
f^{\prime}(\eta)=\left(b^{2}+a b+a^{2}+2\right) \frac{f^{\prime}(\zeta)}{3 \zeta^{2}+2}
\]


\banswer{
	证明略,利用柯西中值定理和拉格朗日中值定理,$ b^{3} -a^{3}=(b-a)(b^{2} + ab +a^{2}) $
}


\item 	
(本题满分 10 分)	
函数 $f(x)$ 在 $[0,2]$ 上二阶可导, 且对任意的 $x \in[0,2]$, 有 $|f(x)| \leqslant 1,\left|f^{\prime \prime}(x)\right| \leqslant 1$ 。 证明: 对于任意 $x \in[0,2],\left|f^{\prime}(x)\right| \leqslant 2$ 成立。

\banswer{
	证明略,在$ x $点处进行泰勒展开,放缩
}

	
	
\end{enumerate}


