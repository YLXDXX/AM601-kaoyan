\bta{2007}


\begin{enumerate}
	%\renewcommand{\labelenumi}{\arabic{enumi}.}
	% A(\Alph) a(\alph) I(\Roman) i(\roman) 1(\arabic)
	%设定全局标号series=example	%引用全局变量resume=example
	%[topsep=-0.3em,parsep=-0.3em,itemsep=-0.3em,partopsep=-0.3em]
	%可使用leftmargin调整列表环境左边的空白长度 [leftmargin=0em]
	\item
填空题 (本题满分 30 分,每小题 6 分)
\begin{enumerate}
	%\renewcommand{\labelenumi}{\arabic{enumi}.}
	% A(\Alph) a(\alph) I(\Roman) i(\roman) 1(\arabic)
	%设定全局标号series=example	%引用全局变量resume=example
	%[topsep=-0.3em,parsep=-0.3em,itemsep=-0.3em,partopsep=-0.3em]
	%可使用leftmargin调整列表环境左边的空白长度 [leftmargin=0em]
	\item
$\lim\limits _{x \rightarrow 0} \frac{\tan x-\sin x}{\ln \left(1+x^{3}\right)}$ \tk{$  \frac{ 1 }{ 2 }  $} 。


\item
设函数 $f(x, y)$ 可微, $f(0,0)=0, f_{x}^{\prime}(0,0)=m, f_{y}^{\prime}(0,0)=n, \phi(t)=f(t, f(t, t))$, 则 $\phi^{\prime}(0)=$ \tk{$m n+m+n^{2}$} 。


\item
    $\int \frac{d x}{1+\sqrt[3]{x+2}}=$ \tk{$\frac{3}{2}(x+2)^{\frac{2}{3}}-3(x+2)^{\frac{1}{3}}+3 \ln \abs{1+\sqrt[3]{x+2}}+C$} 。


\item
微分方程 $x^{2} y^{\prime}+x y=y^{2}$ 满足 $y(1)=2$ 的通解为 \tk{$ 2x $} 。

\nynote{伯努利方程}


\item
    设 $\Sigma$ 是曲面 $x^{2}+y^{2}+z^{2}=a^{2}$ 的外侧, $\cos \alpha, \cos \beta, \cos \gamma$ 是其外法向量的方向余 弦,则 $\iint_{\Sigma} \frac{x \cos \alpha+y \cos \beta+z \cos \gamma}{\left(x^{2}+y^{2}+z^{2}\right)^{\frac{3}{2}}} d S=$ \tk{$ 4\pi $} 。

\end{enumerate}

\item 
选择题 (本题满分 30 分,每小题 6 分)
\begin{enumerate}
	%\renewcommand{\labelenumi}{\arabic{enumi}.}
	% A(\Alph) a(\alph) I(\Roman) i(\roman) 1(\arabic)
	%设定全局标号series=example	%引用全局变量resume=example
	%[topsep=-0.3em,parsep=-0.3em,itemsep=-0.3em,partopsep=-0.3em]
	%可使用leftmargin调整列表环境左边的空白长度 [leftmargin=0em]
	\item
设 $f(x)=\left\{\begin{array}{ll}\frac{\sin (x-1)}{e^{x-1} - a}\left(\frac{1}{x}-b\right) & x \neq 1 \\ 2 & x=1\end{array}\right.$, 若 $f(x)$ 在 $x=1$ 处连续, 则 \xzanswer{B} 

	
\fourchoices
{$a=0, b=1$}
{$a=1, b=-1$}
{$a=-1, b=1$}
{$a=1, b=0$}

	
	
	\item
	设 $f(x), g(x)$ 都在 $x_{0}$ 处二阶可导, 且 $f\left(x_{0}\right)=g\left(x_{0}\right)=0, f^{\prime}\left(x_{0}\right) \cdot g^{\prime}\left(x_{0}\right)>0$, 则 \xzanswer{C} 
	
	
\fourchoices
{$x_{0}$ 不是 $f(x) \cdot g(x)$ 的驻点}
{$x_{0}$ 是 $f(x) \cdot g(x)$ 的驻点, 但不是 $f(x) \cdot g(x)$ 的极值点}
{$x_{0}$ 是 $f(x) \cdot g(x)$ 的驻点, 且是它的极小值点}
{$x_{0}$ 是 $f(x) \cdot g(x)$ 的驻点, 且是它的极大值点}

	
\item 
已知连续函数 $f(x)$ 满足 $f(x)=f(2 a-x)(a \neq 0), c$ 为任意常数, $\int_{-c}^{c} f(a-x) d x=$
 \xzanswer{D} 


\fourchoices
{$2 \int_{0}^{c} f(2 a-x) d x$}
{$2 \int_{-c}^{c} f(2 a-x) d x$}
{$ 0 $}
{$2 \int_{0}^{c} f(a-x) d x$}
	

\item 
点 $P_{1}(-2,3,1)$ 关于直线 $L: x=y=z$ 的对称点 $P_{2}$ 的坐标是 \xzanswer{D} 


\fourchoices
{$\left(-\frac{2}{3}, 1, \frac{1}{3}\right)$}
{$\left(\frac{2}{3},-1,-\frac{1}{3}\right)$}
{$\left(-\frac{10}{3}, \frac{5}{3},-\frac{1}{3}\right)$}
{$\left(\frac{10}{3},-\frac{5}{3}, \frac{1}{3}\right)$}

\nynote{
    取$P(1,1,1)$,则$\mathbf{OP}\times \mathbf{OP_1} = -\mathbf{OP}\times \mathbf{OP_2}$,且$PP_1=PP_2$,去掉$P_1$关于$P$的对称点,列方程可解。
}


\item
设 $f(x)$ 在区间 $[-\pi, \pi]$ 上连续, 且满足 $f(x+\pi)=-f(x)$, 则 $f(x)$ 的傅里叶系数 $a_{2 n}(n=1,2 \cdots)$ 等于 \xzanswer{A} 


\fourchoices
{$ 0 $}
{$\pi$}
{$\frac{1}{\pi}$}
{$\frac{4}{\pi}$}


	
\end{enumerate}



\item 
(本题满分 10 分)
已知 $f(x)$ 在 $(-\infty,+\infty)$ 内有二阶连续导数,且 $f(0)=0$, 又 
\[ 
\varphi(x)=\left\{\begin{aligned}&f^{\prime}(0) && x=0 \\ 
	&\frac{e^{x}}{x} f(x) && x \neq 0\end{aligned}\right.
 \]
 求 $\varphi(x) ^{\prime} $。

\nynote{$\varphi(0)$处的导数要用定义另外求。}
\banswer{
    \begin{equation*}
        \varphi^{\prime}(x)=
        \begin{cases}
            f^\prime (0) + f^{\prime\prime} (0),\qquad &x = 0\\
            \frac{x e^{x} f(x)+x e^{x} f^{\prime}(x)-e^{x} f(x)}{x^{2}},\qquad &x\neq 0\\
        \end{cases}
    \end{equation*}
}


\item 
(本题满分 10 分)
求满足 $x=\int_{0}^{x} f(t) d t+\int_{0}^{x} t f(t-x) d t$ 的可微函数 $f(x)$。


\nynote{
    化简等式可以得到$f'(x) + f(-x) = 0$,两边求导,化简得$f''(x) + f(x) = 0$,
    解方程,套每步求导之前的等式可定$C_1$、$C_2$。
}
\banswer{
	$f(x)= \cos x - \sin x$
}


\item 
(本题满分 10 分)
若 $u=f(z y x), f(0)=0, f^{\prime}(1)=1$, 且 $\frac{\partial^{3} u}{\partial x \partial y \partial z}=x^{2} y^{2} z^{2} f^{\prime \prime \prime}(x y z)$, 求函数 $u$。

\nynote{
    展开偏导数,令$t=xyz$可得$f'(t) + 3tf''(t) = 0$,解方程。
}
\banswer{
	$f(x y z)=\frac{3}{2}(x y z)^{\frac{2}{3}}$
}



\item 
(本题满分 10 分)
设 $L$ 是分段光滑的简单闭曲线,且点 $(2,0),(-2,0)$ 均在闭曲线 $L$ 所围成区域的内部, 计算曲线积分 $I=\oint_{L}\left[\frac{y}{(2-x)^{2}+y^{2}}+\frac{y}{(2+x)^{2}+y^{2}}\right] d x+\left[\frac{2-x}{(2-x)^{2}+y^{2}}+\frac{2+x}{(2+x)^{2}+y^{2}}\right] d y$, 其中 $L$ 取正向。

\nynote{见西安交通大学出版社《高等数学典型题》\textbf{10-45},积分里面是全微分,挖洞,斯托克斯公式。}

\banswer{
	$I=-4 \pi$
}



\item 
(本题满分 10 分)
求方程 $4 x^{4} y^{\prime \prime \prime}-4 x^{3} y^{\prime \prime}+4 x^{2} y^{\prime}=1$ 的形如 $y=a x^{-1}$ 的特解,进而求该方程的通解。

\nynote{
    欧拉方程,
    令$x = e^t$可化为常系数方程,
    先解齐次,再带上特解就可以了。
}
\banswer{
    $y=C_{1}+\left(C_{2}+C_{3} \ln x\right) x^{2}-\frac{1}{36}x^{-1}$
}


\newpage
\item 
(本题满分 10 分)
在曲线 $\frac{x^{2}}{4}+y^{2}=1$ 上找到一个位于第一象限的点, 使得该曲线在该点处的切线与 该曲线以及 $x$ 轴和 $y$ 轴所围成的图形面积最小, 并求此最小面积。

\nynote{可设切线为$ \frac{x}{a} + \frac{y}{b} = 1$,$a$、$b$为截距,借$\Delta=0$利用基本不等式求解。}

\banswer{
    $(\sqrt 2, \frac{\sqrt 2}{2})$,\quad
    $S_{min} = 2-\frac{\pi}{2}$
}



\item 
(本题满分 10 分)
当 $0<K=\sqrt{a^{2}+b^{2}}<r<R, \quad D: r^{2} \leqslant x^{2}+y^{2} \leqslant R^{2}$, 证明
\[
\frac{\pi\left(R^{2}-r^{2}\right)}{R+K} \leqslant \iint_{D} \frac{d \sigma}{\sqrt{(x-a)^{2}+(y-b)^{2}}} \leqslant \frac{\pi\left(R^{2}-r^{2}\right)}{r-K}
\]

\nynote{三角形三边关系不等式}

\banswer{
	证明略
}



\item 
(本题满分 10 分)
设函数 $f(x)$ 在区间 $[0,1]$ 上可导, 且 $f(0)=0, f(1)=1$, 证明在区间 $[0,1]$ 上存在 两点 $x_{1}, x_{2}$, 使得 $\frac{1}{f^{\prime}\left(x_{1}\right)}+\frac{1}{f^{\prime}\left(x_{2}\right)}=2$。

\nynote{2009年十一题的一种特殊情况,令$f(x_0)= \frac{1}{2}$,再用拉格朗日中值定理就可以了。}

\banswer{
	证明略
}


\item 
(本题满分 10 分)
设级数 $\sum\limits_{n=1}^{\infty} u_{n}$ 的各项 $u_{n}>0, n=1,2, \cdots,\left\{v_{n}\right\}$ 为一项正实数数列, 记 $a_{n}=\frac{u_{n} v_{n}}{u_{n+1}}-v_{n+1}$, 证明:如果 $\lim\limits _{n \rightarrow \infty} a_{n}=a$, 且 $a$ 为有限正数或者正无穷, 则 $\sum\limits_{n=1}^{\infty} u_{n}$ 收敛。


\banswer{
证明略	
}

	
	
	
\end{enumerate}


