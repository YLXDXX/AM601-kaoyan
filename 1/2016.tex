\bta{2016}

\begin{enumerate}
	%\renewcommand{\labelenumi}{\arabic{enumi}.}
	% A(\Alph) a(\alph) I(\Roman) i(\roman) 1(\arabic)
	%设定全局标号series=example	%引用全局变量resume=example
	%[topsep=-0.3em,parsep=-0.3em,itemsep=-0.3em,partopsep=-0.3em]
	%可使用leftmargin调整列表环境左边的空白长度 [leftmargin=0em]
	\item
	选择题 (本题满分 50 分,每小题 5 分)
	\begin{enumerate}
		%\renewcommand{\labelenumi}{\arabic{enumi}.}
		% A(\Alph) a(\alph) I(\Roman) i(\roman) 1(\arabic)
		%设定全局标号series=example	%引用全局变量resume=example
		%[topsep=-0.3em,parsep=-0.3em,itemsep=-0.3em,partopsep=-0.3em]
		%可使用leftmargin调整列表环境左边的空白长度 [leftmargin=0em]
		\item
		对于函数 $f(x)$, 当 $x$ 是有理数时, $f(x)=\sin \pi x$ ;当 $x$ 为无理数时, $f(x)=0$, 则正确的结论是 \xzanswer{D} 
		

\fourchoices
{$f(x)$ 在 $(-\infty,+\infty)$ 内无界}
{$f(x)$ 处处不连续}
{$f(x)$ 在 $(-\infty,+\infty)$ 上处处连续}
{$f(x)$ 在整数点上连续}

\item 
设 $f^{\prime}\left(\sin ^{2} x\right)=\cos ^{2} x$, 则 $f(x)$ 等于 \xzanswer{C} 


\fourchoices
{$x-\frac{1}{2} x^{2}$}
{$x+\frac{1}{2} x^{2}$}
{$x-\frac{1}{2} x^{2}-C$}
{$\frac{1}{2} x^{2}-x+C$}

\item 
设 $y_{1}$ 和 $y_{2}$ 均为微分方程 $y^{\prime \prime}+2 y^{\prime}+y=2 x e^{-x}$ 的特解,则如下论断不正确的为 \xzanswer{D} 


\fourchoices
{$\frac{1}{2} y_{1}+\frac{1}{2} y_{2}$ 可能形如 $\left(c_{1} x^{3}+c_{2} x^{2}\right) e^{-x}$, 其中 $c_{1}, c_{2}$ 为常数}
{$\frac{1}{2} y_{1}-\frac{1}{2} y_{2}$ 可能形如 $\left(c_{3} x+c_{4}\right) e^{-x}$, 其中 $c_{3}, c_{4}$ 为常数}
{$2 y_{1}+y_{2}$ 可能形如 $\left(c_{5} x^{3}+c_{6} x^{2}\right) e^{-x}$, 其中 $c_{5}, c_{6}$ 为常数}
{$2 y_{1}-y_{2}$ 可能形如 $\left(c_{7} x+c_{4}\right) e^{-x}$, 其中 $c_{7}, c_{8}$ 为常数}

		
\item 
$ \int_{0}^{\pi} \frac{x \sin x}{1+\cos ^{2} x} d x= $ \xzanswer{C} 


\fourchoices
{$\frac{\pi}{2}$}
{$\frac{\pi}{4}$}
{$\frac{\pi^{2}}{4}$}
{$\frac{\pi^{2}}{6}$}

\item 
已知实向量 $a, b$ 满足 $(a+b) \perp(a-b),|a+b|=1,|a|=1$, 则 $|a \times b|=$ \xzanswer{A} 


\fourchoices
{$\frac{\sqrt{3}}{2}$}
{$ 1 $}
{$\frac{\sqrt{2}}{2}$}
{$\frac{1}{6}$}

\item 
设 $f(x)$ 为连续函数, $F(t)=\int_{1}^{t} d y \int_{y}^{t} f(x) d x$, 则 $F^{\prime}(2)=$ \xzanswer{B} 


\fourchoices
{$2 f(0)$}
{$f(2)$}
{$-f(2)$}
{$ 0 $}

\item 
正项级数 $\sum\limits_{n=1}^{+\infty} a_{n} $ 收敛是级数 $\sum\limits_{n=1}^{+\infty} a_{n}^{2} $ 收敛的 \xzanswer{B} 


\fourchoices
{充要条件}
{充分条件,但非必要条件}
{必要条件,但非充分条件}
{既非充分条件,也非必要条件}

\item 
$ \lim\limits _{x \rightarrow 0, y \rightarrow 0} \frac{x^{2} y}{\sqrt[3]{x^{8}+y^{12}}}= $ \xzanswer{D} 


\fourchoices
{$ 0 $}
{$\frac{1}{\sqrt[3]{3}}$}
{$\frac{1}{\sqrt[3]{2}}$}
{不存在}

\item 
求 $\lim\limits _{x \rightarrow 0}\left(\frac{\sin }{x}\right)^{\frac{1}{x^{2}}}=$ \xzanswer{B} 


\fourchoices
{$e^{\frac{1}{6}}$}
{$e^{-\frac{1}{6}}$}
{$e^{\frac{1}{2}}$}
{$e^{-\frac{1}{2}}$}

\item 
若直线 $\left\{\begin{array}{l}x-y+z-1=0 \\ 2 x+y=0\end{array}\right.$ 与平面 $x+k y-z-5=0$ 平行,则 $k$ 的取值为 \xzanswer{C} 


\fourchoices
{$ -1 $}
{$ 0 $}
{$ 2 $}
{$ 1 $}




		
	\end{enumerate}
	


\item 
(本题满分 10 分)
若 $\lim\limits _{n \rightarrow \infty} x_{n}=a, \lim\limits _{n \rightarrow \infty} y_{n}=b$, 计算下列极限
\[
\lim\limits _{n \rightarrow \infty} \frac{x_{1} y_{n}+x_{2} y_{n-1}+\cdots+x_{n} y_{1}}{n}
\]
并写出具体过程。
\banswer{
	极限为$ ab $,过程略。\\
	这个是数学分析中一道常规的题目,需要用到数学分析中的一些结论。\\
	①$ \lim\limits _{n \rightarrow \infty} x_{n}=a $可等价描述为:存在无穷小序列$  \alpha_{n} $使得$ x_{n}=a+ \alpha _{n} $\\
	②记$ \beta_{n}=\frac{ \alpha _{1}+\cdots +  \alpha _{n}}{n} $,$ \beta_{n} $为无穷小序列
}


\item 
(本题满分 10 分)
求满足下列初始值的常微分方程 $y=y(x)$: 
\[
\left\{\begin{aligned}
	&y^{\prime \prime}=y^{\prime}(2 y+2) \\
	&y(0)=-1 \\
	&y^{\prime}(0)=1
\end{aligned}\right.
\]
的解。

\banswer{
	%满足初始条件的特解为
	$ y=\tan x -1 $
}


\item 
(本题满分 10 分)
一空间物体由球面 $(x-1)^{2}+y^{2}+z^{2}=1$ 的内侧和锥面 $x=\sqrt{y^{2}+z^{2}}$ 朝向 $x$ 轴 正向的一侧所界定, 其在点 $(x, y, z)$ 的密度为 $\rho(x, y, z)=1-\left(y^{2}+z^{2}\right)$, 试求该物体的质量。

\banswer{
	$ M=\frac{19}{30} \pi $。
	变换坐标系,可将球心放在$ (0,0,1) $点处,然后用球坐标系解;也可将球心放在原点$ (0,0,0) $处,在$ xoy $平面投影为圆。
\begin{note}
	锥面将球面分割成了两个部分,需要分清是哪个部分。
\end{note}
}


\item 
(本题满分 10 分)
将函数 $f(x)=(\pi-|x|)^{2}(-\pi \leqslant x \leqslant \pi)$ 展开成傅里叶级数,并求 $\sum\limits_{n=1}^{+\infty} \frac{1}{n^{2}}$ 的和。

\banswer{
	$f(x)=\frac{1}{3} \pi^{2}+\sum\limits_{n=1}^{\infty} \frac{4}{n^{2}} \cos n x$,利用$ f(0)=\pi^{2} $得
	$\sum\limits_{n=1}^{\infty} \frac{1}{n^{2}}=\frac{1}{6} \pi^{2}$
}


\item 
(本题满分 10 分)
设函数 $f(x)=\int_{0}^{x} g(t) d t$, 求 $\int_{0}^{1} e^{-x^{2}} f(x) d x$ 的值, 其中 $g(t)=5 t^{4}+3 t^{2}+1$。

\banswer{
	$\int_{0}^{1} e^{-x^{2}} f(x) d x= 2-\frac{4}{e} $
}


\newpage
\item 
(本题满分 10 分)
设在上半平面 $D=\{(x, y) \mid y>0\}$ 内, 函数 $f(x, y)$ 具有连续偏导数, 且对任意的 $t>0$ 都有 $f(t x, t y)=t^{-2} f(x, y)$ 。 证明:对 $D$ 内的任意分段光滑的有向简单闭曲线 $L$, 都有
\[
\oint_{L} y f(x, y) d x-x f(x, y) d y=0
\]

\banswer{
	证明略。跟位力定理中证明$ n $次齐次函数的特殊情况一样。
}


\item 
(本题满分 10 分)
计算曲面积分
\[
I=\iint_{\Sigma} x z d y d z+2 y z d z d x+3 x y d x d y
\]
其中, $\Sigma$ 为曲面 $z=1-x^{2}-\frac{1}{4} y^{2}$ $(0 \leqslant z \leqslant 1)$ 的上侧。

\banswer{
	$I= \pi $
}


\item 
(本题满分 10 分)
函数 $f(x)$ 在 $(0,\infty)$ 上可导, 且 $\lim\limits _{x \rightarrow \infty} f^{\prime}(x)=0$, 证明:
\[
\lim\limits _{x \rightarrow \infty} \frac{f(x)}{x}=0
\]

\banswer{
	证明略。构造函数$ f(x)=\int_{0}^{x} f ^{\prime} (x) dx + f(0) $,再分情况讨论,可证。
}


\item 
(本题满分 10 分)
已知函数 $f(x)$ 在 $(-\infty,+\infty)$ 上连续, 且 $F(x)=\frac{1}{2 \delta} \int_{-\delta}^{\delta} f(x+t) d t$, 其中 $\delta>0$。 证明: $F^{\prime}(x)$ 存在且连续, 其中 $x \in(-\infty,+\infty)$。


\banswer{
	证明略。先证$ F ^{\prime} (x) $存在,$F(x)=\frac{1}{2 \delta} \left[  \int_{-\delta}^{0} f(x+t) d t + \int_{0}^{\delta} f(x+t) d t \right] $,作变量替换$ k=x+t $,利用极限定义可证。再证$ F ^{\prime} (x) $连续。
}


\item 
(本题满分 10 分)
设函数 $f(x)$ 在 $[a, b]$ 上二阶可导, $f^{\prime}(a)=f^{\prime}(b)=0$ 。证明:存在一点 $\xi \in(a, b)$, 使 得
\[
\left|f^{\prime \prime}(\xi)\right| \geq \frac{4}{(b-a)^{2}}|f(b)-f(a)|
\]

	\banswer{
		证明略。将$ f(x) $在$ a $、$ b $两点用泰勒展开到二阶,再令$ x=\frac{a+b}{2} $,可证。
	}
	
	
	
\end{enumerate}


